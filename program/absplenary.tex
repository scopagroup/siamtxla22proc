\begin{center}
\refstepcounter{dummy}\label{plenary}
{\Large \bfseries Abstracts of Plenary Lectures}
\end{center}
\vspace{1ex}


\label{plenary1}
\plenabs
{Dr. Detlef Hohl}
{Chief Scientist, Computation and Data Science, Shell}
{Content-based image retrieval for industrial material with deep learning}
{Industrial materials images are an important application domain for content-based image retrieval (CBIR). Users need to quickly search databases for images that exhibit similar appearance, properties and/or features to reduce analysis turnaround time and cost. The images in this study are 2D images of millimeter-scale rock samples acquired at micrometer resolution with light microscopy and micro-CT. Labeled rock images are expensive and time consuming to acquire and thus are typically only available in the tens of thousands. Training a high-capacity deep learning (DL) model from scratch is therefore not practicable due to data paucity. To overcome this ``few shot learning'' challenge, we propose leveraging pre-trained common DL models in conjunction with transfer learning.  We present a novel DL architecture that combines Siamese networks with a loss function that integrates classification and regression terms. For efficient inference, we use a highly compressed image feature representation, computed offline, to search the database for images similar to a query image. Numerical experiments demonstrate superior retrieval performance of our new architecture compared with other DL and custom-feature based approaches.}
\bigskip





\label{plenary2}
\plenabs
{Dr. Carol Woodward}
{Distinguished Member of the Technical Staff, Center for Applied Scientific Computing, Lawrence Livermore National Laboratory}
{Time integration methods and software for scientific simulations}
{Time-dependent systems are at the heart of numerous scientific applications requiring simulation. While single rate, fixed step size time integration methods have been used for decades, adaptive step methods and schemes that can efficiently evolve problems with multiple time scales have not yet been fully engaged in many science applications. In this talk, I will overview current adaptive methods and discuss new multirate methods that address multiphysics problems. The SUNDIALS time integration software library will be presented as a vehicle for getting innovative numerical mathematics into applications. Lastly, I will present examples of use of SUNDIALS in scientific applications on state-of-the-art computers.\\
This work was performed under the auspices of the U.S. Department of Energy by Lawrence Livermore National Laboratory under Contract DE-AC52-07NA27344. Lawrence Livermore National Security, LLC. LLNL-ABS-834136.}
\bigskip


\label{plenary3}
\plenabs
{Dr. Karen E. Willcox}
{Director, Oden Institute for Computational Engineering and Sciences, The University of Texas at Austin}
{Beyond forward simulations: From reduced-order models to digital twins with computational science}
{Digital twins represent the next frontier in the impact of computational science on grand challenges across science, technology and society. A digital twin is a computational model or set of coupled models that evolves over time to persistently represent the structure, behavior, and context of a unique physical system or process. A digital twin is characterized by a dynamic and continuous two-way flow of information between the computational models and the physical system. This talk will highlight the important roles of reduced-order modeling and uncertainty quantification in achieving robust, reliable digital twins at scale. The talk will present our recent work on developing cancer patient digital twins in collaboration with the Oden Institute Center for Computational Oncology.}
\bigskip


\label{plenary4}
\plenabs
{Dr. Minh-Binh Tran}
{Assistant Professor, Department of Mathematics, Texas A\&M}
{Some recent results on wave turbulence theory}
{Wave turbulence describes the dynamics of both classical and non-classical nonlinear waves out of thermal equilibrium. In this talk, I will present our recent results on the rigorous justification of wave turbulence theory, starting from the stochastic Zakharov-Kuznetsov (ZK) equation, a multidimensional KdV type equation, on a hypercubic lattice. To the best of our knowledge, the work provides the first rigorous derivation of nonlinear 3-wave kinetic equations, for both homogeneous and inhomogeneous cases. Moreover, this is the first derivation for wave kinetic equations in the lattice setting and out-of-equilibrium. This is joint work with Gigliola Staffilani (MIT).}
