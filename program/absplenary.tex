\begin{center}
{\Large \bfseries Abstracts of Plenary Lectures}
\end{center}
\vspace{1ex}


\label{plenary1}
\abs
{{\bfseries Dr. Detlef Hohl}}{Chief Scientist, Computation and Data Science, Shell}
{Title}
{Abstract}
\bigskip


\label{plenary2}
\abs
{{\bfseries Dr. Carol Woodward}}{Project Leader, Center for Applied Scientific Computing, Lawrence Livermore National Laboratory}
{Title}
{Abstract}
\bigskip


\label{plenary3}
\abs
{{\bfseries Dr. Karen E. Willcox}}{Director, Oden Institute for Computational Engineering and Sciences, The University of Texas at Austin}
{Beyond forward simulations: From reduced-order models to digital twins with computational science}
{Digital twins represent the next frontier in the impact of computational science on grand challenges across science, technology and society. A digital twin is a computational model or set of coupled models that evolves over time to persistently represent the structure, behavior, and context of a unique physical system or process. A digital twin is characterized by a dynamic and continuous two-way flow of information between the computational models and the physical system. This talk will highlight the important roles of reduced-order modeling and uncertainty quantification in achieving robust, reliable digital twins at scale. The talk will present our recent work on developing cancer patient digital twins in collaboration with the Oden Institute Center for Computational Oncology.}
\bigskip


\label{plenary4}
\abs
{{\bfseries Dr. Minh-Binh Tran}}{Assistant Professor, Department of Mathematics, Texas A\&M}
{Some recent results on wave turbulence theory}
{Wave turbulence describes the dynamics of both classical and non-classical nonlinear waves out of thermal equilibrium. In this talk, I will present our recent results on the rigorous justification of wave turbulence theory, starting from the stochastic Zakharov-Kuznetsov (ZK) equation, a multidimensional KdV type equation, on a hypercubic lattice. To the best of our knowledge, the work provides the first rigorous derivation of nonlinear 3-wave kinetic equations, for both homogeneous and inhomogeneous cases. Moreover, this is the first derivation for wave kinetic equations in the lattice setting and out-of-equilibrium. This is joint work with Gigliola Staffilani (MIT).}

