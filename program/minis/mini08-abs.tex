\refstepcounter{dummy}\label{mini08}

\miniabs
{Sparse representation and model reduction for scientific problems}
{Organizers: Y. Lou, J. Hu \& Y. Yang}
{Many scientific problems involve predictive models of multi-scale and multi-physics systems, most of which are extremely complex and computationally expensive to simulate. Yet, a common feature of these problems is that there is often an underlying sparse or reduced representation which can greatly ease the heavy computation burden. This mini-symposium will bring researchers working on these relevant regimes to exchange ideas and encourage collaborations.}



\begin{addmargin}[2em]{0em}
\vspace{2ex}
\abs
{Blind image inpainting: from sparse multiscale representations to CNNs}
{Demetrio Labate}
{University of Houston}
{Image inpainting is an image processing task aimed at recovering missing blocks of data in an image or a video. In the first part of the talk, I will show that sparse multiscale representations offer a well-justified theoretical setting to address the image inpainting problem. Since images found in most applications are dominated by edges, I will assume a simplified image model consisting of distributions supported on curvilinear singularities. I will prove that the theoretical performance of image inpainting depends on the microlocal properties of the representation system, namely exact image recovery is achieved if the size of the missing singularity is smaller than the size of the structure elements of the representation system. As a result, shearlet-based image inpainting algorithms significantly outperforms traditional multiscale methods due to the superiour microlocal properties of shearlets. In the second part of the talk, I will show how to apply this theoretical observation to improve a state-of-the-art algorithm for blind image inpainting based on Convolutional Neural Networks.  This is a collaborative work with J. Schmalfuss, E. Scheurer, H. Zhao, N. Karantzas and A. Bruhn}


\vspace{1.5ex}
\abs
{Riemannian Optimization for Projection Robust Optimal Transport}
{Shiqian Ma}
{Rice University}
{The optimal transport problem is known to suffer the curse of dimensionality. A recently proposed approach to mitigate the curse of dimensionality is to project the sampled data from the high dimensional probability distribution onto a lower-dimensional subspace, and then compute the optimal transport between the projected data. However, this approach requires to solve a max-min problem over the Stiefel manifold, which is very challenging in practice. In this talk, we propose a Riemannian block coordinate descent (RBCD) method to solve this problem. We analyze the complexity of arithmetic operations for RBCD to obtain an $\epsilon$-stationary point, and show that it significantly improves the corresponding complexity of existing methods. Numerical results on both synthetic and real datasets demonstrate that our method is more efficient than existing methods, especially when the number of sampled data is very large.}


\vspace{1.5ex}
\abs
{Quantifying uncertainty at scale via structure-exploiting sparse grid approximation}
{Ionut-Gabriel Farcas$^{1}$, Gabriele Merlo$^{1}$ and Frank Jenko$^{2}$}
{1:The University of Texas at Austin, 2: Max Planck Institute for Plasma Physics}
{In many fields of science, remarkably comprehensive and realistic computational models are available nowadays.

Often, the respective numerical calculations call for the use of powerful supercomputers, and therefore only a limited number of cases can be investigated explicitly.

This prevents straightforward approaches to important tasks like uncertainty quantification and sensitivity analysis.

As it turns out, this challenge can be overcome via our recently developed sensitivity-driven dimension-adaptive sparse grid interpolation strategy.

The method exploits, via adaptivity, the structure of the underlying model (such as lower intrinsic dimensionality and anisotropic coupling of the uncertain inputs) to enable efficient and accurate uncertainty quantification and sensitivity analysis at scale.

We demonstrate the efficiency of our approach in the context of fusion research, in a realistic, computationally expensive scenario of turbulent transport in a magnetic confinement device with eight uncertain parameters and more than 264 million degrees of freedom in phase space, reducing the effort by at least two orders of magnitude.

In addition, we show that our method intrinsically provides an accurate surrogate model that is nine orders of magnitude cheaper than the high-fidelity model.}


\vspace{1.5ex}
\abs
{A general framework of rotational sparse approximation in uncertainty quantification}
{Mengqi Hu, Yifei Lou, and Xiu Yang}
{1: Lehigh University, 2: University of Texas at Dallas}
{This paper proposes a general framework to estimate coefficients of generalized  polynomial chaos (gPC) used in uncertainty quantification via rotational sparse approximation. In particular, we aim to identify a rotation matrix such that the gPC expansion of a set of random variables after the rotation has a sparser representation. However, this rotational approach alters the underlying linear system to be solved, which makes finding the sparse coefficients more difficult than the case without rotation.}


\vspace{1.5ex}
\abs
{Low-rank methods for solving high-dimensional collisional kinetic equations}
{Jingwei Hu}
{University of Washington}
{Kinetic equations describe the nonlinear equilibrium dynamics of a complex system using a probability density function. Despite their important role in multiscale modeling to bridge microscopic and macroscopic scales, numerically solving kinetic equations is computationally demanding as they lie in the six-dimensional phase space. Dynamical low-rank method is a dimension-reduction technique that has been recently applied to kinetic theory, yet most of the endeavor is devoted to linear or collisionless problems. In this talk, we introduce efficient dynamical low-rank methods for BGK and Vlasov-Fokker-Planck equations, building on certain prior knowledge about the low-rank structure of the solution. This talk is based on the joint work with J. Coughlin, L. Einkemmer, and L. Ying.}


\vspace{1.5ex}
\abs
{An Inverse Problem in Mean Field Games from Partial Boundary Measurement}
{Siting Liu}
{University of California Los Angeles}
{In this talk, we consider a novel inverse problem in mean-field games (MFG). We aim to recover the MFG model parameters that govern the underlying interactions among the population based on a limited set of noisy partial observations of the population dynamics under the limited aperture. Due to its severe ill-posedness, obtaining a good quality reconstruction is very difficult. Nonetheless, it is vital to recovering the model parameters stably and efficiently to uncover the underlying causes of population dynamics for practical needs.
	Our work focuses on the simultaneous recovery of running cost and interaction energy in the MFG equations from a finite number of boundary measurements of population profile and boundary movement. To achieve this goal, we formalize the inverse problem as a constrained optimization problem of a least squares residual functional under suitable norms. We then develop a fast and robust operator splitting algorithm to solve the optimization using techniques including harmonic extensions, three-operator splitting scheme, and primal-dual hybrid gradient method. Numerical experiments illustrate the effectiveness and robustness of the algorithm.
	A future direction will be to develop a faster algorithm for inverse problems in higher dimensions with the help of machine learning techniques and neural network architecture.}


\vspace{1.5ex}
\abs
{Adaptive deep density approximation for Fokker-Planck equations}
{Xiaoliang Wan}
{Department of Mathematics and Center for Computation and Technology, Louisiana State University, USA}
{In this talk we present an adaptive deep density approximation strategy based on KRnet for solving Fokker-Planck (F-P) equations. F-P equations are usually high-dimensional and defined on an unbounded domain, which limits the application of traditional grid based numerical methods. With the Knothe-Rosenblatt rearrangement, KRnet improves the normalizing flow based on real NVP and provides an explicit PDF model as an effective solution candidate for the Fokker-Planck equations. KRnet has a weaker dependence on dimensionality than traditional computational approaches and can efficiently approximate general high-dimensional density functions. To obtain effective stochastic collocation points for the training set, we develop an adaptive sampling procedure, where samples are generated iteratively using the approximate PDF given by KRnet. Numerical experiments will be presented.}


\vspace{1.5ex}
\abs
{Sparse representation of seismic data using chains of destructors}
{Sergey Fomel}
{Jackson School of Geosciences, the University of Texas at Austin}
{TBA}
\end{addmargin}
