\refstepcounter{dummy}\label{mini28}

\miniabs
{Modeling of air flow and droplet transport for biomedical applications}
{Organizers: Vladimir Ajaev \& Andrea Barreiro}
{Mathematical and computational modeling of air flow in the respiratory system is important for a range of applications including prevention of infectious diseases such as tuberculosis and COVID-19, understanding the functioning of the olfactory system, and improving therapeutic efficiency of drugs delivered by aerosol inhalation. The mini-symposium will focus on recent progress in the computational modeling of air flow in nasal cavity and respiratory airways, as well as the transport of microscale droplets which can carry infections or inhaled medication.    The physics of evaporation and condensation of such droplets will be discussed with implications to their trajectories and ultimate locations of their deposition. After deposition, the transport of components of droplet through mucus layer is of interest and can be studied using the methods of computational fluid mechanics. Potential applications of machine learning in combination with the standard numerical solutions of the Navier-Stokes equations, as well as other future research directions in the field will among the topics for the mini-symposium.}

\begin{addmargin}[2em]{0em}
\vspace{2ex}
%Session 1
\abs
{Building the Next-Generation Physiologically Realistic Human Respiratory Digital Twin System for Pulmonary Healthcare}
{Yu Feng}
{Oklahoma State University}
{Nowadays, ``personalized medicine'' is starting to replace the current ``one size fits all'' approach. The goal is to have the right drug with the right dose for the right patient at the right time and location. An example of personalized pulmonary healthcare planning is the targeted pulmonary drug delivery methodology. However, traditional in vitro and in vivo studies are limited and not sufficient for the precision medicine plan development. Specifically, due to the invasive nature and imaging limitations, most animal studies and clinical tests lack operational flexibility and cannot provide high-resolution patient-specific data. Therefore, alternative methods should be developed to conquer these bottlenecks. Models based on the computational fluid-particle dynamics (CFPD) method play a critical role in exploring alternate study designs and provide high-resolution data in a noninvasive, cost-effective, and time-saving manner. The in silico methodologies can fill the knowledge gap due to the deficiency of the traditional in vitro and in vivo methods, as well as make breakthroughs to pave the way to establish a reliable and efficient numerical investigation framework for pulmonary healthcare on a patient-specific level. A physiologically realistic CFPD-based human respiratory system modeling framework has been developed. This clinically validated elastic whole-lung model has been successfully applied to (1) Provide in silico evidence to regulatory authorities by evaluating the comparability between two dry powder inhalers to accelerate the approval process, (2) Optimize mitigation plans to reduce the airborne transmission of SARS-CoV-2 laden droplets in various indoor spaces, (3) Quantify the influence of lung disease progression on inhaled medication transport and delivery, (4) Predict the pharmacokinetics (PK) and human immune system responses after the inhaled medications or infectious virus deposited in the human respiratory system, (5) Assist the development of a new diagnostic method to detect airway obstruction location effectively, and (6) Assist on the development of personalized targeted pulmonary drug delivery to designated lung sites. It can make breakthroughs to establish a reliable and efficient computational modeling framework to drastically reduce the time to boost medical innovation in pulmonary healthcare and risk assessment on a subject-specific level, without compromising human subject safety.}


\vspace{1.5ex}
\abs
{Computational Prediction of Transport, Deposition, and Resultant Immune Response of Nasal Spray Vaccine Droplets to Potentially Prevent COVID-19}
{Hamideh Hayati$^1$, Yu Feng$^1$, Xiaole Chen$^2$, Emily Kolewe$^3$, Catherine Fromen$^3$}
{1:Oklahoma State University2:Nanjing Normal University, China, 3:University of Delaware}
{Intranasal vaccination against COVID-19 caused by SARS-CoV-2 could be highly advantageous over conventional intramuscular vaccination, which can benefit children who are afraid of needles. However, the effectiveness of the intranasal vaccine is highly dependent on the delivery dose to the designated site, i.e., the olfactory region (OR), which is the Angiotensin-converting enzyme 2-rich region. Many factors can influence the aerosolized vaccine transport in the nasal passage and the delivery dose, such as nasal spray cone angle, droplet initial velocity, and composition. Unfortunately, no computational effort has been made to investigate how administration strategies can influence the deposition of vaccine droplets in the nasal cavity, especially for children. This study focuses on the transport, deposition, and triggered immune response of intranasal vaccine droplets to the OR in the nasal cavity of a 6-year-old female to possibly prevent COVID-19. To investigate how administration strategy can influence the nasal vaccine delivery efficiency to OR, a validated multiscale model (i.e., computational fluid-particle dynamics (CFPD) and host-cell dynamics (HCD) model) was employed. Droplet deposition fraction, size change, residence time, and the area percentage of OR covered by the vaccine droplets and triggered immune system response were predicted with different spray cone angles, initial droplet velocities, and compositions. Numerical results indicate that droplet initial velocity and composition have negligible influences on the vaccine delivery efficiency to OR. In contrast, the spray cone angle can significantly impact vaccine delivery efficiency. The triggered immunity is not significantly influenced by the administration investigated due to the low percentage of OR area covered by the droplets. To enhance the effectiveness of the intranasal vaccine to prevent COVID-19 infection, it is necessary to optimize the vaccine formulation and administration strategy so that the vaccine droplets can cover more epithelial cells in OR to minimize the available receptors for SARS-CoV-2.}


\vspace{1.5ex}
\abs
{Analytical Solutions for Fluid Flow and Diffusion around a Slowly Condensing Levitating Liquid Droplet}
{Jacob E. Davis, Vladimir S. Ajaev}
{Southern Methodist University}
{Studies of interaction of microscale droplets with moist air flow are important for applications such as tranmission of infectious diseases and drug delivery by aerosol inhalation. We consider a slowly condensing droplet levitating near a surface of evaporating liquid and develop a mathematical model to describe diffusion, heat transfer, and fluid flow in the system. The method of separation of variables in bipolar coordinates is used to obtain series expansions for all physical quantities. This framework allows us to determine temperature profiles and condensation rates at the surface of the droplet. We find that the dependence of the equilibrium concentration on temperature is necessary to accurately model the phase change at the surface of the liquid. The condensation of vapor leads to the temperature in the droplet being, on average, higher than the surrounding air.Temperature and concentration gradients lead to significant differences between phase change rates at the top and bottom of the droplet. Fluid flow is described in terms of the Stokes stream function expressed in bipolar coordinates. Force acting on the droplet from the moist air flow is calculated as a function of distance between the droplet and the surface of evaporating liquid. Supported by NSF grant DMS-2009741.}


\vspace{1.5ex}
%Session 2 (3-3 talks)
\abs
{Predicting Transport and Deposition of Multicomponent E-cigarette Aerosols in a Subject-specific Airway Model with Different Nicotine Forms: An in silico Study}
{Ted Sperry, Yu Feng}
{Oklahoma State University}
{Predicting the transport and deposition of e-cigarette aerosols in the human respiratory system is essential to understanding how initial e-liquid compositions, especially different nicotine forms in new generations of e-cigarette products, can influence the absorption of nicotine in the human lung. Using a newly developed computational fluid dynamics (CFD) based numerical method which integrated the species transport and discrete phase (DPM) models, this study simulated and compared the transport dynamics of multicomponent e-cigarette aerosols in a subject-specific human respiratory system, and investigated how nicotine forms, nicotine mass fraction, PG/VG ratio, and acid/nicotine ratio in the e-liquids can influence the transport, evaporation/condensation dynamics, and deposition/absorption patterns in the human airway from mouth to generation 6 (G6). Specifically, the experimentally calibrated and validated CFPD model is able to predict the gas-liquid phase change dynamics of water, PG, VG, and nicotine in the aerosols during their transport through the pulmonary route. Freebase nicotine and nicotine salt were both investigated, with different PG/VG ratios and benzoic or lactic acids. Simulation results indicate that compared with free-based nicotine e-liquid, using nicotine salt with Benzoic and Lactate Acids will reduce the headspace nicotine saturation pressure, thereby reducing the evaporation rate of nicotine, which leads to lower nicotine absorption in the human upper airway and higher nicotine absorption in small airways. In addition, increasing PG/VG ratio will also reduce the headspace nicotine saturation pressure, further reducing the evaporation rate of nicotine. In summary, a CFD-based species transport-DPM model has been developed and validated to quantify how e-liquid composition can influence the transport, evaporation/condensation, and deposition/absorption of inhaled multicomponent e-cigarette aerosol in human respiratory systems. Future work will include (1) investigating how disease-specific lung airway deformation kinematics can influence the inhaled nicotine distributions, and (2) quantifying the pharmacokinetics of nicotine after deposition/absorption in the human body.}


\vspace{1.5ex}
%Session 2 (if 4-2 talks)
\abs
{Mathematical Modeling of Phase Change in Respiratory Droplets}
{Vladimir S. Ajaev, Art Taychameekiatchai, James Barrett}
{Southern Methodist University}
{Infectious diseases transmitted by tiny droplets of respiratory fluids affect tens of millions of people worldwide. Better understanding of the mechanism of transmission of infections can lead to improvements in both treatment and protection strategies. The focus of our work is on understanding how phase change processes such as evaporation and condensation affect the overall dynamics of respiratory droplets and the deposition of these droplets in the airways. We start by considering phase change in a spherical droplet surrounded by moist air. We find droplet radius over time in the framework of a model that accounts for the diffusion process, Stefan flow, presence of solutes, and changes in ambient humidity/temperature. Conditions are defined when droplets remain large enough to carry viable pathogens, but small enough to be rapidly carried by air. We then develop a lubrication-type model of sessile droplets containing solutes such as salt. The model incorporates  nonequilibirum effects during evaporation from the liquid surface and an increase of solute concentration as a result of solvent evaporation. The presence of solute leads to reduction of the evaporation rates at initial stages of evolution but the trend is reversed at the later stages, resulting in lower lifetimes of evaporating droplets.  Evaporative cooling is also considered in the framework that accounts for heat conduction in the substrate and shown to increase the droplet lifetime. Extensions of the model to the more realistic descriptions of respiratory droplets are discussed. Supported by NSF grant DMS-2009741.}


\vspace{1.5ex}
\abs
{Investigating the Impact of Mechanosensation on Retronasal Olfaction}
{Abdullah Saifee$^1$, Andrea Barreiro$^1$,Cheng Ly$^2$,Woodrow Shew$^3$}
{1:Southern Methodist University, 2:Virginia Commonwealth University, 3:University of Arkansas}
{Retronasal olfaction, which refers to the perception of odors that occurs during exhalation, accounts for partial flavor perception. But unlike orthonasal olfaction (smells during inhalation), retronasal olfaction is still rather poorly understood. There have been studies which indicate that even for identical odors, the brain activations that occur are different between the two types of olfaction. Prior experiments showed that synaptic inputs to the olfactory bulb from the nose are different for ortho- and retronasal olfaction in rats. How the nasal epithelium is stimulated for the two types of olfaction is still to be identified. Our hypothesis is that at the sensory periphery, the retro and ortho nasal stimuli produce distinct spatiotemporal patterns of mechanosensory excitation of olfactory receptor neurons. Experiments show that mechanical forces by air flow can lead to robust neural responses in the olfactory bulb. So it is likely that the airflow patterns will be quite different for retronasal flow compared to orthonasal flow, leading to different mechanical forces on olfactory receptor neurons (ORNs) and different input to olfactory bulb (OB). Ultimately, we will test this hypothesis with fluid dynamics simulations in experimentally obtained, animal-specific nasal cavity models.  Today we will show preliminary results demonstrating that shear stress forces differ for orthonasal vs. retronasal air flow; i.e., inspiration vs. exhalation, in an idealized model of a nasal cavity.}


\end{addmargin}


