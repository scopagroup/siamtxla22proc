\refstepcounter{dummy}\label{mini07}

\miniabs
{Spectral theory of Schrodinger operators and related topics}
{Organizers: W. Liu, R. Matos \& F. Yang}
{Spectral properties of Schrodinger operators have been intensely studied in the past decades due to their central relevance to quantum physics and also motivated many mathematical questions of independent interest. For instance, phase transitions in a physical system can often be detected through changes in the spectral types of the modeling operator. Schrodinger operators and techniques developed to understand their dynamics have also been key to the comprehension of transport properties of quasicrystals, crystals with random impurities, nonlinear Schrodinger equations, KDV equations, spin models in the presence of disorder, and many other math physics models.}

\begin{addmargin}[2em]{0em}
\vspace{2ex}
\abs
{Some spectral inequality for Schrödinger equations with power growth potentials.}
{Jiuyi Zhu}
{Louisiana State University}
{We obtain a sharp spectral inequality for Schrödinger equations with power growth potentials. This sharp spectral inequality depends on the radius and thickness of the sensor sets, and the growth rate of the potentials. The proof relies on quantitative global and local Carleman estimates to obtain quantitative three-ball inequalities.}


\vspace{1.5ex}
\abs
{1-Dim Half-line Schrödinger Operators with $H^{-1}$ Potentials}
{Xingya Wang}
{Rice University}
{In this talk, I will present some spectral results of Schrödinger operators with locally $H^{-1}$ potentials. In the first part, we will recover some general spectral theoretical results in the current setting, including Last-Simon-type criteria for the presence and absence of the absolutely continuous spectrum on the open half-line. In the second part, we will focus on potentials which are decaying in a locally $H^{-1}$ sense and present an analogue of short-range decay in the distributional setting. In particular, we will examine a class of Pearson-type distributional potentials and establish a spectral transition between short-range and long-range decay.}


\vspace{1.5ex}
\abs
{Continuity of the Lyapunov exponent for analytic multi-frequency quasi-periodic cocycles}
{Matthew Taylor Powell}
{UC Irvine}
{The purpose of this talk is to discuss our recent work on multi-frequency quasi-periodic cocycles, establishing continuity (both in cocycle and jointly in cocycle and frequency) of the Lyapunov exponent for non-identically singular cocycles. Analogous results for one-frequency cocycles have been known for over a decade, but the multi-frequency results have been limited to either Diophantine frequencies (continuity in cocycle) or $SL(2,\mathbb{C})$ cocycles (joint continuity). We will discuss the main points of our argument, which extends earlier work of Bourgain.}


\vspace{1.5ex}
\abs
{On the Irreducibility of Bloch and Fermi Varieties}
{Matthew Faust}
{Texas A\&M University}
{Understanding the irreducibility of Bloch and Fermi varieties for discrete periodic operators is important in the study of the spectrum of periodic operators, providing insight into the structure of spectral edges, embedded eigenvalues, and other applications.  In this talk we will present several new criteria for obtaining irreducibility of Bloch and Fermi varieties for infinite families of discrete periodic operators.}


\vspace{1.5ex}
\abs
{Nonlinear Schrodinger Equation with Growing Potential}
{Setenay Akduman}
{Texas A\&M University and Izmir Democracy University}
{This study deals with nonlinear Schrodinger (NLS) equations on infinite met-ric graphs.  Motivation for the study of NLS equation comes from nonlinearoptics and fiber optics.  If cables are branching, it is natural to look at quan-tum graphs.  Assuming the linear potential is infinitely growing, we prove anexistence of solutions that covers both self-focusing and defocusing cases.  Ourapproach is variational and based on generalized Nehari manifold. Thank Professor Alexander Pankov for his guidance, remarks and com-ments for the occurrence of this joint work.}


\vspace{1.5ex}
\abs
{On Pleijel's nodal domain theorem for quantum graphs}
{Matthias Hofmann}
{Texas A\&M University}
{We establish metric graph counterparts of Pleijel's theorem on the asymptotics of the number of nodal domains $\nu_n$ of the $n$-th eigenfunction(s) of a broad class of operators on compact metric graphs, including Schr\"{o}dinger operators with $L^1$-potentials and a variety of vertex conditions as well as the $p$-Laplacian with natural vertex conditions, and without any assumptions on the lengths of the edges, the topology of the graph, or the behaviour of the eigenfunctions at the vertices. Among other things, these results characterise the accumulation points of the sequence $(\frac{\nu_n}{n})_{n\in\mathbb{N}}$, which are shown always to form a finite subset of $(0,1]$. This extends the previously known result that $\nu_n\sim n$ \textit{generically}, for certain realisations of the Laplacian, in several directions. In particular, in the special cases of the Laplacian with natural conditions, we show that for graphs any graph with pairwise commensurable edge lengths and at least one cycle, one can find eigenfunctions thereon for which ${\nu_n} \not\sim n$; but in this case even the set of points of accumulation may depend on the choice of eigenbasis.\\
Joint work with James Kennedy, Delio Mugnolo, and Marvin Plümer.}



\vspace{1.5ex}
\abs
{Electrostatic partners (Stieltjes meets Hermite and Padé)}
{Andrei Martinez-Finkelshtein}
{Baylor University}
{The well-known electrostatic interpretation of the zeros of classical orthogonal polynomials goes back to the 1885 work of Stieltjes. It has been extended to several contexts, such as orthogonal and quasi-orthogonal polynomials on the real line and the unit circle, for classical and semiclassical weights.\\
Multiple orthogonal (or Hermite-Padé) polynomials satisfy a system of orthogonality conditions with respect to a set of measures. They find applications in number theory, approximation theory, and stochastic processes, and their analytic theory, extremely rich, has been developing since the 1980s. However, no electrostatic interpretation of the zeros of such polynomials was known. In this talk, I will present such a model for the case of type II Hermite-Padé polynomials. It introduces the notion of an “electrostatic partner” that allows a unified description of all the known cases.\\
This is joint work with R. Orive (Universidad de La Laguna, Canary Islands, Spain) and J. Sanchez-Lara (Granada University, Spain).}


\vspace{1.5ex}
\abs
{Complete Non-Selfadjointness for Schr\"odinger Operators on the Half-Line}
{Christoph Fischbacher}
{Baylor University}
{We investigate complete non-selfadjointness for all maximally dissipative extensions of a Schr\"odinger operator on a half-line with dissipative bounded potential and dissipative boundary condition. We show that all maximally dissipative extensions that preserve the differential expression are completely non-selfadjoint. However, it is possible for maximally dissipative extensions to have a one-dimensional reducing subspace on which the operator is selfadjoint. We give a characterization of these extensions and the corresponding subspaces and present a specific example.\\
(Joint work with Sergey Naboko and Ian Wood)}
\end{addmargin}
