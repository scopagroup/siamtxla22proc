\mini
{mini03}
{Mathematical modeling for biological dynamics}
{Organizers: Zhuolin Qu, Lale Asik \& Xiang-Sheng Wang}
{Mathematical models are powerful tools for understanding and informing complex biological phenomena. In recent years, there has been broad interest in applying mathematics to study a variety of biological fields, such as epidemiology, ecology, and neurology. Mathematical models at different spatial and temporal scales have been developed to focus on population-level dynamics, within-host processes, as well as multiscale dynamics that span several biological scales and capture the feedback between them. The utility of the proposed models requires a solid model formulation from realistic biological phenomena, rigorous analysis using mathematical theories, and accurately solved by numerical methods. This mini-symposium will highlight the new developments in these areas and bring together researchers who work on various models for biological systems from the perspectives of modeling, analysis, and computation. It will serve as a platform to present recent progress, exchange research ideas, extend academic networks, and seek future cooperation.}
{Location: CBB 106}

\begin{talks}
\item\talk
{Stochastic Avian Influenza Model}
{U. Bulut$^{1}$, T. Oraby$^{2}$, and E. Suazo$^{2}$}
{1: University of the Incarnate Word, 2: The University of Texas Rio Grande Valley}
\item\talk
{Multiscale Modeling of Infectious Disease: The Case of Malaria}
{Juan B. Guti\'errez$^{1}$}
{1: University of Texas at San Antonio}
\item\talk
{Reconciling contrasting effects of nitrogen on pathogen transmission and host immunity using stoichiometric models }
{ Dedmer B. Van de Waal $^{1,2}$, Lauren A. White $^{3}$, Rebecca Everett$^{4}$, Lale Asik$^{5,6}$, Elizabeth T. Borer$^{7}$, Thijs Frenken $^{1,8}$, Angélica L. González$^{9}$, Rachel Paseka$^{7}$, Eric W. Seabloom$^{7}$, Alexander T. Strauss $^{7,10}$, and Angela Peace $^{6}$}
{1: Department of Aquatic Ecology, Netherlands Institute of Ecology (NIOO-KNAW), Wageningen, The Netherlands, 2: Department of Freshwater and Marine Ecology, Institute for Biodiversity and Ecosystem Dynamics, University of Amsterdam, Amsterdam, The Netherlands, 3: National Socio-Environmental Synthesis Center (SESYNC), University of Maryland, Annapolis, MD, US, 4:Department of Mathematics and Statistics, Haverford College, Haverford, PA, USA, 5: Department of Mathematics and Statistics, University of the Incarnate Word, San Antonio, TX, USA, 6: Department of Mathematics and Statistics, Texas Tech University, Lubbock, TX, USA, 7: Department of Ecology, Evolution, and Behavior, University of Minnesota, St. Paul, MN, USA, 8: Great Lakes Institute for Environmental Research (GLIER), University of Windsor, Windsor, ON, Canada, 9: Department of Biology and Center for Computational and Integrative Biology, Rutgers University, Camden, NJ, US, 10: Odum School of Ecology. University of Georgia, Athens, GA, USA}

\item\talk
{Modeling Immunity to Malaria with an Age-Structured PDE Framework}
{Zhuolin Qu$^{1}$, Denis Patterson$^{2}$, Lauren Childs$^{3}$, Christina Edholm$^{4}$, Joan Ponce$^{5}$, Olivia Prosper$^{6}$, and Lihong Zhao$^{7}$}
{1: University of Texas at San Antonio, 2: Princeton University, 3: Virginia Tech, 4: Scripps College, 5: University of California, Los Angeles, 6: University of Tennessee, Knoxville, 7: University of California, Merced}
\end{talks}
\room
