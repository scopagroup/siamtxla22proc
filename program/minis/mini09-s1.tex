\mini
{mini09}
{Mathematical physics and graph theory}
{Organizers: R. Han, J. Fillman \& S. Shipman}
{Quantum mechanics is a beautiful combination of mathematical elegance and descriptive insight into the physical world. Many different models have been proposed to understand phenomenons in quantum mechanic, but much of the literature was devoted to various models defined on the standard Zd lattice. Over the recent years, there have been an increasing interest in other lattice structures, which arise from different crystals, e.g. the hexagonal lattice in graphene. This has led to a beautiful combination of mathematical physics and graph theory. The aim of this mini-symposium intends to bring in experts in the TX-LA area to discuss and exchange insights over the recent developments. The proposed mini-symposium will be focused on the following areas of mathematical physics: $\bullet$~multi-layer graphene and other crystal,  structures; $\bullet$~quantum graph models; $\bullet$~reducibility of Fermi surface and algebraic geometry; $\bullet$~critical points of periodic operators.}
{Location: CBB 104}

\begin{talks}
\item\talk
{Eigenvalue statistics for the disordered Hubbard model within Hartree-Fock theory}
{Rodrigo Matos}
{Texas A\&M University}
\item\talk
{A spectral statistic of quantum graphs without the semiclassical limit}
{Jon Harrison}
{Baylor University}
\item\talk
{Limit-Periodic Dirac Operators with Thin Spectra}
{Milivoje Lukić}
{Rice University}
\item\talk
{Analytic tongue boundaries and Cantor spectrum}
{Long Li}
{Rice University}
\end{talks}
\room
