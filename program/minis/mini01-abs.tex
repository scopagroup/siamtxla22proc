\refstepcounter{dummy}\label{mini01}

\miniabs
{Recent advances in large-scale inverse problems: Numerics, theory, and applications}
{Organizer: Alexander Mamonov, Andreas Mang \& Daniel Onofrei}
{Despite formidable advances in recent years, significant challenges remain. In inverse problems, parameters are typically related to indirect measurements by a mathematical model (for example, a PDE) in a highly nonlinear way, resulting in non-convex, non-linear optimization problems. These problems are challenging to solve in an efficient way. We will discuss recent advances in numerical methods and the theory of inverse problems to address these challenges. This minisymposium aims to attract researchers at the forefront of inverse problems, inference, and data science to present their latest work on fast algorithms and theory in inverse problems, and exciting applications.}


\begin{addmargin}[2em]{0em}
\vspace{2ex}
\abs
{Inverse problems of subsurface flows with low permeability fault structures}
{Jeonghun (John) Lee}
{Department of Mathematics, Baylor University}
{In this work we consider inverse problems of subsurface flow models with low permeability fault structures. We first consider deterministic inversion of fault transmissibility parameters via constrained optimization approach with Newton conjugate gradient methods. We then consider efficient Bayesian inversion using the Laplace approximation of posterior at MAP point. This is a joint work with Umberto Villa, Tan Bui-Thanh, Omar Ghattas at the Oden Institute of Computational Engineering and Sciences in University of Texas Austin.}

\vspace{1.5ex}
\abs
{An inverse solver for a multispecies tumor growth model}
{Ali Ghafouri \& George Biros}
{Oden Institute, The University of Texas at Austin}
{Biophysical models of tumor growth at the tissue level can be used for patient stratification, preoperative planning, treatment planning, and prognosis. They also help bridge imaging phenotype with molecular drivers of cancer. Here we focus on the mathematical structure of a parameter estimation problem for multi-species model of tumor growth. This model is designed for glioblastomas but in principle it can be used to other solid tumors.  We establish connections with classical inverse problem theory, and we discuss some unique challenges. It is a single-shot inversion (that is we using only one time snapshot), it has a large number of parameters, and involves nonlinear, non-differential operators. We propose an inversion methodology that attempts to address some of these challenges, and we present numerical results that illustrate the strengths and weaknesses of the proposed scheme.}


\vspace{1.5ex}
\abs
{Spatio-temporal quantification of pathological tau spreading in Alzheimer's disease}
{Zheyu Wen, Ali Ghafouri \& George Biros}
{Oden Institute, The University of Texas at Austin}
{Tau lesions (tau) are one of the main biomarkers of Alzheimer's disease (AD). Quantitatively describing how tau spreads in human brains can help with AD diagnosis and prognosis. Tau can be imaged spatially using positron emission tomography (tau-PET). Our goal is to use tau-PET images along with traditional magnetic resonance imaging to learn a spatio-temporal model of tau propagation. In this talk we will discuss the mathematical and computational challenges of the underlying methodology as well as a set of new algorithms that enable quantification and classification of tau spreading. We test our method on a cohort of subjects selected from publically available datasets.}


\vspace{1.5ex}
\abs
{Deep generative models and accelerated MRI}
{Brett Levac, Alexandros G. Dimaki \& Jon Tamir}
{Department of Electrical and Computer Engineering,  The University of Texas at Austin}
{Inverse problems are ubiquitous in computational sciences and engineering. Despite formidable advances in recent years, significant challenges still remain. In inverse problems, parameters are typically related to indirect measurements by a mathematical model (for example, a PDE) in a highly nonlinear way, resulting in non-convex, non-linear optimization problems. These problems are challenging to solve in an efficient way due to both complexity of the objective and the large amounts of collected data. We will discuss recent advances in numerical methods and theory of inverse problems to address these challenges. This minisymposium aims to attract researchers at the forefront of inverse problems, inference, and data science to present their latest work on fast algorithms and theory in inverse problems, and exciting applications.}


\vspace{1.5ex}
\abs
{Some results on inverse problems to elliptic {PDE}s with solution data and their implications in operator learning}
{Kui Ren}
{Department of Applied Physics and Applied Mathematics and Data Science Institute, Columbia University}
{In recent years, there have been great interests in discovering structures of partial differential equations from given solution data. Very promising theory and computational algorithms have been proposed for such identification problems in different settings. We will try to review some recent understandings of such PDE learning problems from the perspective of inverse problems. In particularly, we will highlight a few computational and analytical understandings on learning a second-order elliptic PDE from single and multiple solutions.}


\vspace{1.5ex}
\abs
{Estimating the noise level in seismic data while overcoming cycle skipping}
{Susan Minkoff$^{1}$, Huiyi Chen$^{1}$ \& William Symes$^{2}$}
{1: Department of Mathematical Sciences, University of Texas at Dallas, 2:  Department of Computational and Applied Mathematics, Rice University}
{Full waveform inversion (FWI) suffers from the well-known cycle skipping problem in which local gradient-based optimization may fail to converge to geologically meaningful earth models if the initial guess for the optimization is not close enough to the true earth model. Extension methods attempt to overcome the cycle skipping problem by enlarging the space of acceptable models. In the case of source extension, the inverse problem is extended to allow for estimation of both the medium parameters (velocity) and the source wavelet via the addition of a penalty term. The extension does not require the source to be compactly supported in time. This extended objective function can be minimized efficiently  via the discrepancy principle in which the source time function and velocity are estimated in a nested loop. While this approach has been shown to overcome cycle skipping, it does require use of a well-tuned penalty weight which depends on the data noise level, which is generally unknown. In this talk I will describe and illustrate an algorithm to simultaneously solve the inverse problem while accurately estimating the data noise level.}


\vspace{1.5ex}
\abs
{Conductivity imaging from thermal noise}
{Trent DeGiovanni \& Fernando Guevara Vasquez}
{University of Utah}
{We present a method for imaging the conductivity of a body from measurements of thermal noise currents between the body and the ground. Concretely we show that if the variances of the thermal noise currents are known, the inverse problem can be formulated as a deterministic internal functional inverse problem identical to the one occurring in ultrasound modulated electrical impedance tomography.}


\vspace{1.5ex}
\abs
{{L}ippmann--{S}chwinger--{L}anczos algorithm for inverse scattering problems}
{V. Druskin$^{1}$, S. Moskow$^{2}$ \& M. Zaslavsky$^{3}$}
{1: Department of Mathematical Sciences, Worcester Polytechnic Institute, 2: Department of Mathematics, Drexel University, 3: Schlumberger-Doll Research Center}
{Data-driven reduced order models (ROMs) are combined with the Lippmann[1]Schwinger integral equation to produce a direct nonlinear inversion method. The ROM is viewed as a Galerkin projection and is sparse due to Lanczos orthogonalization. Embedding into the continuous problem, a data-driven internal solution is produced. This internal solution is then used in the Lippmann-Schwinger equation, thus making further iterative updates unnecessary. We show numerical experiments for spectral domain data for which our inversion is far superior to the Born inversion and works as well as when the true internal solution is known.}




\vspace{1.5ex}
\abs
{Carleman Weighted Hilbert Spaces for Coefficient Inverse Problems}
{Michael V. Klibanov}
{Department of Mathematics, University of North Carolina at Charlotte}
{Coefficient Inverse Problems (CIPs) are both ill-posed and highly nonlinear. These two factors cause the non-convexity of conventional least squares cost functionals, which are constructed for numerical solutions of CIPs. The speaker with coauthors has developed a new approach to numerical solutions of CIPs, called convexification. The convexification constructs globally strictly convex cost functionals for a broad class of CIPs. This functional is defined on a bounded convex set of an arbitrary but fixed diameter in an appropriate Hilbert space, which we call Carleman Weighted Hilbert Space. The weight is the Carleman Weight Function, which is used in the Carleman estimate for a corresponding PDE operator. Uniqueness and existence of the minimizer of such a functional on that set is established. Convergence of minimizers to the true solution of the CIP is proven, provided that the noise in the data tends to zero. Many numerical examples, including ones with experimentally collected data, confirm the theory.\\
Some of these results will be presented in my talk.\\
Main contributors are (in the alphabetical order): Vo Khoa, Thuy Le and Loc Nguyen.}
% Preferred date: Saturday, November 5


\vspace{1.5ex}
\abs
{New sampling indicator functions for stable imaging of photonic crystals}
{Dinh-Liem Nguyen}
{Kansas State University}
{This talk is concerned with the inverse problem of determining the shape of penetrable periodic scatterers using electromagnetic waves. This inverse problem arises from the imaging of photonic crystals using electromagnetic inverse scattering. We develop a sampling method with a novel indicator function for solving this inverse problem. This indicator function is very simple to implement and robust against noise in the data. The resolution and stability analysis of the indicator function is analyzed. Our numerical study shows that the proposed sampling method is more stable than the factorization method and more efficient than the direct or orthogonality sampling method in reconstructing periodic scatterers. This is based on joint work with Kale Stahl and Trung Truong.}
% Preferred date: Saturday, November 5 or Friday, November 4


\vspace{1.5ex}
\abs
{Reconstructing a space-dependent source term of the Helmholtz equation via the quasi-reversibility method}
{Loc Hoang Nguyen}
{Department of Mathematics and Statistics, University of North Carolina at Charlotte}
{Our aim is to solve an important inverse source problem which is the linearization of the well-known inverse scattering problem. We propose to truncate the Fourier series of the solution to the governing equation with respect to a special basis of $L^2$. By this, we obtain a system of linear elliptic equations. Solutions to this system are the Fourier coefficients of the solution to the governing equation. After computing these Fourier coefficients, we can directly find the desired source function. Numerical examples are presented.}
% Preferred date: Saturday November 5


\vspace{1.5ex}
\abs
{Active control of electromagnetic fields in layered media}
{Chaoxian Qi$^{1,*}$, Neil Jerome A. Egarguin$^2$, Daniel Onofrei$^3$ \& Jiefu Chen$^1$}
{1: Department of Electrical and Computer Engineering, University of Houston, 2: Institute of Mathematical Sciences and Physics, University of the Philippines Los Ba\~nos, 3: Department of Mathematics, University of Houston}
{In this talk, we consider the problem of actively manipulating electromagnetic (EM) fields in layered media. We aim to characterize an EM source given some predetermined desired field patterns in prescribed disjoint exterior regions. The source characterization problem is treated as an inverse problem that requires solving an ill-posed optimization problem. The optimal current distribution is sought after such that the EM source can approximate the given EM fields in exterior regions. This study considers the case when the source and control regions are in a layered media, which can model various applications, such as enhanced subsurface sensing, radio communication through the water-air interface, etc.}



\vspace{1.5ex}
\abs
{Fast approximations of high-rank Hessians: Applications to seismic inversion and uncertainty quantification}
{Mathew Hu$^{1,*}$, Nick Alger$^{1}$, Longfei Gao$^{1}$, Omar Ghattas$^{1}$ \& Rami Nammour$^{2}$}
{1: Oden Institute, The University of Texas at Austin; 2: Total E\&P Research and Technology}
{We propose a high-rank, computationally efficient method to approximate the normal operator (Hessian) arising in the linearized seismic reflection inversion problem in two spatial dimensions. Computing the exact Hessian is intractable for large-scale problems. In fact, performing the Hessian vector product requires solving PDE problems. Moreover, slow decay of the Hessian's eigenvalues means that low-rank approximation of the Hessian is costly.\\
Our first approach is based on the locality of the Hessian impulses. The approximation includes two steps. First, apply Hessian on several particular vectors to draw information of impulses, and modify the impulses on boundaries if necessary. Second, build the operator with impulses and preset weight functions by a product-convolution scheme. Our second approach is based on the theory that the Hessian is a pseudodifferential operator under certain conditions. The approximation also includes two steps. First, apply the Hessian on a specific probing vector. Second, determine the symbol of the pseudodifferential operator by solving a smaller size optimization problem.\\
Both approaches approximate the Hessian operator within a small number of applications.\\
The approximation can be expressed as an H-matrix, which facilitates fast matrix computations, such as inversion, to obtain a preconditioner. We used the inverse of the approximation as a preconditioner in the second-order Newton-Krylov methods applied to full-waveform inversion (FWI). The preconditioner is highly desirable here because such preconditioners can reduce the required number of Krylov iterations within each Newton step, thereby reducing the computational cost.\\
We validate the method on the Marmousi model with constant density. Numerical experiments demonstrate that the preconditioner can reduce the computational costs substantially.}



\vspace{1.5ex}
\abs
{Matrix-free PSF approximation of ice-sheet Hessians}
{Nick Alger$^{1,*}$, Tucker Hartland$^{2}$, Noemi Petra$^{2}$, Omar Ghattas$^{1}$}
{1: Oden Institute, The University of Texas at Austin; 2: Applied Mathematics, University of California, Merced}
{We present an efficient matrix-free point spread function (PSF) method for approximating operators that have locally supported non-negative integral kernels. The method uses impulses of the operator (computed at a collection of scattered points) and interpolates translated and scaled versions of these impulses to approximate entries of the integral kernel. Impulse responses are computed by applying the operator to a small number of Dirac combs associated with ``batches'' of point sources. By solving an ellipsoid packing problem, we choose as many point sources as possible per batch, while ensuring that the supports of the impulses within each batch do not overlap. Support ellipsoids are estimated a-priori via a new procedure that involves applying the operator to a small number of polynomial functions. The ability to rapidly evaluate kernel entries allows us to construct a hierarchical matrix (H-matrix) approximation of the operator. Fast H-matrix methods are then used to perform further matrix computations in nearly linear complexity. We illustrate our approach by applying the method to approximate the Hessian in an ice sheet flow PDE constrained inverse problem. Our results show that the method can accurately approximate the high rank ice sheet Hessian using only a small number of operator applies.}

\end{addmargin}
