\refstepcounter{dummy}\label{mini19}

\miniabs
{Modeling the heart-brain axis and age-related pathology}
{Organizers: Travis Thompson}
{The methodical study of human anatomy dates back to at least the 16th century when the Belgian physician Andreas Vesalius published his seminal work, ``De humani corporis fabrica libri septem''.  We now understand a good deal about the large-scale mechanics and function of the many organs and systems within the human body and clinical progress has greatly extended our life spans.  Extended life spans have led to new concerns, including the need to more fully understand age-related pathologies such as heart and brain diseases.

The heart and brain are central to the study of human physiology and pathology.  Increasing evidence implicates the heart-brain axis in several age-related diseases and disorders, including heart failure, epilepsy, stroke and dementia, among others.  The in-vivo study of the heart and brain is often invasive and impractical.  Mathematical modeling, using numerical methods and medical imaging, provides an alternative means to noninvasively study the heart and brain in humans.

This minisymposium brings together an interdisciplinary community of mathematicians and medical researchers who are designing and using mathematical models, numerical methods, machine learning and imaging techniques to study important topics towards developing an understanding of the heart-brain axis and its relationship to age-related pathology, including: cardiomechanics; circulation; oxygen transport; neural network activity; neuroglia and neurodegenerative diseases.}


\begin{addmargin}[2em]{0em}
\vspace{2ex}
%Session 1
\abs
{Senescence, Sangre, Senility and Simulation: Mathematics at the intersection of the heart and the brain}
{Travis B.~Thompson$^{1}$}
{1: Department of Mathematics and Statistics, Texas Tech University}
{The heart and the brain have long been recognized as the crucible of human life.  The two are bonded together; one delivers vital nutriments to the body while the other provides the vivifying signaling milieu.   The importance of both organs was recognized as early as the 5th century BC.  Nearly a century later, the careful anatomical studies of Ibn al-Nafis, Andreas Vesalius and William Harvey laid the methodological groundwork for the modern era of quantitative medicine.  Renaissance anatomists described the heart and central nervous system in terms of mechanical principles such as pumps, causeways, and electrical impulses ruled by the principles of physics and intelligible in terms of mathematics.

Understanding the functional and pathological relationships of the heart-brain axis is of contemporary clinical importance.  The kinship between these sister organs leads to strongly associated risk factors, especially with age, for several of the leading causes of death worldwide.  Animal models provide a mechanistic basis for understanding these risk factors while mathematical methods offer an ethical, non-invasive approach for analysis and for clinical testing in simulated human environments.
In this talk, a perspective on the importance of the heart-brain axis, and aging, will be presented and connections between vascular factors, brain and especially neurodegenerative pathology will be discussed.  Along the way, we will see an example of how mathematical modeling can be used, alongside patient-specific medical imaging data, to construct a mechanistic model of neurodegenerative disease progression that highlights how the heart of mankind can affect the aging mind.}


\vspace{1.5ex}
\abs
{Multi-scale computational models of cardiac and brain tissues}
{Michael S. Sacks$^{1}$ and David S. Li$^{1}$}
{1:Willerson Center for Cardiovascular Modeling and Simulation, Oden Institute and the Department of Biomedical Engineering, University of Texas at Austin }
{Complex organ systems such as the heart and brain are constructed and function on many structural scales.  To gain deeper insights into their function, multi-scale models have been proposed. Recent developments in high resolution microscopic imaging and computational technology has lead to the ability to more fully simulate 3D structures at larger scales from smaller scale features.  To demonstrate this approach, we developed a high-fidelity, micro-anatomically realistic 3D finite element model of right ventricle free wall (RVFW) myocardium by combining high-resolution imaging and supercomputer-based simulations. We first developed a representative tissue element (RTE) model at the sub-tissue scale by specializing the hyperelastic anisotropic structurally-based constitutive relations for myofibers and ECM collagen, and equi-biaxial and non-equibiaxial loading conditions were simulated using the open-source software FEniCS to compute the effective stress-strain response of the RTE. To estimate the model parameters of the RTE model, we first fitted a 'top-down' biaxial stress-strain behavior with our previous structurally based (tissue-scale) model, informed by the measured myofiber and collagen fiber composition and orientation distributions. Next, we employed a multi-scale approach to determine the tissue-level (5 x 5 x 0.7 mm specimen size) RVFW biaxial behavior via 'bottom-up' homogenization of the fitted RTE model, recapitulating the histologically measured myofiber and collagen orientation to the biaxial mechanical data. Our homogenization approach successfully reproduced the tissue-level mechanical behavior of our previous studies in all biaxial deformation modes, suggesting that the 3D micro-anatomical arrangement of myofibers and ECM collagen is indeed a primary mechanism driving myofiber-collagen interactions.  We discuss how similar approaches can be used for brain tissue.mechanics.}


\vspace{1.5ex}
\abs
{Oscillopathies of Brain and Heart: Lessons From the Computational Medicine Clinic}
{David Paydarfar$^{1,2}$}
{1: Dell Medical School, Mulva Clinic for the Neurosciences 2: Oden Institute for Computational Engineering and Sciences,
The University of Texas at Austin}
{Abnormal oscillations are implicated in many disease states of the brain and heart. Examples are rhythmic discharges of neurons in epilepsy and reentrant excitation underlying ventricular tachycardia in heart disease. Biological oscillations can also be vital to normal physiology and disease states result from their loss of rhythmicity. For example, preterm infants commonly suffer from bouts of severe apnea and bradycardia due to immaturity of brainstem control of cardio-respiratory function. In this talk, Dr. Paydarfar will present theoretical, experimental, and clinical observations on the initiation and termination of neural and cardiac rhythms at the cellular, tissue and organism levels. Mathematical and computational methods can provide important insights into the etiology, prevention and treatment of oscillopathies.}


\vspace{1.5ex}
%Session 2
\abs
{Reduced models for solute transport and numerical convergence of solutions of PDEs with line source}
{Beatrice Riviere$^{1}$ and Charles Puelz$^{2}$}
{1: Dept.~of Comp.~and Appl.~Mathematics, Rice University 2: Dept.~of Pediatrics, Baylor College of Medicine}
{The modeling of solute exchanges between an organ and its vasculature is a key component  in understanding
treatment of diseases. In this talk, we first discuss reduced models of solute transport in blood vessels of
varying cross-section and with arbitrary axial velocity profile.  The numerical discretization uses a locally
implicit discontinuous Galerkin method. Second, the numerical analysis of elliptic and parabolic partial differential equations
with line source is shown for the finite element and discontinuous Galerkin methods. Optimal convergence of the numerical
method is recovered away from the line source.}


\vspace{1.5ex}
\abs
{Hierarchical Modular Structure of the Drosophila Connectome}
{Alexander B. Kunin$^{1,2}$, Jiahao Guo$^{1}$, Kevin E. Bassler$^{1}$, Xaq Pitkow$^{2,3}$ and Krešimir Josić$^{1}$ }
{1: University of Houston 2: Baylor College of Medicine 3: Rice University}
{The organization of neural circuitry in the brain plays a crucial role in brain function. Previous studies of the organization of the brain have generally had to trade off between coarse descriptions at a large scale and fine descriptions at small scales. We now have reconstructions of tens to hundreds of thousands of neurons at synaptic resolution, enabling investigations into the interplay between global, modular organization, and cell type-specific wiring. To do so we have applied novel community detection methods to analyze the Hemibrain data set, a synapse-level reconstruction of 21 thousand neurons and over 3.5 million synaptic connections in the brain of Drosophila. We develop an understanding of the complex structure composing the Drosophila brain by first finding the community structure at the largest scale. We then resolve the structure at increasingly finer scales, finding that the brain is organized hierarchically with smaller structures consisting mostly of subdivided larger-scale communities. The process continues until the communities become small enough for biological identification. Our methods identify well-known features of the fly brain's sensory pathways. For example, manual efforts have identified a layered structure in the fan-shaped body, with the ventral six layers thought to play a role in navigation, and the dorsal layers playing a role in sleep and modulating internal state. Our methods not only automatically recover this layered structure, but also find that the ventral two layers are distinguished from the remaining layers and from each other, reflecting distinct connectivity patterns to downstream and upstream areas. These methods show that the fine-scale, local network reconstruction made possible by modern experimental methods are sufficiently detailed to identify both large and small scale organizational features of the brain.}


\vspace{1.5ex}
\abs
{A deep learning framework for the automated detection and morphological analysis of GFAP-labeled astrocytes in micrographs}
{Demetrio Labate$^{1}$, Yewen Huang$^{1}$, Anna Kruyer$^{2}$, Sarah Syed$^{1}$, Cihan Kayasandik$^{3}$, Manos Papadakis$^{1}$}
{1: Department of Mathematics, University of Houston 2:Medical University of South Carolina 3:Istanbul Medipol University}
{Astrocytes, a subtype of glial cells with a complex star-shaped morphology, are active players in many aspects of the physiology of the central nervous system (CNS) where they provide structural and functional support to neurons. Astrocytes and neurons form tripartite synapses to establish bidirectional communications, with astrocytes modulating neuronal synaptic currents. However, there is still a major knowledge gap in understanding the highly complex role of astrocytes within the CNS, which is manifested by their morphological complexity, heterogeneity and unique ability to change size and shape. The current knowledge gap is due in part to the limitations of existing image analysis algorithms that are unable to detect and analyze astrocytes with sufficient accuracy and efficiency.  To address this limitation, we introduce a new computational framework for the automated detection and segmentation of GFAP-immunolabeled astrocytes in brightfield or fluorescent micrographs. Our novel approach integrates a deep learning framework with ideas from multiscale representation to provide both accurate and interpretable results. Extensive numerical experiments using multiple image datasets show that our method performs very competitively against both conventional and state-of-the-art methods, including images where astrocytes are very dense. The ability to automatically detect astrocytes and extract reliable information about their morphology has important practical implications including the development of efficient algorithms for astrocyte analysis and classification that will advance the understanding of the role of astrocytes in the physiology of the CNS and its pathologies.  In the spirit of reproducible research, our numerical code and annotated data are available open source and freely available to the scientific community.}


\vspace{1.5ex}
\abs
{Robust Incorporation of DTMRI Data in Soft Tissue Modeling}
{Christian Goodbrake$^{1}$, Kenneth Meyer$^{1}$, Jack Hale$^{2}$ and Michael S. Sacks$^{1}$}
{1: Oden Institute for Computational Engineering and Sciences and the Department of Biomedical Engineering, The University of Texas at Austin 2: The University of Luxembourg}
{Both cardiovascular and neurodegenerative diseases manifest as deleterious changes in organ structure at the individual level. This deterioration often differs markedly both between individuals and at different locations within an individual organ. As such, in order to accurately model and simulate the progression of these diseases, this changing organ and tissue structure must be accurately assessed and incorporated into these simulations and models. One prominent method of assessing organ structure is diffusion tensor magnetic resonance imaging (DTMRI), which non-invasively measures tissue’s local diffusivity. Mechanical properties such as degree of anisotropy and principal tissue orientation are then inferred from this measurement, under the assumption that both a tissue’s local diffusivity and its mechanical properties are reflective of its physical structure, and hence are correlated. This data possesses numerous intrinsic symmetries, which when treated naively can lead to models with spurious, unstable, or inconsistent predictions. We present a general framework for consistently and robustly incorporating DTMRI data into hyperelastic models such that the intrinsic symmetries of this data are fully respected by model’s predictions, both in the generically orthotropic case, and the special cases where the data adopts stronger symmetry, such as transverse isotropy or isotropy. By representing the hyperelastic strain energy as a Taylor series in the Green Lagrange strain and the principal diffusivities, we establish the general form of strain energies of arbitrary and independent orders in diffusivity and strain. Further, we exploit the graded structure of polynomial rings to generate compatibility conditions that automatically ensure consistency when the principal diffusivities appear with greater multiplicity. We then provide a computational implementation of such a model, using automatic differentiation with FEniCSx both to perform the differentiable spectral decomposition of the diffusion tensor necessary for the forward solve, and to generate and solve the adjoint equations for gradient computation to fit model parameters to experimental data.}


\vspace{1.5ex}
%-Session 3
\abs
{Computer modeling and simulation of blood flow and tissue deformation in congenital heart defects}
{Charles Puelz$^{1}$, Dan Lior$^{1}$, Craig Rusin$^{1}$, Colin Edwards$^{2}$ and Silvana Molossi$^{1}$}
{1: Department of Pediatrics, Division of Cardiology, Baylor College of Medicine and Texas Children's Hospital 2: Department of Mechanical Engineering, Rice University}
{This talk will focus on the computer modeling of blood flow in patients with abnormal coronary arteries. I will describe the pipeline we use to create these models, which begins with computed tomography images and ends with a computer simulation of blood flow in the aorta, aortic valve, and coronaries. The numerical approach used for these simulations is a version of the immersed boundary method that allows for the prediction of blood flow, blood pressure, and tissue deformations.}


\vspace{1.5ex}
\abs
{Mathematical models and methods in computational neurology}
{Pedro D. Maia$^{1}$}
{1: Department of Mathematics, The University of Texas at Arlington}
{The emerging field of computational neurology provides an important window of opportunity for modeling of complex biophysical phenomena, for scientific computing, for understanding functionality disruption in neural networks, and for applying machine-learning methods for diagnosis and personalized medicine. In this talk, I will illustrate some of our latest results across different spatial scales spanning a broad array of mathematical techniques such as: (i) numerical methods for nonlinear PDEs for solving inhomogeneous active cable equations, (ii) spike-train metrics for quantifying information loss on compromised neural signals, (iii) applied inverse-problem techniques for finding the origins of neurodegeneration, and (iv) data methods in medical imaging.}


\vspace{1.5ex}
\abs
{Telomerase Therapy Reverses Vascular Senescence}
{Anahita Mojiri$^{1}$, Xu Qiu$^{1}$, Elisa Morales$^{1}$, Luay Boulahouache$^{1}$, Chiara Mancino$^{1}$, Rhonda Holgate$^{1}$, Chongming Jiang$^{1}$, Abbie Johnson$^{1}$, Brandon K. Walther$^{1}$, Guangyu Wang$^{1}$, John P. Cooke$^{1}$ MD PhD}
{1: Houston Methodist Research Institute}
{Hutchinson-Gilford Progeria Syndrome (HGPS) accelerates aging and vascular disease with mortality in the teen years due to myocardial infarction and stroke. In HGPS, a mutation in lamin A (progerin) alters nuclear morphology and gene expression. Differentiated endothelial cells (ECs) derived from HGPS-iPSCs exhibited hallmarks of senescence including replication arrest, DNA damage, and short telomeres. HGPS-ECs are dysfunctional and generate inflammatory cytokines that alter adjacent cells. Interestingly, HGPS-ECs compared to HGPS-vascular smooth muscle cells had a more severe senescent phenotype as assessed by telomere size and proliferation rate. We hypothesized that a telomere-directed therapy may restore telomere length and attenuate signs of senescence in human progeria cells and progeria mice models.  Telomerase mRNA (hTERT) normalized HGPS-ECs morphology and functions; restored nitric oxide generation, acetylated-LDL uptake, and angiogenesis; and, reduced the expression of inflammatory cytokines. hTERT improved replicative capacity, reversed cellular senescence phenotypes such as nuclear morphology, normalized p16, p21, lamin B1 expression, reduced DNA damage signals, and restored histone protein expression in HGPS-EC. Intriguingly, the aberrant transcriptional profile observed in HGPS was normalized, and over 250 genes were fully restored by hTERT treatment.  HGPS mice treated with mTERT lentivirus manifested similar improvements with a reduction in the vasculature of the inflammatory marker VCAM1, and the DNA damage marker $\gamma$H2A.X. Lastly, mTERT therapy increased the lifespan of HGPS mice significantly without adverse cytopathological effects. In conclusion, vascular rejuvenation using telomerase mRNA is a promising approach for HGPS and perhaps other age-related vascular diseases. Currently, we are optimizing the lipid nanoparticles encasing the mTERT RNA for reliable delivery to the vasculature.}

\end{addmargin}


