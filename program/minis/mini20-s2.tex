\mini
{mini20}
{Recent Developments in Model Reduction and Low Rank Algorithms}
{Organizers: Zhichao Peng \& Min Wang}
{Numerical simulations of many real world scientific and engineering problems from chemically reacting flows to plasma physics involve a large number of degrees of freedom. This makes the outer-loop applications such as optimization, control, design, sensing and uncertainty quantification computationally expensive. Model reduction techniques and low rank algorithms, which explore and utilize the underlying low rank feature of the underlying problem, could dramatically accelerate these large-scale simulations. In this minisyposium, we would focus on the recent developments in model reduction and low rank algorithms such as machine learning based methods, structure preserving methods and least squares methods.}
{Location: CBB 120}

\begin{talks}
\item\talk
{Least-squares Parametric Reduced-order Modeling}
{Petar Mlinari\'{c} and Serkan G\"{u}\u{g}ercin}
{Virginia Tech}
\item\talk
{Structure-preserving machine learning moment closures for the radiative transfer equation}
{Juntao Huang$^{1}$, Yingda Cheng$^{2}$, Andrew J. Christlieb$^{2}$, Luke F. Roberts$^{3}$ and Wen-An Yong$^{4}$}
{1: Texas Tech University, 2: Michigan State University, 3: Los Alamos National Lab, 4: Tsinghua University}
\item\talk
{Fast Online Adaptive Enrichment for Poroelasticity with High Contrast}
{Xin Su$^{1}$, Sai-Mang Pun$^{1}$}
{1: Texas A\&M University}
\item\talk
{A new reduced order model of  linear parabolic PDEs}
{Yangwen Zhang$^{1}$, Noel Walkington$^{1}$, Franziska Weber$^{2}$}
{1: Carnegie Mellon University, 2: University of California Berkeley}
\end{talks}
\room
