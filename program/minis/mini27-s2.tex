\mini
{mini27}
{Challenges and opportunities in computational science and engineering: Perspectives from data-driven learning and model reduction}
{Organizers: Ionut Farcas, Marco Tezzele \& Aniketh Kalur}
{The high computational cost of realistic simulations of real-world phenomena - even on parallel supercomputers - renders tasks that require ensembles of such simulations, i.e., outer-loop applications, such as uncertainty quantification, control, parameter inference or design optimization, computationally infeasible. To this end, model-order reduction can be used to construct fast and accurate reduced models of complex simulations and therefore speed up outer-loop applications. In addition, primarily due to the tremendous recent advances in computing, data-driven learning and modeling emerged as a viable and practically feasible way of addressing the computational challenges in the aforementioned complex tasks as well. The aim of the mini-symposium is to foster discussion on data-driven methods, system identification, model order reduction, and uncertainty quantification for complex applications in computational science and engineering. The talks will present methodological developments as well as applications from different engineering fields.}
{Location: CBB 120}

\begin{talks}
\item\talk
{Discovering Model Error with interpretability and Data-Assimilation: Sparse observations of multi-scale flows}
{Rambod Mojgani$^{1}$, Ashesh Chattopadhyay$^{1}$, Pedram Hassanzadeh$^{1}$}
{1: Rice University}
\item\talk
{Compiler-based Differentiable Programming for Accelerated Simulations}
{Ludger Paehler$^{1,2}$ and Jan Hueckelheim$^{2}$ and Johannes Doerfert$^{3}$}
{1: Technical University of Munich, 2: Argonne National Laboratory, 3: Lawrence Livermore National Laboratory}
\item\talk
{Advances in parameter space reduction with applications in naval engineering}
{M. Tezzele$^{1}$ and G. Rozza$^{2}$}
{1: University of Texas at Austin, 2: International School for Advanced Studies}
\end{talks}
\room
