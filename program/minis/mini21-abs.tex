\refstepcounter{dummy}\label{mini21}

\miniabs
{Recent Advances of Numerical Simulations for Fluid Flows and Applications}
{Organizers: Yong Yang \& Yonghua Yan}
{Numerical simulations have become vital research tools in fluid flows. This mini-symposium is dedicated to recent developments of numerical simulations for fluid flows and applications. All numerical methods including the finite difference method, the finite volume method, the finite element method, and the lattice Boltzmann method are welcome.}

\begin{addmargin}[2em]{0em}
\vspace{2ex}
\abs
{Study on the asymmetrical structure in late transitional boundary layer using DNS and Rortex}
{Yong Yang$^{1}$, Yonghua Yan$^{2}$, and Caixia Chen$^{3}$}
{1: West Texas A\&M University, 2: Jackson State University, 3: Tougaloo College}
{The late transitional boundary layer is simulated by the Direct Numerical Simulation (DNS) on a flat plate at Mach number 0.5 and the asymmetrical structures are investigated. To study the source of the asymmetry from symmetric structures, we utilized the newly developed vortex visualization method, Rortex, which has an advantage in identifying vortex cores. The study shows that the origin of asymmetrical structures of the transitional flow matches with the local extreme zone of Rortex. The evolution of the asymmetrical flow structures developed along with large scale vortex cores that are defined by Rortex, which shows the generation of the asymmetry have strong correlations with fluid rotation. This study provides new insights for a deeper understanding of mechanism of turbulence.}


\vspace{1.5ex}
\abs
{Numerical study of the vortical structures in MVG controlled supersonic flow under different Mach numbers}
{Yonghua Yan$^{1}$, Yong Yang$^{2}$, and Shiming Yuan$^{1}$}
{1: Jackson State University, 2: West Texas A\&M University}
{MVG (Micro Vortex generator) is a small passive control device used to control boundary layer flow. Previous study found that the ring-like vortex generated by MVG plays a very important role in reducing the separation zone caused by ramp shock waves. A clear understanding of the mechanism of the structure of vortices contributes to further understanding of SWBLI (Shock Wave Boundary Layer Interaction) and the control of flow separation. The purpose of this study was to investigate the effect of different incoming flow velocities on the vortex structure generated by the MVG. This study investigates MVG-controlled supersonic flow at five different Mach numbers ranging from 1.5 to 5.0. Numerical results show that the large eddy structure generated by the MVG changes as the Mach number increases. At large Mach numbers, the structure in the lower boundary layer will seriously affect the vortex structure generated by the MVG, thereby affecting the formation of momentum deficit and ring-like vortices.}


\vspace{1.5ex}
\abs
{LBM simulation of swallowing process with dysphagia}
{Caixia Chen$^{1}$, Yonghua Yan$^{2}$, Yong Yang$^{3}$, and Demetric L. Barnes$^{2}$}
{1: Tougaloo College, 2: Jackson State University, 3: West Texas A\&M University}
{Modeling and numerical simulation of the process of swallowing bolus food during the pharyngeal stage is conducted to understand the problems caused by dysphagia, which refers to difficulty eating or swallowing. Understanding normal swallowing mechanisms and how they are disrupted is an important patient safety goal in providing care. In this study, bolus flow is regarded as a free surface jet in hydrodynamics. The multiphase LBM (Lattice Boltzmann Method) algorithm is used to model the bolus flow. Two-dimensional numerical results of bolus flow with different flow parameters, such as viscosity and initial velocity of the flow, were obtained without considering the effect of respiration. In the absence of normal swallowing ability, the possibility of choking on water/food during swallowing was investigated. This study provides some guidance for understanding dysphagia and finding corresponding aids.}


\vspace{1.5ex}
\abs
{Power spectrum analysis of the interaction between large vortices and ramp shock waves in MVG controlled supersonic ramp flow}
{Demetric l Banines$^{1}$, Yong Yang$^{2}$, Shiming Yuan$^{1}$, and Yonghua Yan$^{1}$}
{1: Jackson State University, 2: West Texas A\&M University}
{After decades of intensive research, the driving source of flow frequency instabilities in shock boundary layer interaction (SWBLI) remains unknown. In this study, high-resolution large eddy simulation (LES) was adopted to simulate supersonic ramp flow controlled by a micro-vortex generator (MVG). Power spectrum analysis of the interaction between the large-scale vortices and ramp shock waves were conducted at the ramp corner. The frequency distributions of large-scale vortex and shock oscillations at different Mach numbers in the MVG-controlled SWBLI region are investigated. In each case, the low frequencies of the flow instabilities are dominated by the frequencies of the large-scale eddies. A strong correlation is observed between the low-frequency distribution of large-scale vortices and shock oscillations. Although the vortex structure became more complex, the correlation did not decrease with larger Mach numbers.}
\end{addmargin}
