\mini
{mini01}
{Recent advances in large-scale inverse problems: Numerics, theory, and applications}
{Organizer: Alexander Mamonov, Andreas Mang \& Daniel Onofrei}
{Despite formidable advances in recent years, significant challenges remain. In inverse problems, parameters are typically related to indirect measurements by a mathematical model (for example, a PDE) in a highly nonlinear way, resulting in non-convex, non-linear optimization problems. These problems are challenging to solve in an efficient way. We will discuss recent advances in numerical methods and the theory of inverse problems to address these challenges. This minisymposium aims to attract researchers at the forefront of inverse problems, inference, and data science to present their latest work on fast algorithms and theory in inverse problems, and exciting applications.}
{Location: CEMO 101}


\begin{talks}
\item\talk
{Inverse problems of subsurface flows with low permeability fault structures}
{Jeonghun (John) Lee}
{Department of Mathematics, Baylor University}
\item\talk
{Spatio-temporal quantification of pathological tau spreading in Alzheimer's disease}
{Zheyu Wen, Ali Ghafouri \& George Biros}
{Oden Institute, The University of Texas at Austin}
\item\talk
{An inverse solver for a multispecies tumor growth model}
{Ali Ghafouri \& George Biros}
{Oden Institute, University of Texas at Austin}
\item\talk
{TBD}
{Alexandros G. Dimakis}
{Oden Institute, University of Texas at Austin}
\end{talks}

\room
