\mini
{mini01}
{Recent advances in large-scale inverse problems: Numerics, theory, and applications}
{Organizer: Alexander Mamonov, Andreas Mang \& Daniel Onofrei}
{Despite formidable advances in recent years, significant challenges remain. In inverse problems, parameters are typically related to indirect measurements by a mathematical model (for example, a PDE) in a highly nonlinear way, resulting in non-convex, non-linear optimization problems. These problems are challenging to solve in an efficient way. We will discuss recent advances in numerical methods and the theory of inverse problems to address these challenges. This minisymposium aims to attract researchers at the forefront of inverse problems, inference, and data science to present their latest work on fast algorithms and theory in inverse problems, and exciting applications.}
{Location}


\begin{talks}
\item\talk
{Some results on inverse problems to elliptic {PDE}s with solution data and their implications in operator learning}
{Kui Ren}
{Department of Applied Physics and Applied Mathematics and Data Science Institute, Columbia University}
%{In recent years, there have been great interests in discovering structures of partial differential equations from given solution data. Very promising theory and computational algorithms have been proposed for such identification problems in different settings. We will try to review some recent understandings of such PDE learning problems from the perspective of inverse problems. In particularly, we will highlight a few computational and analytical understandings on learning a second-order elliptic PDE from single and multiple solutions.}
\item\talk
{Estimating the noise level in seismic data while overcoming cycle skipping}
{Susan Minkoff$^{1}$, Huiyi Chen$^{1}$ \& William Symes$^{2}$}
{1: Department of Mathematical Sciences, University of Texas at Dallas, 2:  Department of Computational and Applied Mathematics, Rice University}
%{Full waveform inversion (FWI) suffers from the well-known cycle skipping problem in which local gradient-based optimization may fail to converge to geologically meaningful earth models if the initial guess for the optimization is not close enough to the true earth model. Extension methods attempt to overcome the cycle skipping problem by enlarging the space of acceptable models. In the case of source extension, the inverse problem is extended to allow for estimation of both the medium parameters (velocity) and the source wavelet via the addition of a penalty term. The extension does not require the source to be compactly supported in time. This extended objective function can be minimized efficiently  via the discrepancy principle in which the source time function and velocity are estimated in a nested loop. While this approach has been shown to overcome cycle skipping, it does require use of a well-tuned penalty weight which depends on the data noise level, which is generally unknown. In this talk I will describe and illustrate an algorithm to simultaneously solve the inverse problem while accurately estimating the data noise level.}
\item\talk
{Inverse problems of subsurface flows with low permeability fault structures}
{Jeonghun (John) Lee}
{Department of Mathematics, Baylor University}
% {In this work we consider inverse problems of subsurface flow models with low permeability fault structures. We first consider deterministic inversion of fault transmissibility parameters via constrained optimization approach with Newton conjugate gradient methods. We then consider efficient Bayesian inversion using the Laplace approximation of posterior at MAP point. This is a joint work with Umberto Villa, Tan Bui-Thanh, Omar Ghattas at the Oden Institute of Computational Engineering and Sciences in University of Texas Austin.}
\item\talk
{An inverse solver for a multispecies tumor growth model}
{Ali Ghafouri \& George Biros}
{Oden Institute, University of Texas at Austin}
% {Biophysical models of tumor growth at the tissue level can be used for patient stratification, preoperative planning, treatment planning, and prognosis. They also help bridge imaging phenotype with molecular drivers of cancer. Here we focus on the mathematical structure of a parameter estimation problem for multi-species model of tumor growth. This model is designed for glioblastomas but in principle it can be used to other solid tumors.  We establish connections with classical inverse problem theory, and we discuss some unique challenges. It is a single-shot inversion (that is we using only one time snapshot), it has a large number of parameters, and involves nonlinear, non-differential operators. We propose an inversion methodology that attempts to address some of these challenges, and we present numerical results that illustrate the strengths and weaknesses of the proposed scheme.}
\end{talks}

\room

