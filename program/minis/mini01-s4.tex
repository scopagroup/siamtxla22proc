\mini
{mini01}
{Recent advances in large-scale inverse problems: Numerics, theory, and applications}
{Organizer: Alexander Mamonov, Andreas Mang \& Daniel Onofrei}
{Despite formidable advances in recent years, significant challenges remain. In inverse problems, parameters are typically related to indirect measurements by a mathematical model (for example, a PDE) in a highly nonlinear way, resulting in non-convex, non-linear optimization problems. These problems are challenging to solve in an efficient way. We will discuss recent advances in numerical methods and the theory of inverse problems to address these challenges. This minisymposium aims to attract researchers at the forefront of inverse problems, inference, and data science to present their latest work on fast algorithms and theory in inverse problems, and exciting applications.}
{Location: CEMO 105}


\begin{talks}
\item\talk
{\st{Fast approximations of high-rank Hessians: Applications to seismic inversion and uncertainty quantification}}
{\st{Mathew Hu$^{1,*}$, Nick Alger$^{1}$, Longfei Gao$^{1}$, Omar Ghattas$^{1}$ \& Rami Nammour$^{2}$}}
{\st{1: Oden Institute, The University of Texas at Austin; 2: Total E\&P Research and Technology} {\color{p1color}Talk moved to Saturday, 8:30 AM}}
\item\talk
{\st{Matrix-free PSF approximation of ice-sheet Hessians}}
{\st{Nick Alger$^{1,*}$, Tucker Hartland$^{2}$, Noemi Petra$^{2}$, Omar Ghattas$^{1}$}}
{\st{1: Oden Institute, The University of Texas at Austin; 2: Applied Mathematics, University of California, Merced} {\color{p1color}Talk moved to Saturday, 9:00 AM}}
\item\talk
{TBA}
{TBA}
{TBA}
\item\talk
{TBA}
{TBA}
{TBA}
\end{talks}

\room
