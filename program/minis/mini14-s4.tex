\mini
{mini14}
{Mathematics and Computation in Biomedicine}
{Organizers: Sebastian Acosta and Charles Puelz}
{This mini-symposium is concerned with new developments in mathematical modeling, numerical methods, and computational science for applications in biomedicine in the broadest possible sense. Topics include biomechanics, cardiovascular simulations, inverse problems, imaging, computational oncology, epidemiology and machine learning.}
{Location: CBB 124}

\begin{talks}
\item\talk
{Fast algorithms for diffeomorphic image registration}
{Andreas Mang}
{Department of Mathematics, University of Houston}
\item\talk
{PocketNet: A smaller neural network for medical image analysis}
{Adrian Celaya, Jonas A. Actor, Rajarajeswari Muthusivarajan, Evan
Gates, Caroline Chung, Dawid Schellingerhout, Beatrice Riviere, and
David Fuentes}
{Rice University and MD Anderson Cancer Center}
\item\talk
{Optimal transport-based segmentation and classification of ECG signals: Preliminary results and challenges}
{Cesar A Uribe$^{1}$, Edward Nguyen$^{1}$, Sebastian Acosta$^{2}$, Emily Zhang$^{3}$, Jelena Lazic$^{1}$}
{1: Rice University, 2: Baylor College of Medicine, 3: Wellesley College}
\item\talk
{Using deep learning and macroscopic imaging of porcine heart valve leaflets to predict uniaxial stress-strain responses}
{Luis Hector Victor, CJ Barberan, Richard G. Baranuik, and Jane Grande-Allen}
{Rice University}
\end{talks}
\room
