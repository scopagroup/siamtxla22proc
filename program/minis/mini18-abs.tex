\label{mini18}

\miniabs
{}
{Organizers: }
{}


\vspace{2ex}
%Session 1
\abs
{Dirac points for the honeycomb lattice with impenetrable obstacles}
{Junshan Lin$^{1}$, Wei Li$^{2}$, Hai Zhang$^{3}$}
{1: Auburn University, 2: DePaul University, 3: Hong Kong University of Science and Technology}
{Dirac points are special vertices in the band structure when two bands of the spectrum for the operator touch in a linear conical fashion, and their investigations play an important role in the design of novel topological materials. In this talk, I will discuss Dirac points for the honeycomb lattice with impenetrable obstacles arranged periodically in a homogeneous medium. I will discuss both the Dirichlet and Neumann eigenvalue problems and prove the existence of Dirac points for both eigenvalue problems at crossing of the lower band surfaces as well as higher band surfaces. In addition, quantitative analysis for the eigenvalues and the slopes of two conical dispersion surfaces near each Dirac point will be presented by a combination of the layer potential technique and asymptotic analysis.}


\vspace{1.5ex}
\abs
{Optimal control of the Landau--de Gennes model of nematic liquid crystals}
{Shawn Walker$^{1}$ and Thomas Surowiec$^{2}$}
{1: LSU, 2: Philipps University of Marburg}
{This talk presents an optimal control framework for the time-dependent, Landau--de Gennes (LdG) model of nematic liquid crystals.  Since the LdG energy is non-convex, we develop parabolic, optimal control techniques for controlling the $L^2$ gradient flow of the LdG energy, which is uniquely solvable.  The controls are through the boundary conditions (by weak anchoring) and a body force term. We seek to find optimal controls that drive the LdG Q-tensor variable toward a desired "texture" state.  The objective functional we minimize is of tracking type with additional regularization terms for the controls.  To the best of our knowledge, this is the first time PDE-based optimal control has been developed for the LdG model. Existence of a minimizer for the control problem is established.  Moreover, with various regularity estimates, we prove first order Frech\'{e}t differentiability results for the control objective, by introducing an adjoint PDE, thus allowing gradient based optimization methods.  In the talk, we highlight the analytical issues that arise, especially those due to the gradient flow being a parabolic system.  We then describe a finite element discretization of the full control problem and present numerical simulations in two and three dimensions that exhibit point and line defects.}


\vspace{1.5ex}
\abs
{Embedded eigenvalues for discrete magnetic Schrödinger operators}
{Jorge Villalobos$^{1}$ and Stephen Shipman$^{1}$}
{1: LSU}
{Reducibility of the Fermi surface for a periodic operator is a key for the existence of embedded eigenvalues caused by a local defect.  We consider a discrete model for a multilayer quantum system, such as stacked graphene, subject to a perpendicular magnetic field.  Some techniques for constructing embedded eigenvalues extend from non-magnetic operators to magnetic ones, but the magnetic case is more complex because a typical magnetic operator on a periodic graph is merely quasi-periodic.}


\vspace{1.5ex}
\abs
{Effective Impedance Condition for Thin Metasurfaces}
{Zachary Jermain$^{1}$, Robert Lipton$^{1}$}
{1: LSU}
{Here we examine the optical response of thin metasurfaces consisting of arrangements of periodic metallic nanoparticles placed on a dielectric film layer. The dielectric film layer is composed of a Liquid Crystal Elastomer (LCE) layer which can isomerize and change its configuration for specific frequencies of incident light called the ``pump frequency”. With this unique property one can design a dynamic metasurface which actuates to change thickness by interaction with light alone. Ultimately one can control the optical response of an incident electromagnetic wave by controlling the thickness of the LCE layer, i.e. control light with light. We aim to help inform the design of these metasurfaces by deriving effective properties which control the optical response. Specifically, we use asymptotic methods to reduce the metasurface to an effective surface impedance condition. The impedance condition will depend on the geometry of the metallic nanoparticles, the material properties of the metal and film, the periodicity of the nanoparticles, and the thickness of the LCE layer. Finally, we look to find the relationship between the impedance condition and plasmon resonances which occur at specific frequencies of incident light.}


\vspace{1.5ex}
%Session 2
\abs
{Domain wall junctions and networks in Dirac topological materials}
{P. Cazeaux$^{1}$, D. Massatt$^{2}$, G. Bal$^3$, and S. Quinn$^3$}
{1: Virginia Tech, 2: LSU, 3: University of Chicago}
{In this talk, we will discuss effective Dirac models for topological modes propagating along domain wall in the presence of large gapped domains, such as the AB/BA triangular domains in small-angle twisted bilayer graphene in the presence of a strong potential difference between layers, which has attracted interest in recent years. We will rigorously define topological edge invariants associated with massive Dirac operators and their robustness with respect to perturbations and geometries such as domain wall junctions, and introduce discretization strategies for these nonstandard effective Dirac equations, both for the computation of edge invariants as well as the simulation of time-dependent wavepacket propagation on the network formed by the domain walls. Using numerical simulations, we will illustrate the robust behavior in the presence of strong disorder, curved and/or intersecting domain walls.}


\vspace{1.5ex}
\abs
{Shape Optimization of Microstructures Governed by Maxwell's Equations}
{Manaswinee Bezbaruah$^{1}$, Matthias Maier$^{1}$, Winnifried Wollner$^{2}$}
{1: Texas A\&M University, 2: Universitaet Hamburg, Germany}
{This talk is concerned with a class of shape optimization problems involving optical metamaterial comprised of periodic nanoscale inclusions. We will first summarize the underlying microscale model and a corresponding homogenization theory that will serve as a basis for the shape optimization problem. We then introduce an arbitrary Lagrangian-Eulerian (ALE) formulation of the cell problems and formulate an optimization problem.}


\vspace{1.5ex}
\abs
{Two new finite element schemes for a time-domain carpet cloak model with metamaterials}
{Jichun Li$^{1}$, Chi-Wang Shu$^{2}$ and Wei Yang$^{3}$}
{1: University of Nevada Las Vegas, 2: Brown University, 3. Xiangtan University}
{This talk is concerned about a time-domain carpet cloak model, which was originally derived in our previous work (Li et al., SIAM J. Appl. Math., 74(4), pp. 1136–1151, 2014). However, the stability of the proposed explicit finite element scheme is only proved under the time step constraint $\tau = O(h^2)$, which is too restricted.  To overcome this disadvantage, in this new work we propose two new finite element schemes for solving this carpet cloak model: one is the implicit Crank-Nicolson (CN) scheme, and another one is the explicit leap-frog (LF) scheme. By using a totally new energy, we prove the unconditional stability for the CN scheme and  conditional stability for the LF scheme under the practical CFL constraint $\tau = O(h)$.   Optimal error estimates are also established for both schemes. Finally, numerical results  are presented to support our analysis and demonstrate the cloaking phenomenon.}


\vspace{1.5ex}
\abs
{Bloch Waves for Maxwell's Equations in High-Contrast Electromagnetic Crystals}
{Robert Viator}
{Swarthmore College}
{We investigate the Bloch spectrum of a 3-dimensional high-contrast photonic crystal. The Bloch eigen-values, for fixed quasi-momentum, are expanded in a power series in the material contrast parameter in the high-contrast limit. We achieve this power series, together with a radius of convergence, by decomposing an appropriate vectorial Sobolev space into three mutually orthogonal subspaces which are curl-free in certain subdomains of the period cell. We also identify the limit spectrum in the periodic (zero-quasi-momentum) case. Time permitting, we will describe a wide class of crystal geometries which permit the above described analytic structure of the Bloch eigenvalues.}



