\label{mini24}


\miniabs
{Nonlinear Physics Models and Mathematical Structures}
{Organizers: Vesselin Vatchev, Zhijun Qiao, Wilson Zuniga Galindo \& Erwin Suazo}
{The aim of the mini symposium is participants to present talks and engage in a discussion of a variety of recent work and results in the area of Nonlinear Physics. The emphasis is on nonlinear PDE’s, properties of their solutions, and theoretical or numerical methods for solving them with direct application to Physics Models. Topics of interest are developing models, numerical solutions and methods designed to better capture Physical characteristics of PDE’s, and in more general terms providing mathematical structures to describe Physics phenomenon.}

\begin{addmargin}[2em]{0em}
\vspace{2ex}
%Session 1
\abs
{5th-order CH equation with pseudo-peakons}
{Zhijun Qiao }
{UT Rio Grande Valley}
{In this talk, I will majorly focus on the scalar peakon models developed in the last 30 years. Most integrable peakon equations come from the negative order flow in the hierarchy. I will take some examples to explain higher order models with peakons or pseudo-peakons we proposed recently. Some open problems will also be addressed for discussion in the end.}


\vspace{1.5ex}
\abs
{Interactions of Traveling Waves in Low Order Approximations
for the Two-Dimensional Euler Equations}
{Julio Paez}
{UT Rio Grande Valley}
{In this talk, we discuss traveling waves with elastic and inelastic interactions in a low order of approximation. The waves are derived from the two-dimensional Euler equations in quadratic or linear order of approximation.  A new type of inelastic interaction that is closely related to numerical study observations of the Euler equations is obtained in linear order.}


\vspace{1.5ex}
\abs
{Engineered Approximate Solutions to the Two-Dimensional Euler Equations}
{Vesselin Vatchev}
{UT Rio Grande Valley}
{The two-dimensional Euler equations (EE) model wave propagation in a shallow thin channel. The relative simplicity of the physical setting allows for experimental observations, but the nonlinearity of the system is very challenging for general theoretical results. One of the most effective approaches to study EE is by using asymptotic expansion of the variables to derive an approximate system. The order of approximation is measured in the powers of the small wave amplitude. Boussinesq, KdV and Fifth order KdV, Camassa-Holm equations are a few examples obtained by using the technique with different order of approximation. The resulting systems have traveling wave solutions, but they are still approximation of the equation’s order to the solutions of EE. Experiments indicate the practical importance of traveling wave solutions with special features which cannot be easily related to a solution of an approximate system. In the talk we discuss a modification of the asymptotic expansion method to engineer surface wave interactions (which are not solutions of approximate systems)  that are practical, of quadratic order of approximation, and easy to study.}


\vspace{1.5ex}
\abs
{Probabilistic solutions of fractional differential and partial differential equations and their Monte Carlo simulations}
{Erwin Suazo}
{UT Rio Grande Valley}
{The work in this paper is four-fold. Firstly, we introduce an alternative approach to solve fractional ordinary differential equations as an expected value of a random time process. Using the latter, we present an interesting numerical approach based on Monte Carlo integration to simulate solutions of fractional ordinary and partial differential equations. Thirdly, we show that this approach allows us to find the fundamental solutions for fractional partial differential equations (PDEs), in which the fractional derivative in time is in the Caputo sense and the fractional in space one is in the Riesz-Feller sense. Lastly, using Riccati equation, we study families of fractional PDEs with variable coefficients which allow explicit solutions. Those solutions connect Lie symmetries to fractional PDEs. This is joint work with Dr. T. Oraby and graduate student H. Arrubla.}


\vspace{1.5ex}
%Session 2
\abs
{p-Adic Statistical Field Theory and Deep Belief Networks}
{W. A. Zúñiga-Galindo }
{UT Rio Grande Valley}
{The talk aims to present the results of our preprint arXiv:2207.13877. In this work we initiate the study of the correspondence between p-adic statistical field theories (SFTs) and neural networks (NNs). In general quantum field theories over a p-adic spacetime can be formulated in a rigorous way. Nowadays these theories are considered just mathematical toy models for understanding the problems of the true theories. In this work we show these theories are deeply connected with the deep belief networks (DBNs). Hinton et al. constructed DBNs by stacking several restricted Boltzmann machines (RBMs). The purpose of this construction is to obtain a network with a hierarchical structure (a deep learning architecture). An RBM corresponds to a certain spin glass, thus a DBN should correspond to an ultrametric (hierarchical) spin glass. A model of such a system can be easily constructed by using p-adic numbers. In our approach, a p-adic SFT corresponds to a p-adic continuous DBN, and a discretization of this theory corresponds to a p-adic discrete DBN. We show that these last machines are universal approximators.}


\vspace{1.5ex}
\abs
{Quantum Control of Qubits and Qutrits}
{Baboucarr Dibba}
{UT Rio Grande Valley}
{The main setup of quantum control is that the optimization of data output and noise
decoherence and such optimality is to create certain the integrity, confidentiality, and
credibility and additionally the non-interference of noise in any data transmission.
Putting into consideration the coherent states with the bichromatic cavity modes,
as we developed the Euler angle-dressed state used in calculating the amplitudes
of the three-level quantized systems. The atom-field entanglement is taken under
consideration for the measured three-level systems applying the Phoenix-Knight
formalism and corresponding population inversion with an observation of the collapse
and revival of different signal stages for the systems. A very impressive concept is
taken through an experiment and in theory in superconducting quantum circuits, that
provide a platform for manipulating microwave photons. As the optimum generation of
entangled states is of great significance to quantum information science.

Theoretically, utilizing an approach centered on the widespread excited Raman-adiabatic passage
technique and invariant-based shortcuts(STA) to adiabaticity achieves these objectives for quantum state transfers in common three-level systems whilst considering coupled photonic grid as a harmonic oscillator with frequency and time-varying mass. We show numerically that the technique in order with a wide set of control
parameters, initiating the timescales nearer to the quantum speed limit, in addition,
considering the presence of environmental disturbance.}


\vspace{1.5ex}
\abs
{p-adic Cellular Neural Networks: Applications to Image Processing}
{Brian Zambrano}
{UT Rio Grande Valley}
{In this talk, we present two applications of some p-adic Cellular Neural Networks (CNN) modeled by PDEs. First, an edge detector based on p-adic CNN for gray images comparing the results with the Canny edge detector. Second, we also present a new p-adic CNN with a diffusion term. This network can be considered a p-adic generalization of a Stated Control- CNN.   We use this new network to process some noised images and compare them with Perona-Malik’s results.}


\vspace{1.5ex}
\abs
{TBA}
{Nathaniel Mayes}
{UT Rio Grande Valley}
{TBA}
\end{addmargin}




