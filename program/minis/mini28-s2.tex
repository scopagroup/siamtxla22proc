\mini
{mini28}
{Modeling of air flow and droplet transport for biomedical applications}
{Organizers: Vladimir Ajaev \& Andrea Barreiro}
{Mathematical and computational modeling of air flow in the respiratory system is important for a range of applications including prevention of infectious diseases such as tuberculosis and COVID-19, understanding the functioning of the olfactory system, and improving therapeutic efficiency of drugs delivered by aerosol inhalation. The mini-symposium will focus on recent progress in the computational modeling of air flow in nasal cavity and respiratory airways, as well as the transport of microscale droplets which can carry infections or inhaled medication.    The physics of evaporation and condensation of such droplets will be discussed with implications to their trajectories and ultimate locations of their deposition. After deposition, the transport of components of droplet through mucus layer is of interest and can be studied using the methods of computational fluid mechanics. Potential applications of machine learning in combination with the standard numerical solutions of the Navier-Stokes equations, as well as other future research directions in the field will among the topics for the mini-symposium.}
{Location: CBB 118}

\begin{talks}
\item\talk
{Predicting Transport and Deposition of Multicomponent E-cigarette Aerosols in a Subject-specific Airway Model with Different Nicotine Forms: An in silico Study}
{Ted Sperry, Yu Feng}
{Oklahoma State University}
\item\talk
{Mathematical Modeling of Phase Change in Respiratory Droplets}
{Vladimir S. Ajaev, Art Taychameekiatchai, James Barrett}
{Southern Methodist University}
\item\talk
{Investigating the Impact of Mechanosensation on Retronasal Olfaction}
{Abdullah Saifee$^1$, Andrea Barreiro$^1$,Cheng Ly$^2$,Woodrow Shew$^3$}
{1:Southern Methodist University, 2:Virginia Commonwealth University, 3:University of Arkansas}
\end{talks}
\room
