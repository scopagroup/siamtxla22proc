\mini
{mini25}
{High Order Methods for Computational Hydrodynamics}
{Organizers: Madison Sheridan \& Bennett Clayton}
{The construction of accurate numerical methods is essential for simulating transport phenomena in the field of computational hydrodynamics. The aim of this mini-symposium is to discuss the current state of higher-order accurate methods for simulating realistic hydrodynamics. In particular, we are interested in numerical methods that are higher-order accurate in space and time while maintaining structure preserving properties. The partial differential equations we have in mind are those with dominant hyperbolic features such as: (i) the compressible Euler Equations; (ii) the Shallow Water Equations; (iii) the compressible Navier-Stokes Equations; (iv) radiation transport equations such as the Boltzmann Equation. These equations are often strongly nonlinear and pose numerous challenges when one tries to discretize and solve them numerically.}
{Location: CBB 214}

\begin{talks}
\item\talk
{The immersed interface method for simulating flows around rigid objects represented by triangular meshes}
{Sheng Xu}
{Southern Methodist University}
\item\talk
{A parallel high-order overset framework for compressible turbulent flows}
{Amir Akbarzadeh$^1$, Lai Wang$^1$, Freddie Witherden$^2$, Antony Jameson$^{1,2}$}
{$1$: Texas A\&M University, Department of Aerospace Engineering, $1$: Texas A\&M University, Department of Ocean Engineering}
\item\talk
{Guaranteed upper bound for the maximum speed\\ of propagation in the Riemann problem}
{Bennet Clayton$^1$, Jean-Luc Guermond$^1$, Bojan Popov$^1$}
{1: Department of Mathematics, Texas A\&M University, 3368 TAMU, College Station, TX 77843, USA.}
\item\talk
{Higher-order methods for phase-resolving wave/structure interaction}
{Chris Kees and Wen-Huai Tsao and Rebecca Schurr}
{Louisiana State University}
\end{talks}
\room
