\label{mini14}

\miniabs
{Mathematics and Computation in Biomedicine}
{Organizers: Sebastian Acosta and Charles Puelz}
{This mini-symposium is concerned with new developments in mathematical modeling, numerical methods, and computational science for applications in biomedicine in the broadest possible sense. Topics include biomechanics, cardiovascular simulations, inverse problems, imaging, computational oncology, epidemiology and machine learning.}


\begin{addmargin}[2em]{0em}
\vspace{2ex}
\abs
{Stochastics epidemic models with infection-age dependent infectivity in large populations}
{Guodong (Gordon) Pang}
{Rice University}
{In this talk we will discuss several stochastic epidemic models recently developed to account for general infectious durations, infection-age dependent infectivity and/or progress loss of immunity/varying susceptibility, extending the standard epidemic models (including SIR, SEIR, SIRS, SEIRS). Each individual in the population is attached with a random function/process that represents the infectivity force to exert on other individuals. This approach models infection-age dependent infectivity, and is also extended to include a random function that represents the immunity/susceptibility attached to each individual. A typical infectivity function first increases and then decreases from the epoch of becoming infected to the time of recovery, while a typical susceptibility function gradually increases from the time of recovery to the time of losing immunity and becoming fully susceptible. We analyze the population dynamics of the infected and recovered individuals, and the total infectivity and susceptibility processes by establishing the scaling limits (functional law of large numbers and central limit theorems) in large populations. The limits are deterministic and stochastic Volterra integral equations, respectively. We also discuss some new PDEs models arising from the scaling limits.}


\vspace{1.5ex}
\abs
{On the endemic behavior of a competitive tri-virus SIS networked model}
{Sebin Gracy, Mengbin Ye, Brian D.O. Anderson, and Cesar A. Uribe}
{Rice University}
{We study the endemic behavior of a multi-competitive networked susceptible-infected-susceptible (SIS) model. In particular, we focus on the case where there are three competing viruses (i.e., tri-virus system). First, we show that the tri-virus system is not monotone. Thereafter, we provide a necessary and sufficient condition for local exponential convergence to a boundary equilibrium (exactly one virus is alive, the other two are dead) and identify a special case that admits the existence and local exponential attractivity of a line of coexisting equilibria (at least two viruses are active). Finally, we identify a particular case (subsumed by the aforementioned special case) such that for all nonzero initial infection levels, the dynamics of the tri-virus system converge to a plane of coexisting equilibria.}


\vspace{1.5ex}
\abs
{Spatio-temporal quantification of pathological tau spreading in Alzheimer's disease}
{Zheyu Wen, Ali Ghafouri, George Biros}
{The University of Texas at Austin}
{Tau lesions (tau) are one of the main biomarkers of Alzheimer's disease (AD). Quantitatively describing how tau spreads in human brains can help with AD diagnosis and prognosis. Tau can be imaged spatially using positron emission tomography (tau-PET). Our goal is to use tau-PET images along with traditional magnetic resonance imaging to learn a spatio-temporal model of tau propagation. In this talk we will discuss the mathematical and computational challenges of the underlying methodology as well as a set of new algorithms that enable quantification and classification of tau spreading. We test our method on a cohort of subjects selected from publicly available datasets.}


\vspace{1.5ex}
\abs
{Mechanistic models of Alzheimer's disease}
{Travis B.~Thompson}
{Texas Tech University}
{Can mathematics understand Alzheimer's disease (AD)?  The discovery of AD, in 1906, marks a watershed at the intersection of neurology, biochemistry and neuroscience.  The first AD patient exhibited a progressive loss of memory, language and behavioral control; upon death, distinctive plaques and neurofibrillary tangles were reported in the brain histology.  These observations prompted questions: How, and why, do plaques form; are they the cause or consequence of cognitive aberrations; are there viable treatments for the disease? Nearly a century passed before any significant progress was made in AD research but, over the last 25 years, breakthrough discoveries have provided significant advancements. The systematic, in vivo study of human AD is ethically constrained but insights from animal models, contemporary improvements in medical imaging and novel methods from mathematics and computing are circumventing the barriers for AD research in Man. In this talk, several in vitro and histopathological observations regarding the progression of protein pathology in AD will be briefly presented.  Following this, the recent \textit{network dynamical systems} (NDS) framework for modeling neurodegenerative dynamics will be introduced; several models within the NDS framework will be surveyed and their coupling to human neuroimaging data will be discussed.  Overall, we will see that mechanistic network mathematical models and scientific computing are emerging as a valuable tool for understanding AD in vivo and, with further research, are likely to provide a novel path forward for the design and testing of treatment and intervention.}


\vspace{1.5ex}
\abs
{Fast algorithms for diffeomorphic image registration}
{Andreas Mang}
{Department of Mathematics, University of Houston}
{We will discuss the implementation and analysis of efficient numerical methods for diffeomorphic image registration. Image registration is a nonlinear, ill-posed inverse problem that poses significant mathematical and computational challenges. We seek to identify a spatial transformation that establishes point-wise correspondences between two images of the same scene. In our formulation, the spatial transformation is parametrized by a smooth, time-dependent velocity field. This velocity field is found by minimizing a variational optimization problem governed by hyperbolic transport equations. Our contributions are the implementation of efficient numerical algorithms for evaluating forward and adjoint operators, fast second-order algorithms for numerical optimization, efficient approaches for preconditioning, and the deployment of our methodology on dedicated high-performance computing architectures. This is joint work with George Biros, Malte Brunn, Naveen Himthani, Jae-Youn Kim, and Miriam Schulte.}


\vspace{1.5ex}
\abs
{Registering MRA images to 4D flow MRI images}
{Dan Lior}
{Baylor College of Medicine}
{A velocity field and intensity field, respectively representing blood flow and vessel tissue of a given patient, are often acquired in the same MRI series. Each of these fields contains valuable information for vessel modeling and analysis that the other lacks. There is an obvious need to combine the fields. An obstacle to this end is the misalignment of the fields. There are several factors contributing to the misalignment, one of which is a shift that can occur in the position and orientation of the patient between scans in the series. A robust method to rigidly register the two fields is based on centerline extraction. This method will be presented.}


\vspace{1.5ex}
\abs
{Optimal transport-based segmentation and classification of ECG signals: Preliminary results and challenges}
{Cesar A Uribe$^{1}$, Edward Nguyen$^{1}$, Sebastian Acosta$^{2}$, Emily Zhang$^{3}$, Jelena Lazic$^{1}$}
{1: Rice University, 2: Baylor College of Medicine, 3: Wellesley College }
{In this talk, we present preliminary results on the pre-processing and classification of ECG signals based on geometric methods. Specifically, we show the advantages of using optimal transport-based metrics for reducing segmentation errors of ECG signals. Moreover, we provide evidence that indicates that optimal transport-based clustering methods improve the classification performance over Euclidean-based approaches for the task detection of Junctional Ectopic Tachycardia in postoperative children.}


\vspace{1.5ex}
\abs
{Generalized broken ray transforms in tomography}
{Gaik Ambartsoumian$^{1}$ and Mohammad J. Latifi$^{2}$}
{1: University of Texas at Arlington, 2: Dartmouth College}
{Mathematical models of various imaging modalities are based on integral transforms mapping a function (representing the image) to its integrals along specific families of curves or surfaces. Those integrals are generated by external measurements of physical signals, which are sent into the imaging object, get modified as they pass through its medium and are captured by sensors after exiting the object. The mathematical task of image reconstruction is then equivalent to recovering the image function from the appropriate family of its integrals, i.e. inverting the corresponding integral transform (often called a generalized Radon transform).  A classic example is computerized tomography (CT), where the measurements of reduced intensity of X-rays that have passed though the body correspond to the X-ray transform of the attenuation coefficient of the medium. Image reconstruction in CT is achieved through inversion of the X-ray transform. In this talk, we will discuss several novel imaging techniques using scattered particles, which lead to the study of generalized Radon transforms integrating along trajectories and surfaces containing a ``vertex''.  The relevant applications include single-scattering X-ray tomography, single-scattering optical tomography, and Compton camera imaging. We will present recent results about injectivity, inversion, stability and other properties of the broken ray transform, conical Radon transform and the star transform.}


\vspace{1.5ex}
\abs
{Estimation of aortic valve interstitial contractile behaviors using an inverse finite element approach}
{Alex Khang and Michael S. Sacks}
{University of Texas at Austin}
{Aortic valve interstitial cells (AVICs) reside within the leaflet tissues of the aortic valve and function to replenish, restore, and remodel extracellular matrix components. AVIC contractility is brought about through the contractile properties of the underlying stress fibers and plays a crucial role in processes such as wound healing and mechanotransduction. Currently, it is technically challenging to directly investigate AVIC contractile behaviors within the dense leaflet tissues. As a result, optically clear poly (ethylene glycol) (PEG) hydrogel matrices have been used to study AVIC contractility through means of 3D traction force microscopy (3DTFM). However, the stiffness of the hydrogel material within the vicinity of the AVIC is difficult to measure directly and is further confounded by the remodeling activity of the AVIC. Ambiguity in the local hydrogel mechanical properties can lead to large errors in computed cellular tractions. Herein, we developed an inverse computational approach to estimate AVIC induced remodeling of the hydrogel material. The capabilities of the model were validated with a ground truth data set generated via a test problem comprised of an experimentally measured AVIC geometry and a prescribed modulus field containing unmodified, stiffened, and degraded regions. The inverse model was able to estimate the ground truth data set with high accuracy. When applied to AVICs assessed via 3DTFM, the model estimated regions of significant stiffening and degradation local to the AVIC. We observed that stiffening was largely localized at AVIC protrusions and was likely a result of collagen deposition as confirmed by immunostaining for collagen type 1. Degradation was more spatially uniform and present in regions further away from the AVIC surface and likely a result of enzymatic activity. Our results indicate that AVICs substantially modify the local hydrogel mechanics, which were successfully quantified by our computational model. Looking forward, the established approach will allow for more accurate computation of AVIC contractile force levels and lead to elucidation of stress fiber properties.}


\vspace{1.5ex}
\abs
{Control of stochastic signaling pathways in esophageal cancer}
{Souvik Roy, Zui Pan and Zain Khan}
{University of Texas at Arlington}
{In this talk, we present a new framework for controlling aberrant signaling pathways in esophageal cancer. The dynamics of signaling pathways is given by a stochastic process that models the randomness present in the system. The stochastic dynamics is then represented by the Fokker-Planck (FP) partial differential equation that governs the evolution of the associated probability density function. We solve a FP feedback control problem to determine the optimal combination therapies for controlling the signaling pathway states. Finally, we demonstrate the efficiency of the proposed framework through numerical results with combination drugs. This work was funded by the National Science Foundation (Award number: DMS 2212938) and the Interdisciplinary Research Program (Award number: 2021-772)}


\vspace{1.5ex}
\abs
{Modeling supraventricular tachycardia using dynamic computer-generated left atrium}
{Bryant Wyatt, Avery Campbell, Gavin McIntosh, and Melanie Little}
{Tarleton State University}
{Supraventricular Tachycardia (SVT) is when the heart’s upper chambers beat either too quickly or out of rhythm with the heart’s lower chambers. This out-of-step heart beating is a leading cause of strokes, heart attacks, and heart failure. The most successful treatment for SVT is catheter ablation, a process where an electrophysiologist (EP) maps the heart to find areas with abnormal electrical activity. The EP then runs a catheter into the heart to burn the abnormal area, blocking the electrical signals. Much is not known about what triggers SVT and where to place scar tissue for optimal patient outcomes. We have produced a dynamic model of the left atrium accelerated on NVIDIA GPUs. An interface will allow researchers to insert ectopic signals into the simulated atria and ablate sections of the atria allowing them to rapidly gain insight into what causes SVTs and how to terminate them.}


\vspace{1.5ex}
\abs
{A hemodynamic comparison of single ventricle patients}
{Alyssa M. Taylor-LaPole$^{1}$, Mitchel J. Colebank$^{2}$, Justin D. Weigand$^{3,4}$, Mette S. Olufsen$^{1}$, Charles Puelz$^{3,4}$}
{1: NC State University, 2: University of California, Irvine, 3: Baylor College of Medicine, 4: Texas Children’s Hospital }
{Hypoplastic left heart syndrome (HLHS) is a congenital heart disease that affects about 1,025 infants in the US each year. HLHS patients are born with an underdeveloped aorta and left heart, receiving a series of three surgeries to create a univentricular circulatory system called the Fontan circuit. Patients typically survive into early adulthood but suffer from reduced cardiac output leading to insufficient cerebral and gut perfusion. Currently, clinical imaging data of the neck and chest vasculature is used to assess patients, but it is difficult to use imaging data to assess deficiencies outside of the imaged region. Data from patients used in this paper include three-dimensional, magnetic resonance angiograms (MRA), time-resolved phase-contrast cardiac magnetic resonance images (4D-MRI), and sphygmomanometer blood pressure measurements. The 4D-MRI images provide detailed insight into velocity and flow in vessels within the imaged region, but they cannot predict flow in the rest of the body, nor do they provide values of blood pressure. This study combines MRA, 4D-MRI, and pressure data with a 1D fluid dynamics model to predict hemodynamics in the aorta and the peripheral vessels, including the cerebral and gut vasculature. To study the effects of surgical reconstruction of an HLHS aorta, simulations for both HLHS and matched control patients with a native aorta and double outlet right ventricle (DORV) physiology are compared. We also use perfusion plots of the liver and cerebral tissues to investigate differences in flow to these organs. Our results demonstrate the HLHS patient has hypertensive pressures in the brain as well as reduced flow to the gut. Wave-intensity analysis suggests the HLHS patient has an irregular circulatory function during light upright exercise conditions and that predicted wall-shear stresses are lower than normal.}


\vspace{1.5ex}
\abs
{Optimal experimental design for quantitative MRI with MR fingerprinting}
{Bo Zhao, Evan Scope Crafts, and Hengfa Lu}
{University of Texas at Austin}
{Magnetic Resonance (MR) Fingerprinting is a recent breakthrough in quantitative magnetic resonance imaging, which enables the rapid quantification of multiple MR tissue parameter maps in a single imaging experiment. The original MR Fingerprinting experiments feature a randomized encoding strategy, which applies a sequence of random acquisition parameters to probe the spin system. Despite its empirical success, the optimality of this encoding scheme has not been thoroughly examined since the invention of the technique. In this talk, I will present an optimal experimental design framework to characterize and optimize the encoding process of MR Fingerprinting. Specifically, we will present a discrete-time dynamic system to model magnetization evolutions and further utilize a principled estimation-theoretic metric to optimize the acquisition parameters of MR Fingerprinting. The proposed optimal design method enables a substantial improvement of signal-to-noise efficiency of the acquisition process. In addition, our recent computational algorithm utilizing a B-spline based low-dimensional representation significantly improves the computational efficiency of the optimal design technique.}


\vspace{1.5ex}
\abs
{PocketNet: A smaller neural network for medical image analysis}
{Adrian Celaya, Jonas A. Actor, Rajarajeswari Muthusivarajan, Evan
Gates, Caroline Chung, Dawid Schellingerhout, Beatrice Riviere, and
David Fuentes}
{Rice University and MD Anderson Cancer Center}
{Medical imaging deep learning models are often large and complex,
requiring specialized hardware to train and evaluate these models. To
address such issues, we propose the PocketNet paradigm to reduce the
size of deep learning models by throttling the growth of the number of
channels in convolutional neural networks. We demonstrate that, for a
range of segmentation and classification tasks, PocketNet
architectures produce results comparable to that of conventional
neural networks while reducing the number of parameters by multiple
orders of magnitude, using up to 90\% less GPU memory, and speeding up
training times by up to 40\%, thereby allowing such models to be
trained and deployed in resource-constrained settings.}


\vspace{1.5ex}
\abs
{Early detection of cardiac arrest in infant with congenital heart defects using convolutional denoising autoencoders}
{Arko Barman, Kunal Rai, Chiraag Kaushik, Frank Yang, Tucker Reinhardt, Andrew Pham, Aneel Damaraju, Mubbasheer Ahmed, Sebastian Acosta, Parag Jain}
{Rice University and Baylor College of Medicine}
{Early detection of cardiac arrest can pave the way to better outcomes for patients suffering from diseases such as hypoplastic left heart syndrome (HLHS), a severe congenital heart defect that, if left untreated, leads to death in 95\% of cases within a few weeks of birth. Even after surgical treatment, these patients continue to be at high risk of sudden cardiac arrests (SCA). The difficulty in early detection of these SCAs is due to subtle irregularities of electrocardiogram (ECG) morphology in such patients, which are often missed by physicians monitoring these ECG signals. To address this problem, we propose the use of a convolutional denoising autoencoder (CDAE) architecture in conjunction with change-point detection, using a novel metric that we denote as Cumulative Reconstruction Error (CuRE). Our proposed pipeline uses 4-lead ECG data to perform early detection of patient-specific cardiac instability. The method is robust to noise due to the use of CDAE architecture and to inter-patient variability since the proposed analysis is patient-specific in nature. Our method achieves a sensitivity of 71\% with an average early detection time of 1.132 +/- 0.286 (95\% confidence interval) hours before cardiac arrest, with instability detection as early as 2.927 hours before cardiac arrest.}


\vspace{1.5ex}
\abs
{Using deep learning and macroscopic imaging of porcine heart valve leaflets to predict uniaxial stress-strain responses}
{Luis Hector Victor, CJ Barberan, Richard G. Baranuik, and Jane Grande-Allen}
{Rice University}
{Heart valves consist of leaflets that normally open and shut, guaranteeing unidirectional flow; however, their proper function is limited by their degradation due to a range of disease processes. For this reason, the study of leaflet mechanics is important for understanding the effect of cardiovascular diseases,and designing prosthetics and treatments for heart valve disease. Although traditional mechanical testing of heart valve leaflets (HVLs) is the standard for evaluating mechanical behavior, it is time-intensive, tedious, technical, and requires specialized and expensive equipment. On the other hand, imaging and deep learning (DL) networks, such as convolutional neural networks (CNNs), are readily available and cost-effective, yet investigators have not leveraged these tools to study the mechanics of HVLs. In this work, we determined the influence of a curated dataset, consisting of uniaxially tensile tested porcine aortic valve (PAV) leaflets, and imaging of their aortic surface, on the ability of a CNN to predict the stress-strain response of the leaflets. We used a third-degree polynomial to fit an individual sample's stress-strain curve, which was used as the ground truth for training. Our findings indicate that our framework is robust against using relatively few samples, excellent at predicting the polynomial coefficients needed for reconstructing the toe and linear regions, and independent of the CNN used for making predictions.}

\end{addmargin}
