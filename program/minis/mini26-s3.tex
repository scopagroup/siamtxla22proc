\mini
{mini26}
{Scientific Deep Learning}
{Organizers: Hai Nguyen, Tan Bui-Thanh \& C. G. Krishnanunni}
{The fast growth in practical applications of deep learning in a range of contexts has fueled a renewed interest in deep learning methods over recent years. Subsequently, scientific deep learning is an emerging discipline that merges scientific computing and deep learning. Whilst scientific computing focuses on large-scale models that are derived from scientific laws describing physical phenomena, deep learning focuses on developing data-driven models which require minimal knowledge and prior assumptions. With the contrast between these two approaches follows different advantages: scientific models are effective at extrapolation and can be fitted with small data and few parameters whereas deep learning models require a significant amount of data and a large number of parameters but are not biased by the validity of prior assumptions. Scientific deep learning endeavors to combine the two disciplines in order to develop models that retain the advantages from their respective disciplines. This mini-symposium collects recent works on scientific deep learning methods covering theories, algorithms, and engineering and sciences applications.}
{Location: CBB 122}

\begin{talks}
\item\talk
{Leveraging Data-driven Surrogates to Enable Efficient Density Estimation of Sparse Observable Data on Low-dimensional Manifolds}
{Tian Yu Yen and Tim Wildey}
{Sandia National Laboratories}
\item\talk
{Layerwise Sparsifying Training and Sequential Learning Strategy for Neural Architecture Adaptation}
{C G Krishnanunni and Tan Bui-Thanh}
{University of Texas at Austin}
\item\talk
{Scalable neural network approximation for high-dimensional inverse problems}
{Jinwoo Go and Peng Chen}
{Georgia Institute of Technology}
\item\talk
{Finite Expression Method for Solving High-Dimensional PDEs}
{Haizhao Yang}
{University of Maryland, College Park}
\end{talks}
\room
