\mini
{mini19}
{Modeling the heart-brain axis and age-related pathology}
{Organizers: Travis Thompson}
{The methodical study of human anatomy dates back to at least the 16th century when the Belgian physician Andreas Vesalius published his seminal work, ``De humani corporis fabrica libri septem''.  We now understand a good deal about the large-scale mechanics and function of the many organs and systems within the human body and clinical progress has greatly extended our life spans.  Extended life spans have led to new concerns, including the need to more fully understand age-related pathologies such as heart and brain diseases.

The heart and brain are central to the study of human physiology and pathology.  Increasing evidence implicates the heart-brain axis in several age-related diseases and disorders, including heart failure, epilepsy, stroke and dementia, among others.  The in-vivo study of the heart and brain is often invasive and impractical.  Mathematical modeling, using numerical methods and medical imaging, provides an alternative means to noninvasively study the heart and brain in humans.

This minisymposium brings together an interdisciplinary community of mathematicians and medical researchers who are designing and using mathematical models, numerical methods, machine learning and imaging techniques to study important topics towards developing an understanding of the heart-brain axis and its relationship to age-related pathology, including: cardiomechanics; circulation; oxygen transport; neural network activity; neuroglia and neurodegenerative diseases.}
{Location: CEMO 101}

\begin{talks}
\item\talk
{Reduced models for solute transport and numerical convergence of solutions of PDEs with line source}
{Beatrice Riviere$^{1}$ and Charles Puelz$^{2}$}
{1: Dept.~of Comp.~and Appl.~Mathematics, Rice University 2: Dept.~of Pediatrics, Baylor College of Medicine}
\item\talk
{Hierarchical Modular Structure of the Drosophila Connectome}
{Alexander B. Kunin$^{1,2}$, Jiahao Guo$^{1}$, Kevin E. Bassler$^{1}$, Xaq Pitkow$^{2,3}$ and Krešimir Josić$^{1}$ }
{1: University of Houston 2: Baylor College of Medicine 3: Rice University}
\item\talk
{A deep learning framework for the automated detection and morphological analysis of GFAP-labeled astrocytes in micrographs}
{Demetrio Labate$^{1}$, Yewen Huang$^{1}$, Anna Kruyer$^{2}$, Sarah Syed$^{1}$, Cihan Kayasandik$^{3}$, Manos Papadakis$^{1}$}
{1: Department of Mathematics, University of Houston 2:Medical University of South Carolina 3:Istanbul Medipol University}
\item\talk
{Robust Incorporation of DTMRI Data in Soft Tissue Modeling}
{Christian Goodbrake$^{1}$, Kenneth Meyer$^{1}$, Jack Hale$^{2}$ and Michael S. Sacks$^{1}$}
{1: Oden Institute for Computational Engineering and Sciences and the Department of Biomedical Engineering, The University of Texas at Austin 2: The University of Luxembourg}
\end{talks}
\room
