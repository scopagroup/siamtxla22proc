\mini
{mini07}
{Spectral theory of Schrodinger operators and related topics}
{Organizers: W. Liu, R. Matos \& F. Yang}
{Spectral properties of Schrodinger operators have been intensely studied in the past decades due to their central relevance to quantum physics and also motivated many mathematical questions of independent interest. For instance, phase transitions in a physical system can often be detected through changes in the spectral types of the modeling operator. Schrodinger operators and techniques developed to understand their dynamics have also been key to the comprehension of transport properties of quasicrystals, crystals with random impurities, nonlinear Schrodinger equations, KDV equations, spin models in the presence of disorder, and many other math physics models.}
{Location: CEMO 105}

\begin{talks}
\item\talk
{Some spectral inequality for Schrödinger equations with power growth potentials.}
{Jiuyi Zhu}
{Louisiana State University}
\item\talk
{1-Dim Half-line Schrödinger Operators with $H^{-1}$ Potentials}
{Xingya Wang}
{Rice University}
\item\talk
{Continuity of the Lyapunov exponent for analytic multi-frequency quasi-periodic cocycles}
{Matthew Taylor Powell}
{UC Irvine}
\item\talk
{On the Irreducibility of Bloch and Fermi Varieties}
{Matthew Faust}
{Texas A\&M University}
\end{talks}
\room
