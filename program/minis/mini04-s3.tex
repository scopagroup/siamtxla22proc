\mini
{mini04}
{Modeling, analysis and numerical simulations involving thin structures}
{Organizers: F. Marazzato, A. Bonito, A. Quaini, M. Olshanskii \& F. Marazzato}
{The last three decades have witnessed the development of powerful algorithms and corresponding numerical analysis leading to efficient approximations of the location and behavior of thin structures. Novel methods and modeling techniques have joined the more traditional front tracking techniques and in synergy with the development of ever more powerful and versatile computers, the simulation and understanding of rather complex phenomena are achievable.\\
This mini-symposium gathers experts in the numerical simulation of thin structures with a particular focus on geometric partial differential equations for their technical complexity and practical relevance.}
{Location: CBB 110}

\begin{talks}
\item\talk
{Continuum field theory for the deformations of planar kirigami}
{Paul Plucinsky$^1$, Ian Tobasco$^2$, Yue Zheng$^3$, and Paolo Celli$^4$}
{1: University of Southern California, 2: University of Illinois Chicago, 3: University of Massachusetts Amherst, 4: Stony Brook University}
\item\talk
{Numerical Approximations of Origami with Curved Creases}
{Andrea Bonito}
{Texas A\&M University}
\item\talk
{The Poisson coefficient of zigzag sums}
{Hussein Nassar$^{1}$ and Arthur Leb\'ee$^{2}$}
{1: University of Missouri, 2: \'Ecole des Ponts}
\item\talk
{Computation of Miura Surfaces}
{Frederic Marazzato}
{Louisiana State university}
\end{talks}
\room
