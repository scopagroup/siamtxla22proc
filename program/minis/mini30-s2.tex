\mini
{mini30}
{Data-driven and Nonlinear Model Reduction Methods for Physical Sciences and Engineering}
{Organizers: Rudy Geelen \& Shane McQuarrie}
{The rapidly increasing demand for computer simulations of complex physical, chemical, and other processes poses significant challenges and opportunities for computational scientists and engineers. Despite the remarkable rise of available computer resources and computing technology, the need for model order reduction to cope with highly complex problems is an ever-present reality. While the field of model reduction is relatively mature for linear systems, reducing nonlinear and parametric systems remains an active area of research, especially for large-scale problems. This mini-symposium highlights recent theoretical and algorithmic advancements in data-driven methods and related machine learning techniques for nonlinear model reduction in scientific applications. Such advancements include nonlinear dimensionality reduction, operator learning and inference, scientific machine learning, and physics-informed learning of dynamical systems.}
{Location: CEMO 109}

\begin{talks}
\item\talk
{Why are deep learning-based models of geophysical turbulence long-term unstable?}
{Ashesh Chattopadhyay$^1$ and Pedram Hassanzadeh$^1$}
{1: Rice University}
\item\talk
{Smoothness and Sensitivity of Principal Subspace-valued Map}
{Ruda Zhang$^{1}$}
{1: University of Houston}
\item\talk
{Adaptive planning for digital twins}
{Marco Tezzele$^{1}$ and Karen Willcox$^{1}$}
{1: University of Texas at Austin}
\end{talks}
\room
