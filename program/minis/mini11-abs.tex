\label{mini11}

\miniabs
{Hamiltonian and integrable systems}
{Organizers: B. Feng \& T. Ohsawa}
{Many mathematical models arising from physics and engineering are Hamiltonian systems, and in some special cases, also integrable systems as well. These systems can be studied by analytical, algebraic, and geometric approaches. The purpose of this minisymposium is to bring a group of researchers of Hamiltonian and integrable systems with diverse backgrounds to present their recent findings and exchange ideas for further developments.}

\begin{addmargin}[2em]{0em}
\vspace{2ex}
\abs
{Regularity and Lipschitz optimal transport metric for scalar integrable systems with cusp singularity}
{Geng Chen}
{University of Kansas}
{The talk is concerned with some classes of scalar integrable systems with cusp singularity, such as the Camassa-Holm equation and Novikov equation. It is known that the equations determine a unique flow of conservative solutions within the natural energy space $H^1(R)$. However, the solution flow is not Lipschitz continuous w.r.t. the $H^1$ distance. We will discuss the regularity of solution and the optimal transport metric render Lipschitz continuous dependence.}


\vspace{1.5ex}
\abs
{Semi-discrete Camassa-Holm and modified Camassa-Holm equations and their connection with the discrete KP equation}
{Baofeng Feng}
{University of Texas Rio Grande Valley}
{We will build up a deformed KP-Toda hierarchy from discrete KP equation via Miwa transformations. Then based on the set of bilinear equations and their B{\"a}cklund transformations, we construct semi-discrete Camassa-Holm and modified Camassa-Holm equations.}


\vspace{1.5ex}
\abs
{Crack Problem Under Strain Gradient Elasticity Of Bi-Helmholtz Type}
{Youn-Sha Chan}
{University of Houston-Downtown}
{A crack problem is solved under a higher order of strain gradient elasticity theory of bi-Helmholtz type, and it leads to a linear partial differential equation (PDE) of sixth order. The sixth order PDE can be viewed as a composition of two differential operators: a second order Navier operator coming from the classical linear elasticity theory, and a fourth order bi-Helmholtz operator due to the higher order strain gradient elasticity. The bi-Helmholtz operator consists of two length scales, $l_1$ and $l_2$. The sixth order PDE of the mode III crack problem is transformed to a hypersingular integral equation by the Fourier transform, and the corresponding integral equation is discretized by using the collocation method and a Chebyshev polynomial expansion. The numerical results include displacement profiles, strain, and stress fields under various combinations of $l_1$ and $l_2$.
}

%\vspace{1.5ex}
\abs
{Alchemy as a quantum control problem: Mathematical prospective}
{Denys I. Bondar}
{Tulane University}
{Using the methods of quantum control, we theoretically unveiled an unexplored flexibility of optics that a shaped laser pulse can drive a quantum system to emit light as if it were an arbitrary different system. This realizes an aspect of the alchemist’s dream to make different elements {\it look alike}, albeit for the duration of a laser pulse. I will review different unusual formulations of quantum control problems and outline some of their properties.}

\vspace{1.5ex}
\abs
{On metriplectic dynamics:  joint Hamiltonian and dissipative dynamics}
{Philip J. Morrison}
{University of Texas at Austin}
{Although an early generalization of Lagrangian mechanics to include dissipation was proposed by Rayleigh (1894) and subsequently various other  frameworks for dissipation were given, e.g., for phase separation in Cahn-Hilliard (1958) and Ricci flows in Hamilton (1982),  here we discuss, metriplectic dynamics (MD), a bracket formalism approach begun by the author (1982) for describing systems that have both Hamiltonian and dissipative parts, which places the laws of thermodynamics in a dynamical systems setting.   The motivation of MD is to describe dissipation in a kind of bracket formalism that complements the nondissipative noncanonical Poisson bracket formalism (flows on Poisson manifolds).  Recent thoughts on the topic will be presented.
}


\vspace{1.5ex}
\abs
{Approximation of semiclassical expectation values by symplectic Gaussian wave packet dynamics}
{Tomoki Ohsawa}
{University of Texas at Dallas}
{This talk concerns an approximation of the expectation values of the position and momentum of the solution to the semiclassical Schr\"odinger equation with a Gaussian as the initial condition. Of particular interest is the approximation obtained by our symplectic/Hamiltonian formulation of the Gaussian wave packet dynamics that introduces a correction term to the conventional formulation using the classical Hamiltonian system by Hagedorn and others. The main result is a proof that our formulation gives a higher-order approximation than the classical formulation does to the expectation value dynamics under certain conditions on the potential function. Specifically, as the semiclassical parameter $\varepsilon$ approaches $0$, our dynamics gives an $O(\varepsilon^{3/2})$ approximation of the expectation value dynamics whereas the classical one gives an $O(\varepsilon)$ approximation.}
\end{addmargin}
