\mini
{mini01}
{Recent advances in large-scale inverse problems: Numerics, theory, and applications}
{Organizer: Alexander Mamonov, Andreas Mang \& Daniel Onofrei}
{Despite formidable advances in recent years, significant challenges remain. In inverse problems, parameters are typically related to indirect measurements by a mathematical model (for example, a PDE) in a highly nonlinear way, resulting in non-convex, non-linear optimization problems. These problems are challenging to solve in an efficient way. We will discuss recent advances in numerical methods and the theory of inverse problems to address these challenges. This minisymposium aims to attract researchers at the forefront of inverse problems, inference, and data science to present their latest work on fast algorithms and theory in inverse problems, and exciting applications.}
{Location}
\begin{talks}
\item\talk
{Carleman Weighted Hilbert Spaces for Coefficient Inverse Problems}
{Michael V. Klibanov}
{Department of Mathematics, University of North Carolina at Charlotte}
% {Coefficient Inverse Problems (CIPs) are both ill-posed and highly nonlinear. These two factors cause the non-convexity of conventional least squares cost functionals, which are constructed for numerical solutions of CIPs. The speaker with coauthors has developed a new approach to numerical solutions of CIPs, called convexification. The convexification constructs globally strictly convex cost functionals for a broad class of CIPs. This functional is defined on a bounded convex set of an arbitrary but fixed diameter in an appropriate Hilbert space, which we call Carleman Weighted Hilbert Space. The weight is the Carleman Weight Function, which is used in the Carleman estimate for a corresponding PDE operator. Uniqueness and existence of the minimizer of such a functional on that set is established. Convergence of minimizers to the true solution of the CIP is proven, provided that the noise in the data tends to zero. Many numerical examples, including ones with experimentally collected data, confirm the theory.\\
% Some of these results will be presented in my talk.\\
% Main contributors are (in the alphabetical order): Vo Khoa, Thuy Le and Loc Nguyen.}
% % Preferred date: Saturday, November 5
%
%
\item\talk
{New sampling indicator functions for stable imaging of photonic crystals}
{Dinh-Liem Nguyen}
{Kansas State University}
% {This talk is concerned with the inverse problem of determining the shape of penetrable periodic scatterers using electromagnetic waves. This inverse problem arises from the imaging of photonic crystals using electromagnetic inverse scattering. We develop a sampling method with a novel indicator function for solving this inverse problem. This indicator function is very simple to implement and robust against noise in the data. The resolution and stability analysis of the indicator function is analyzed. Our numerical study shows that the proposed sampling method is more stable than the factorization method and more efficient than the direct or orthogonality sampling method in reconstructing periodic scatterers. This is based on joint work with Kale Stahl and Trung Truong.}
% % Preferred date: Saturday, November 5 or Friday, November 4
%
\item\talk
{Reconstructing a space-dependent source term of the Helmholtz equation via the quasi-reversibility method}
{Loc Hoang Nguyen}
{Department of Mathematics and Statistics, University of North Carolina at Charlotte}
% {Our aim is to solve an important inverse source problem which is the linearization of the well-known inverse scattering problem. We propose to truncate the Fourier series of the solution to the governing equation with respect to a special basis of $L^2$. By this, we obtain a system of linear elliptic equations. Solutions to this system are the Fourier coefficients of the solution to the governing equation. After computing these Fourier coefficients, we can directly find the desired source function. Numerical examples are presented.}
% % Preferred date: Saturday November 5
%
%
\item\talk
{On active control of scalar Helmholtz fields in the presence of impenetrable obstacles and on inverse scattering problems with partial data}
{Lander Besabe \& Daniel Onofrei}
{Department of Mathematics, University of Houston}
% {In this talk, we consider the problem of actively manipulating scalar Helmholtz fields by using a source $D_a$ in the presence of one sound-soft or sound-hard obstacle. We prove the existence of and characterize a necessary input on the boundary $\partial D_a$ such that the radiated field satisfies desired control constraints in near field exterior regions and prescribed far field directions. We will show a numerical scheme to compute these boundary inputs using the method of moments, the addition theorem, Tikhonov regularization, and Laplace spherical functions. Further, we present results in solving inverse scattering problems for impenetrable obstacles given only the measured scattered field due to a fixed frequency and single direction of incidence on a compact exterior region on the near field of the unknown impenetrable scatterer.}
\end{talks}
\room

