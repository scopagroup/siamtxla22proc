\mini
{mini01}
{Recent advances in large-scale inverse problems: Numerics, theory, and applications}
{Organizer: Alexander Mamonov, Andreas Mang \& Daniel Onofrei}
{Despite formidable advances in recent years, significant challenges remain. In inverse problems, parameters are typically related to indirect measurements by a mathematical model (for example, a PDE) in a highly nonlinear way, resulting in non-convex, non-linear optimization problems. These problems are challenging to solve in an efficient way. We will discuss recent advances in numerical methods and the theory of inverse problems to address these challenges. This minisymposium aims to attract researchers at the forefront of inverse problems, inference, and data science to present their latest work on fast algorithms and theory in inverse problems, and exciting applications.}
{Location: CEMO 101}


\begin{talks}
\item\talk
{Carleman Weighted Hilbert Spaces for Coefficient Inverse Problems}
{Michael V. Klibanov}
{Department of Mathematics, University of North Carolina at Charlotte}
\item\talk
{New sampling indicator functions for stable imaging of photonic crystals}
{Dinh-Liem Nguyen}
{Kansas State University}
\item\talk
{Reconstructing a space-dependent source term of the Helmholtz equation via the quasi-reversibility method}
{Loc Hoang Nguyen}
{Department of Mathematics and Statistics, University of North Carolina at Charlotte}
\item\talk
{On active control of scalar Helmholtz fields in the presence of impenetrable obstacles and on inverse scattering problems with partial data}
{Lander Besabe \& Daniel Onofrei}
{Department of Mathematics, University of Houston}
\end{talks}
\room

