\mini
{mini26}
{Scientific Deep Learning}
{Organizers: Hai Nguyen, Tan Bui-Thanh \& C. G. Krishnanunni}
{The fast growth in practical applications of deep learning in a range of contexts has fueled a renewed interest in deep learning methods over recent years. Subsequently, scientific deep learning is an emerging discipline that merges scientific computing and deep learning. Whilst scientific computing focuses on large-scale models that are derived from scientific laws describing physical phenomena, deep learning focuses on developing data-driven models which require minimal knowledge and prior assumptions. With the contrast between these two approaches follows different advantages: scientific models are effective at extrapolation and can be fitted with small data and few parameters whereas deep learning models require a significant amount of data and a large number of parameters but are not biased by the validity of prior assumptions. Scientific deep learning endeavors to combine the two disciplines in order to develop models that retain the advantages from their respective disciplines. This mini-symposium collects recent works on scientific deep learning methods covering theories, algorithms, and engineering and sciences applications.}
{Location: CBB 122}

\begin{talks}
\item\talk
{Finite Expression Method for Solving High-Dimensional PDEs}
{Haizhao$^{1}$}
{1: University of Maryland, College Park}
\item\talk
{Deep Learning of the Evolution of Unknown Systems}
{Victor Churchill$^1$, Dongbin Xiu$^1$}
{1: The Ohio State University}
\item\talk
{Development of a Physics-Informed Machine Learning Method for Pressure Transient Test}
{Daniel Badawi$^1$, Eduardo Gildin$^1$}
{1: Petroleum Engineering Department - Texas A\&M University}
\item\talk
{Learning Data-driven Subgrid-Scale Models: Stability, Extrapolation, and Interpretation}
{Pedram Hassanzadeh$^{1}$, Yifei Guan$^{1}$,  Ashesh Chattopadhyay$^{1}$, Adam Subel$^{1}$ }
{1: Rice University}
\end{talks}
\room
