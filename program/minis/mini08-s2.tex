\mini
{mini08}
{Sparse representation and model reduction for scientific problems}
{Organizers: Y. Lou, J. Hu \& Y. Yang}
{Many scientific problems involve predictive models of multi-scale and multi-physics systems, most of which are extremely complex and computationally expensive to simulate. Yet, a common feature of these problems is that there is often an underlying sparse or reduced representation which can greatly ease the heavy computation burden. This mini-symposium will bring researchers working on these relevant regimes to exchange ideas and encourage collaborations.}
{Location: CBB 118}

\begin{talks}
\item\talk
{Low-rank methods for solving high-dimensional collisional kinetic equations}
{Jingwei Hu}
{University of Washington}
\item\talk
{An Inverse Problem in Mean Field Games from Partial Boundary Measurement}
{Yat Tin Chow$^1$, Samy Wu Fung$^2$, Siting Liu$^3$, Levon Nurbekyan$^3$ and Stanley J. Osher$^3$}
{1: University of California Riverside, 2: Colorado School of Mines, 3: University of California Los Angeles}
\item\talk
{Adaptive deep density approximation for Fokker-Planck equations}
{Xiaoliang Wan}
{Department of Mathematics and Center for Computation and Technology, Louisiana State University, USA}
\item\talk
{Sparse representation of seismic data using chains of destructors}
{Sergey Fomel}
{Jackson School of Geosciences, the University of Texas at Austin}
\end{talks}
\room
