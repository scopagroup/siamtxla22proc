\label{mini23}


\miniabs
{Special Topics in Mathematical Biology}
{Organizers: Summer Atkins \& Hayriye Gulbudak}
{The aim of this session is to provide an opportunity for researchers to meet and discuss recent advances in mathematical biology. Such advances can be new biological findings obtained through the use of tools from the mathematical sciences or  new mathematical ideas and methods that have direct applications to biological investigations. Topics include but are not limited to population dynamics, deterministic or stochastic modeling, optimal control theory, network modeling, and statistical analysis.}


\begin{addmargin}[2em]{0em}
\vspace{2ex}
\abs
{Complexities of the Cytoskeleton: Integration of Scales}
{Keisha Cook}
{School of Mathematical and Statistical Sciences, Clemson University, Clemson, South Carolina, USA}
{Biological systems are traditionally studied as isolated processes (e.g. regulatory pathways, motor protein dynamics, transport of organelles, etc.). Although more recent approaches have been developed to study whole cell dynamics, integrating knowledge across biological levels remains largely unexplored. In experimental processes, we assume that the state of the system is unknown until we sample it. Many scales are necessary to quantify the dynamics of different processes. These may include a magnitude of measurements, multiple detection intensities, or variation in the magnitude of observations. The interconnection between scales, where events happening at one scale are directly influencing events occurring at other scales, can be accomplished using mathematical tools for integration to connect and predict complex biological outcomes. In this work we focus on building inference methods to study the complexity of the cytoskeleton from one scale to another. }


\vspace{1.5ex}
\abs
{Population dynamics under environmental and demographic stochasticity}
{Alexandru Hening {$^{1}$*}, Weiwei Qi{$^{2}$}, Zhongwei Shen{$^{2}$} and Yingfei Yi{$^{2}$}}
{1: Department of Mathematics, Texas A\&M University, College Station, Texas, USA. 2: Department of Mathematics,  University of Alberta, Edmonton, CA}
{This work looks at the long term dynamics of diffusion processes modelling a single species that experiences both demographic and environmental stochasticity. In this setting, the long term dynamics of the population in the absence of demographic stochasticity is determined by the sign of $\Lambda_0$ , the external Lyapunov exponent: $\Lambda_0<$ implies (asymptotic) extinction and $\Lambda_0>$  implies convergence to a unique positive stationary distribution $\mu_0$. If the system is of size $\frac{1}{\epsilon^2}$ for small $\epsilon>0$, the extinction time is finite almost surely. One must therefore analyze the quasi-stationary distribution (QSD) $\mu_\epsilon$ of the system.

We look at what happens when the population size is sent to infinity, i.e., when $\epsilon\to 0$. In contrast to models that only take into account demographic stochasticity, our results demonstrate the significant effect of environmental stochasticity – it turns an exponentially long mean extinction time to a sub-exponential one.}


\vspace{1.5ex}
\abs
{Effect of cross-immunity in a multi-strain cholera model}
{Leah LeJeune{*}, Cameron Browne}
{Department of Mathematics, University of Louisiana at Lafayette, Lafayette, Louisiana, USA}
{Observed in recent cholera outbreaks is the presence of two serotypes, strains of the cholera bacteria that mainly differ in their induced host immunity. Each serotype induces both self-immunity and a degree of cross-immunity to the other strain for some duration. We combine and extend previously studied SIRP and multi-strain models to consider the strain diversity of cholera. We explore various ways of incorporating host immunity into this deterministic multi-strain model, characterizing the dynamics and long-term behavior, particularly in the case of serotype coexistence.}


\vspace{1.5ex}
%Session 2 (so 3-2)
\abs
{Multistationarity and concentration robustness in biochemical reaction networks}
{Badal Joshi$^{1}$, Nidhi Kaihnsa$^{2}$, Tung D. Nguyen$^{3}$*, Anne Shiu$^{3}$}
{1: California State University, San Marcos. 2: Brown University. 3: Texas A$\&$M University}
{Reaction networks are commonly used to model a variety of physical systems ranging from the microscopic world like cell biology and chemistry, to the macroscopic world like epidemiology and evolution biology. Reaction networks arising in applications often exhibit multistationarity-that is, the capacity for two or more steady states. This property is important as it is often associated with the capability for cellular signaling and decision-making. Another biologically relevant property that reaction networks can have is absolute concentration robustness (ACR), which refers to when a steady-state species concentration is maintained even when initial conditions are changed. In this project, our driving motivation is to explore the relationship between the two properties and investigate the prevalence of networks with either property. Our analysis focuses on two ends of the network-size spectrum: small networks with a few species and large networks with many species.}


\vspace{1.5ex}
\abs
{A switch point algorithm applied to a harvesting problem}
{S. Atkins {$^{1}$*},  M. Martcheva{$^{2}$}, and W. Hager{$^{2}$}}
{1: Department of Mathematics, Louisiana State University, Baton Rouge, Louisiana, USA. 2: Department of Mathematics, University of Florida, Gainesville, Florida, USA}
{In this talk we investigate an optimal control problem that seeks an optimal fishing strategy that maximizes the harvesting yield. The underlying question we are studying is whether marine reserves, regions in which fishing is prohibited, can be beneficial to the maximum harvesting yield. The harvesting problem is linear in the control, so parameters of this problem can be set to where the optimal harvesting strategy possesses a singular subarc. In such a scenario, Fuller’s phenomenon or chattering occurs. This is the phenomenon in which the optimal control oscillates infinitely many times over a finite region. Since such a strategy cannot be realistically implemented, we will find suboptimal solutions to this problem by applying a switch point algorithm, a numerical method in which we solve for the optimal control problem with respect to the switches of the control rather than control itself. In using the switch point algorithm, we find some suboptimal harvesting strategies that lead to the incorporation of marine reserves.}

\end{addmargin}




