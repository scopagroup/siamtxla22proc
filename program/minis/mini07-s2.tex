\mini
{mini07}
{Spectral theory of Schrodinger operators and related topics}
{Organizers: W. Liu, R. Matos \& F. Yang}
{Spectral properties of Schrodinger operators have been intensely studied in the past decades due to their central relevance to quantum physics and also motivated many mathematical questions of independent interest. For instance, phase transitions in a physical system can often be detected through changes in the spectral types of the modeling operator. Schrodinger operators and techniques developed to understand their dynamics have also been key to the comprehension of transport properties of quasicrystals, crystals with random impurities, nonlinear Schrodinger equations, KDV equations, spin models in the presence of disorder, and many other math physics models.}
{Location: CEMO 105}

\begin{talks}
\item\talk
{Impediments to diffusion in quantum graphs: geometry-based upper
bounds on the spectral gap}
{Gregory Berkolaiko}
{Texas A\&M University}
\item\talk
{On Pleijel's nodal domain theorem for quantum graphs}
{Matthias Hofmann}
{Texas A\&M University}
\item\talk
{Electrostatic partners (Stieltjes meets Hermite and Padé)}
{Andrei Martinez-Finkelshtein}
{Baylor University}
\item\talk
{Complete Non-Selfadjointness for Schr\"odinger Operators on the Half-Line}
{Christoph Fischbacher}
{Baylor University}
\end{talks}
\room
