\refstepcounter{dummy}\label{mini06}

\miniabs
{Physics-based and data-driven models for engineering applications}
{Organizers: M. Tyagi \& Y. Akkutlu}
{This mini-symposium will cover a wide range of engineering applications using physics-based simulations as well as data-driven machine learning models. Presentation examples would range from subsurface flow phenomena, fluid dynamics, and mathematical foundations of data-science. The proposed mini symposium would also bring together interdisciplinary researchers working on highly relevant societal challenges through their understanding of different aspects of underlying physics and data-driven models.}
\vspace{2ex}


\begin{addmargin}[2em]{0em}

%Session 1
\abs
{Pseudo-spectral methods for the incompressible magnetohydrodynamics equations with variable density}
{Loic Cappanera}
{University of Houston}
{We introduce two numerical methods for the incompressible Navier-Stokes equations with variable density either based on projection methods or artificial compression techniques. These methods are made suitable for pseudo-spectral methods via the use of the momentum, equal to the density times the velocity, as primary unknown and an adequate treatment of the diffusion term. The stiffness matrix of both schemes are time-independent so they can be assembled and preconditioned at initialization. The convergence and stability of both methods are investigated theoretically, and also numerically on various setups using large ratio of density. Applications to magnetohydrodynamics instabilities in liquid metal batteries and aluminum production cells will be presented.}

\vspace{1.5ex}
\abs
{Numerical Investigation of Wave Scour Around Group of Vertical Cylinders }
{Haq Murad Nazari and  Celalettin Ozdemir}
{Louisiana State University}
{In this study, numerical simulations of wave scouring around multiple configurations of vertical piles were performed using SedFOAM, a two-phase Eulerian-Eulerian solver built on OpenFOAM® framework. The kinetic theory of granular flows was used to predict sediment transport in scour configurations. The eddies in two-phase turbulence averaged formulations were resolved using the $k-\omega$ model. The results of the numerical simulations were validated using the experimental results reported in Sumer, B. M and Fredsøe, J. [Wave scour around a group of vertical piles. Journal of Waterway, Port, Coastal, and Ocean Engineering 1998]. The effect of the vortex field on sediment entrainment was rigorously examined. Bed shear stresses from numerical simulations were compared to the experimental results. }

\vspace{1.5ex}
\abs
{Data driven modeling of shear-thinning polymer flooding}
{Prabir Daripa}
{Texas A\&M University}
{Two distinct effects that polymers exhibit are shear thinning and viscoelasticity. The shear thinning effect is important as the polymers used in chemical enhanced oil recovery usually have this property. We propose a data driven approach to incorporate this shear thinning effect through an  effective dynamic viscosity of the shear thinning polysolution. The procedure of viscosity calculation of the polysolution, although based on a very basic power law model, is data driven and uses the values of density, shear rates and power-law coefficients empirically guided by experimental data and local values of concentration of polymer which evolve in time. This method is very general and can be integrated with any method for a Newtonian physics based model of porous media flows. This is exemplified here using a hybrid numerical method developed by Daripa \& Dutta~\cite{DFEMcode,daripa2017modeling,daripa2019convergence}. This method solves a system of coupled elliptic and transport equations modelling Darcy's law based polymer flooding process using a discontinuous finite element method and a modified method of characteristics. Simulations show (i) competing effects of shear thinning and mobility ratio; (ii) injection conditions such as injection rate and injected polymer concentration influence the choice of polymers to optimise cumulative oil recovery; (iii) permeability affects the choice of polymer; (iii) dynamically evolving travelling viscosity waves; and (v)  shallow mixing regions of small scale viscous fingers in homogeneous porous media. This work shows an effective yet easy data driven approach to make design choices of polymers in any given flooding condition. This is joint work with Rohit Mishra.}

\vspace{1.5ex}
\abs
{Physics-Informed Neural Networks for Capacitance-Resistance Models of Reservoir}
{Mayank Tyagi}
{Louisiana State University}
{Reservoir simulators play an important role in the management and optimal production from oil and gas fields. However, the computational costs of detailed simulations can be prohibitively expensive and most certainly not useful for real time decision making. In this presentation, a reduced-order model (ROM) is built using the time-series production data from a real oil and gas field. The Capacitance-Resistance Model (CRM) is chosen here as a reduced-order representation for the reservoir simulator. With the increase in computational power and recent machine learning (ML) approaches, it is apparent that oil and gas industry will eventually adopt the useful models through proper validation. Physics-informed Neural Networks (PINNs) are the neural networks that can enforce the governing equations for the underlying dynamics as a part of building ML models. Results are compared against a detailed reservoir simulation to demonstrate the usefulness of ML models.}


%Session 2
\vspace{1.5ex}
\abs
{Physics-based Data-driven Modeling on Multi-phase Flow in Converging-Diverging Annulus}
{Yitong Hao, Yingjie Tang, Matthew Franchek, Karolos Grigoriadis}
{University of Houston}
{Presented is the development and parameter identification of a multi-phase flow model in the converging-diverging annulus with potential applications of high-pressure-high-temperature (HPHT) annulus flow in subsea blowout preventer (BOP). This work is comprised of: (a) high-fidelity numerical simulations of multi-phase flow with cavitation phenomenon in annulus; (b) reduced order physics-based models of HPHT annulus flow derived from computational studies, targeting evaluation and forecast of the multi-phase flow profile and the cavitation critical status; and (c) experimental validation and calibration in the flow loop test facility using multi-phase flow. This research focuses on characterizing annulus flowing conditions under high pressure events where the potential for fluid cavitation exists. The multi-physics model will be developed using the Buckingham Pi theorem and data analytics processing simulation data, based on which the calculated non-dimensional parameters (Pi) are used as regressors, to develop parametric models estimating the time-based flow profile and the cavitation critical status. This family of parametric models produced from the regressors would be evaluated based on the metrics of mean squared estimation error and R-squared value, in comparison to the previous high-fidelity simulation database. In such process, the model orders will also be selected to balance the trade-offs between model complexities and estimation accuracy. As a result, the data-driven models are developed to describe the cavitation flow profiles in terms of downstream/upstream pressure conditions and non-dimensional geometric parameters of annulus. For industrial applications, the fluids considered in both modeling and experimental validation are currently comprised of water (in both liquid and vapor phases), air, and later with oil and solid particles thus closely emulate drilling fluid rheology.
Using the obtained non-dimensional models, the multi-phase annulus flow profiles can be estimated under both choked and un-choked flow conditions across multi-scale flow channels. With experimental validation and model calibration, the study results can provide engineering guidance for designs of oil and gas drilling systems, the industrial offshore asset integrity analysis, extend the capability and reliability of the academic and industrial research on the subsea fluid system, and ultimately enhance future governing standards to prevent potential failures of subsea drilling systems.    }

\vspace{1.5ex}
\abs
{LES of Tropical Cyclone Winds and Application in Energy Infrastructure Systems}
{Chao Sun}
{Louisiana State University}
{Under climate changing conditions, extreme tropical cyclones have become more frequent and severe, causing extensive damages to critical civil infrastructure systems and residential communities. To estimate extreme wind loading on structures, spectral methods are widely used to generate neutral atmosphere boundary layer winds, which however are limited to describe extreme wind fields that are non-stationary and more turbulent. To overcome this limitation, a high-fidelity high-resolution computational model is developed to simulate hurricane wind field with detailed physics. A large eddy simulation (LES) solver is developed using a sub-grid-scale (SGS) model based on open-source program OpenFOAM. The simulated wind field is validated through comparison with observations. The proposed hurricane boundary layer (HBL) model and a neutral atmosphere boundary layer (ABL) model are compared in tropical storm and category-3 hurricane scenarios. Compared with the HBL model, the ABL model doesn’t consider the mesoscale terms and overestimates the crosswind velocity and the turbulent kinetic energy (TKE) near the ground. As case studies, the HBL model is applied to simulate the performance of offshore wind turbines under different levels of tropical cyclones. An actuator model is used to represent the effect of wind turbines on the wind field. The results reveal the energy harvesting performance under representative wind conditions. Also, the dynamic response of a power transmission line system is modeled to assess its structural safety under high-level tropical cyclones. In summary, the developed LES-based HBL model can capture the main characteristics of tropical cyclone winds and is applicable for modeling critical infrastructure systems exposed to hurricanes at a large scale.}

\vspace{1.5ex}
\abs
{Multiscale Modeling for Clean Energy Transition}
{Kyung Jae Lee and Jiahui You}
{University of Houston}
{Tackling climate change is one of the major challenges facing the U.S. today, and it has led to significant efforts to decarbonize energy use as a way to minimize greenhouse gas emissions. Major ways to achieve this energy transition involve electrifying transportation and increasing renewable energy use in electricity generation. As both electric vehicles and renewable solar–and–wind electricity generation rely on lithium–ion energy storage (i.e., lithium–ion batteries), the demand for lithium has greatly increased during the past decade; it is predicted to escalate along the market growth of electric vehicles and renewable power generation. To enhance and diversify the supply of lithium, we investigate petroleum source rock brines as a sustainable new source of lithium, given that water produced from organic–rich petroleum and natural gas source rocks has been recently revealed as a potential source of substantial amounts of lithium. This opens new pathways to address the natural petroleum source rock systems at various scales. Micro (pore)–scale modeling with Finite Volume Method (FVM), which describes fluid–rock interactions, enables the understanding of mineralization and dissolution of lithium. Macro (basin)–scale modeling with Integrated Finite Discrete Method (IFDM), which describes the release, transport, and accumulation of lithium in petroleum source rock brines, enables the identification of high concentration zones of lithium. Throughout developing the multiscale modeling approach for the new and promising application area, the computational geoscience profession will be able to progress toward a solution to a complex and expensive problem.}

\vspace{1.5ex}
\abs
{Spectral Element Simulation of Particulate Flow and Convective Heat Transfer}
{Don Liu}
{Louisiana Tech University}
{Particulate flows are ubiquitous in nature and engineering. Computational studies have provided some indispensable input and are complementary to lab experiments and theoretical analyses. This study presents a HPC algorithm and implementation in spectral element method for modeling viscous dominant particulate flows involving numerous particles using MPI on distributed memory platforms. In addition, convective and conductive heat transfer flows were simulated as well by adopting the additional energy equation with the thermal dissipation terms included. This will be beneficial to simulate phase-change heat transfer involving melting and solidification etc. Results were verified and validated with relevant discussions.}


\end{addmargin}
