\mini
{mini04}
{Modeling, analysis and numerical simulations involving thin structures}
{Organizers: F. Marazzato, A. Bonito, A. Quaini, M. Olshanskii \& F. Marazzato}
{The last three decades have witnessed the development of powerful algorithms and corresponding numerical analysis leading to efficient approximations of the location and behavior of thin structures. Novel methods and modeling techniques have joined the more traditional front tracking techniques and in synergy with the development of ever more powerful and versatile computers, the simulation and understanding of rather complex phenomena are achievable.\\
This mini-symposium gathers experts in the numerical simulation of thin structures with a particular focus on geometric partial differential equations for their technical complexity and practical relevance.}
{Location: CBB 110}

\begin{talks}
\item\talk
{Convergence of TraceFEM to minimum regularity solutions}
{Lucas Bouck, Ricardo H. Nochetto, Mansur Shakipov$^{1}$ and Vladimir Yushutin$^{2}$}
{1: University of Maryland, 2: Clemson University}
\item\talk
{FEM for surface Navier-Stokes-Cahn-Hilliard equations}
{Yerbol Palzhanov}
{University of Houston}
\item\talk
{Interpolation based immersogeometric analysis with application to Kirchhoff--Love shells}
{Jennifer Fromm$^{1}$, Nils Wunsch$^{2}$,  Ru Xiang $^{1}$, Han Zhao$^{1}$, Kurt Maute$^{2}$, John A. Evans$^{2}$,and David Kamensky$^{1}$}
{1: University of California San Diego, 2: University of Colorado Boulder}
\item\talk
{Navier-Stokes equations on evolving surfaces}
{A. Reusken, P. Brandner, P. Schwering$^{1}$ and M. Olshanskii$^{2}$}
{1: RWTH Aachen University, 2: University of Houston}
\end{talks}
\room
