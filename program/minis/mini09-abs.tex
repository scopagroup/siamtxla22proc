\label{mini09}

\miniabs
{Mathematical physics and graph theory}
{Organizers: R. Han, J. Fillman \& S. Shipman}
{Quantum mechanics is a beautiful combination of mathematical elegance and descriptive insight into the physical world. Many different models have been proposed to understand phenomenons in quantum mechanic, but much of the literature was devoted to various models defined on the standard Zd lattice. Over the recent years, there have been an increasing interest in other lattice structures, which arise from different crystals, e.g. the hexagonal lattice in graphene. This has led to a beautiful combination of mathematical physics and graph theory. The aim of this mini-symposium intends to bring in experts in the TX-LA area to discuss and exchange insights over the recent developments. The proposed mini-symposium will be focused on the following areas of mathematical physics: $\bullet$~multi-layer graphene and other crystal,  structures; $\bullet$~quantum graph models; $\bullet$~reducibility of Fermi surface and algebraic geometry; $\bullet$~critical points of periodic operators.}

\begin{addmargin}[2em]{0em}
\vspace{2ex}
\abs
{Eigenvalue statistics for the disordered Hubbard model within Hartree-Fock theory}
{Rodrigo Matos}
{Texas A\&M University}
{I will present recent progress on the spectral statistics conjecture for the Hubbard model within Hartree-Fock theory. Under weak interactions and for energies in the localization regime which are also Lebesgue points of the density of states, it is shown that a suitable local eigenvalue process converges in distribution to a Poisson process with intensity given by the density of states times Lebesgue measure. If time allows, proof ideas and further research directions will be discussed, including a Minami estimate and its applications.}


\vspace{1.5ex}
\abs
{A spectral statistic of quantum graphs without the semiclassical limit}
{Jon Harrison}
{Baylor University}
{Energy level statistics of quantized chaotic systems are often evaluated in the semiclassical limit via their periodic orbits using the Gutzwiller or related trace formulae. Here, we evaluate a spectral statistic of 4-regular quantum graphs from their periodic orbits without the semiclassical limit. The variance of the n-th coefficient of the characteristic polynomial is determined by the sizes of the sets of distinct primitive periodic orbits with n bonds which have no self-intersections, and the sizes of the sets with a given number of self-intersections which all consist of two sections crossing at a single vertex. Using this we observe the mechanism that connects semiclassical results to the total number of orbits regardless of their structure.  This is joint work with Tori Hudgins at the University of Kansas.}


\vspace{1.5ex}
\abs
{Limit-Periodic Dirac Operators with Thin Spectra}
{Milivoje Lukić}
{Rice University}
{We prove that limit-periodic Dirac operators generically have spectra
of zero Lebesgue measure and that a dense set of them have spectra of
zero Hausdorff dimension. The proof combines ideas of Avila from a
Schr\"odinger setting with a new commutation argument for generating
open spectral gaps. This overcomes an obstacle previously observed in
the literature; namely, in Schr\"odinger-type settings, translation of
the spectral measure corresponds to uniformly small perturbations of
the operator data, but this is not true for Dirac or CMV operators.
The new argument is much more model-independent. To demonstrate this,
we also apply the argument to prove generic zero-measure spectrum for
CMV matrices with limit-periodic Verblunsky coefficients. This is
joint work with Benjamin Eichinger, Jake Fillman, and Ethan Gwaltney.}


\vspace{1.5ex}
\abs
{Analytic tongue boundaries and Cantor spectrum}
{Long Li}
{Rice University}
{We generalized an idea of Broer-Puig-Simo 2003, Puig-Simo 2004, that is, the analyticity of the tongue boundaries for continuous almost Mathieu operators. With this generalized result, we showed that adding a small quasi-periodic perturbation to periodic Verblunsky coefficients could lead to Cantor spectrum of the CMV matrices.}


\vspace{1.5ex}
\abs
{Inverse Uniqueness Result for Hamiltonian System with Measure Coefficients}
{Chunyi Wang}
{Rice University}
{All self-adjoint differential operators of one variable can be written as a linear Hamiltonian system. To be more general, we can introduce Borel measures as coefficients of Hamiltonian systems in order to cover some singular cases, for example, $\delta$ and $\delta’$-interactions in Schr\"odinger operators.

After building the direct spectral theory of Hamiltonian system with measure coefficients, including self-adjoint extension, Weyl-Titchmarsh theory and eigenfunction expansion, I will show the spectral uniqueness result of with the tool of canonical system and de Branges space.}


\vspace{1.5ex}
\abs
{Hyponormal Toeplitz Operators Acting on the Bergman Space}
{Brian Simanek}
{Baylor University}
{We will consider Toeplitz operators with bounded symbol f acting on the Bergman space of the unit disk by multiplication followed by orthogonal projection.  The goal is to understand those symbols f that make the resulting operator hyponormal.  We will pay special attention to the case when the symbol is of the form $f+cg$, where the symbol f yields a hyponormal operator, the symbol g does not yield a hyponormal operator, and c is a complex constant.  Our main results will consider specific choices of f and g that are algebraic functions of z and $\bar{z}$ and describe those constants c for which $f+cg$ yields a hyponormal operator.  This is based on joint work with Trieu Le and Nicole Revilla.}


\vspace{1.5ex}
\abs
{Fermi Isospectrality}
{Frank Sottile}
{Texas A\&M University}
{Given a discrete operator $L$ on a $\mathbb{Z}^d$-periodic graph with a periodic potential,
Floquet theory refines the spectrum of $L$ in terms of the unitary characters of
$\mathbb{Z}^d$.  This gives the Floquet Variety whose level sets at a fixed energy are Fermi
varieties.  A natural question is how much do the resulting Floquet and Fermi
varieties determine the potential and parameters of the graph? Potentials with
the same Floquet (Fermi) varieties are Floquet (Fermi) isospectral.  For the
Schr\"odiniger operator on the grid graph on $\mathbb{Z}^d$ acted on by the free abelian
subgroup $q_1\mathbb{Z} + ... + q_d \mathbb{Z}$ ($q_i$ are pairwise coprime), Kappeler showed that
there are only finitely many potentials Floquet isospectral to a given
potential.  Liu introduced the term Fermi isospectrality and considered it for
separable potentials when $d=2$.

This talk will discuss this history and present continuing work with Faust and
Liu studying Fermi isospectrality for the grid graph when $d=2$.}


\vspace{1.5ex}
\abs
{Quantum Complexity of Permutations}
{Andrew Yu}
{Phillips Academy - Andover}
{Quantum complexity of a unitary measures  the runtime of quantum computers. In this talk, we discuss the complexity of a special type of unitaries, permutations. Let $S_n$ be the symmetric  group of all permutations of  $\{1, \cdots, n\}$ with two generators: the transposition
and the cyclic permutation (denoted by $\sigma$ and $\tau$). The permutations $\{\sigma, \tau, \tau^{-1}\}$ serve as logic gates. We give an explicit construction of  permutations in $S_n$ with quadratic quantum complexity lower bound $\frac{n^2-2n-7}{4}$. We also prove that all permutations in $S_n$ have quadratic  quantum complexity upper bound $3 (n-1)^2$. Finally, we show that almost all permutations in $S_n$ have quadratic quantum complexity lower bound when $n\rightarrow \infty$. The method described in this paper may shed light on the  complexity problem for general unitaries in quantum computation.}
\end{addmargin}
