\label{mini31}

\miniabs
{Mathematical modeling and robust numerical algorithmsin various biological processes}
{Organizers: Shuang Liu}
{Computational and mathematical biology are important new areas in the biological sciences. Recognizing this, I propose a mini–symposium on the topic of computational and mathematical biology entitled mathematical modeling and robust numerical algorithms in various biological processes,which aims to bring the attention of researchers to the broad biological research area and at the same time focuses on a few specific subjects: the growth of brain tumor, cell polarization, cell movement, cell migration, and patterning and tissue formation within the body.Various efficient computational tools and robust numerical algorithms, like powerful machine learning methodologies, phase-field method, and level-set method, have enabled the capability of capturing complicated dynamics in various biological processes.The mini-symposium session consists of 4 invited talks, which are from the perspective of developing mathematical models of complex biological processes, as well as capturing complicated mechanisms with extensive numerical and computational simulations.}

\begin{addmargin}[2em]{0em}
\vspace{2ex}
\abs
{Pattern formation and bistability in a synthetic intercellular genetic toggle}
{B\'arbara de Freitas Magalh\~aes$^{1}$, Gaoyang Fan$^{2}$, Eduardo Sontag$^{3}$, Kre\v simir Josi\'c$^{2}$, and Matthew R. Bennett$^{1}$}
{1:Rice University, 2: University of Houston, 3: Northeastern University}
{Differentiation within multicellular organisms is a complex process that helps to establish spatial patterning and tissue formation within the body. Often, the differentiation of cells is governed by morphogens and intercellular signaling molecules that guide the fate of each cell, frequently using toggle-like regulatory components. Here, we couple a synthetic co-repressive toggle switch with intercellular signaling pathways to create a ``quorum-sensing toggle.'' We show that this circuit not only exhibits population-wide bistability in a well-mixed liquid environment, but also generates patterns of differentiation in colonies grown on agar containing an externally supplied morphogen. We develop a mechanistic mathematical model of the system, to explain how degradation, diffusion, and sequestration of the signaling molecules and inducers determine the observed patterns.}


\vspace{1.5ex}
\abs
{Cell Polarity and Movement with Reaction-Diffusion and Moving Boundary}
{Shuang Liu$^{1}$, Li-Tien Cheng$^{1}$, and Bo Li$^{1}$}
{1:University of California, San Diego}
{Cell polarity and movement are fundamental to many biological functions. Experimental and theoretical studies have indicated that interactions of certain proteins lead to the cell polarization which plays a key role in controlling the cell movement. We study the cell polarity and movement based on a class of biophysical models that consist of reaction-diffusion equations for different proteins and the dynamics of moving cell boundary. Such a moving boundary is often simulated by a phase-field model. We first apply the matched asymptotic analysis to give a rigorous derivation of the sharp- interface model of the cell boundary from a phase-field model. We then develop a robust numerical approach that combines the level-set method to track the sharp boundary of a moving cell and accurate discretization techniques for solving the reaction-diffusion equations on the moving cell region. Our extensive numerical simulations predict the cell polarization under various kinds of stimulus, and capture both the linear and circular trajectories of a moving cell for a long period of time. In particular, we have identified some key parameters controlling different cell trajectories that are less accurately predicted by reduced models. Our work has linked different models and also developed tools that can be adapted for the challenging three-dimensional simulations.}


\vspace{1.5ex}
\abs
{Computational Modeling of Cell Migration in Microfluidic Channel}
{Zengyan Zhang$^{1}$, Yanxiang Zhao$^{2}$, and Jia Zhao$^{1}$}
{1: Utah State University, 2: George Washington University}
{Cell migration plays an important role in various biological processes, such as tissue morphogenesis, wound healing and cancer metastasis, etc.. The mechanisms underlying cellular motility involve generating protrusive patches on the moving interfaces and determining moving directions under the guidance of chemotaxis. In our work, we proposed a phase-field model coupled with a reaction-diffusion system to keep track of the morphology changes of the cell membrane and steer the cell by gradients of attractive chemicals. In this talk, I will introduce our phase-field model for the migration of cell through complex channels and some numerical simulation results will be elaborated.}


\vspace{1.5ex}
\abs
{Parameter Inference in Diffusion-Reaction Models of Glioblastoma Using Physics-Informed Neural Networks}
{Zirui Zhang$^{1}$, Andy Zhu$^{2}$, John Lowengrub$^{1}$}
{1: University of California, Irvine, 2: Carnegie Mellon University}
{Glioblastoma is an aggressive brain tumor that proliferates and infiltrates into the surrounding normal brain tissue. The growth of Glioblastoma is commonly modeled mathematically by diffusion-reaction type partial differential equations (PDEs). These models can be used to predict tumor progression and guide treatment decisions for individual patients. However, this requires parameters and brain anatomies that are patient specific. Inferring patient specific biophysical parameters from medical scans is a very challenging inverse modeling problem because of the lack of temporal data, the complexity of the brain geometry and the need to perform the inference rapidly in order to limit the time between imaging and diagnosis. Physics-informed neural networks (PINNs) have emerged as a new method to solve PDE parameter inference problems efficiently. PINNs embed both the data the PDE into the loss function of the neural networks by automatic differentiation, thus seamlessly integrating the data and the PDE. In this work, we use PINNs to solve the diffusion-reaction PDE model of glioblastoma and infer biophysical parameters from numerical data. The complex brain geometry is handled by the diffuse domain method. We demonstrate the efficiency, accuracy and robustness of our approach.}
\end{addmargin}

