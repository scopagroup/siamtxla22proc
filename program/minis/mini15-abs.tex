\label{mini15}

\miniabs
{}
{Organizers: }
{}

\vspace{2ex}
\abs
{Efficient numerical solutions of Laplace equations with mixed boundary conditions using Steklov eigenfunctions }
{Manki Cho}
{University of Houston at Clear Lake}
{This talk will introduce Steklov eigenproblems on elliptic PDEs. Various ways of finding Steklov eigenpairs have been studied either analytically or numerically. This work will feature some studies of explicit Steklov eigenpairs on polygonal domains. Main results are based on the fact that Steklov eigenfunctions may construct bases of spaces of harmonic functions using only boundary conditions. Solutions of mixed boundary value problems are represented by the series of Steklov eigenfunctions where its coefficients are determined by boundary data of the problems. This idea provides pointwise error estimates in the interior of the region. Moreover, specific quantities of functions such as the central value of the solution or the magnitude of its gradient are accurately estimated by the Steklov expansion method. From the heat conduction problmes to the Dirichlet-to-Neumann operator in electrostatics, its applications will be introduced with numerical results in this talk}


\vspace{1.5ex}
\abs
{Exterior Finite Energy Harmonic Functions}
{Giles Auchmuty$^{1}$  and Qi Han$^{2}$}
{1: University of Houston, 2: Texas A\&M University-San Antonio}
{A space of functions of finite-energy on an exterior domain with a compact, Lipschitz boundary is introduced.
The exterior Poisson's kernel and a reproducing kernel for the harmonic function subspace are described explicitly through solutions to the exterior harmonic Steklov eigen-problems. As an application, an explicit formula for Newtonian capacity is given.}


\vspace{1.5ex}
\abs
{Numerical studies to the Chaplygin gas equation}
{Ling Jin$^{1}$ and Ying Wang$^{2}$}
{1: University of Oklahoma, 2: University of Oklahoma}
{In this talk, we will discuss the numerical solutions to the Riemann problem for Chaplygin gas equation, which is the Euler equations equipped with the state equation p = -1/$\rho$. The spatial discretization is performed using WENO reconstruction and time integration is achieved using TVD RK4. The numerical results confirm high order of accuracy.}


\vspace{1.5ex}
\abs
{Optimal control of Pentadesma fruit harvesting under habitat reduction}
{Benito Chen-Charpentier}
{University of Texas at Arlington}
{The study of the synergetic effects of multiple interacting disturbances
on the dynamical behavior of a biological system has received extensive
attention.  However, the interactions among disturbances are highly
complex and the impacts are still not well understood. In this talk, 
we present a mathematical model based on ordinary differential 
equations to study the effects of exogenous pressures on the 
dynamics of tree ecosystems. Specifically, it incorporates 
the effects of non-lethal harvesting and habitat reduction. 
The resulting model allows the derivation of a general formula 
to determine the rational non-lethal harvesting level and 
habitat size to ensure the sustainability of the plant ecosystem. 
The model will be applied to fruit harvesting of pentadesma trees 
under different habitat sizes.}


