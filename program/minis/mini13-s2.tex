\mini
{mini13}
{Numerical methods and applications for geosciences}
{Organizers: G. S. Jones \& L. Cappanera}
{Computational geosciences often involve complex nonlinear models that couple different physical processes. The study of such problems requires the development of robust and efficient numerical algorithms for high performance computing. In this mini symposium, we focus on numerical methods applied to phenomena arising in geosciences such as reservoir simulation, multiphase flow, water waves, etc. Speakers will present their recent research developments and applications in computational geosciences which includes numerical modeling, numerical analysis, and other computational aspects.}
{Location: CEMO 109}

\begin{talks}
\item\talk
{Multiscale Modeling for Clean Energy Transition}
{Kyung Jae Lee and Jiahui You}
{University of Houston}
\item\talk
{A sequential discontinuous Galerkin method for three-phase flows in porous media}
{Giselle Sosa Jones$^{1}$, Loic Cappanera$^{2}$, and Beatrice Riviere$^{3}$}
{1: Oakland University, 2: University of Houston, 3: Rice University}
\item\talk
{Numerical solution of two-phase poroelasticity equations}
{Beatrice Riviere$^{1}$ and Boqian Shen$^{2}$}
{1: Rice University, 2: KAUST}
\item\talk
{Applications of Space-Time Methods to Multiphase Flow in Porous Media  :  Channel Flow and  Snap Shot Selection for Deep-Learning Reduced-Order Models }
{Mary Wheeler$^{1}$, Thamer  Abbas Alsulaimani$^{2}$ and Hanyu Li$^{3}$}
{1: University of Texas at Austin, 2: Aramco, 3: Lawrence Livermore National Laboratory}
\end{talks}
\room
