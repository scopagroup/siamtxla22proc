\label{mini22}

\miniabs
{}
{Organizers: }
{}

\vspace{2ex}
\abs
{Pure Quartic Solitons in Novel Laser Designs}
{Sabrina Hetzel}
{Southern Methodist University}
{Pure quartic optical solitons are a special class of solitons that arise from the interaction of fourth order dispersion and the self-phase modulation due to the Kerr effect. We adapt the Lugiato-Lefever model to include purely fourth order dispersion instead of second order dispersion to consider this special class of solitons in laser systems. Investigation is done in the dynamics, bifurcation structure and stability of single and double pulse solutions. Numerical simulations are done to provide evidence of different kinds of pulse generation for noisy initial conditions. Linear stability analysis is then conducted on these solutions to provide evidence of stable and unstable double pulse solutions. Pure quartic solitons have a unique characteristic compared to that of conventional solitons (second order dispersion), namely oscillatory tails. By looking at the interaction of individual solitons through the overlap of the tails, we locate the fixed distance between them that is related to the wavelength of these oscillations.}


\vspace{1.5ex}
\abs
{Stability of {G}inzburg-{L}andau Solitons via {F}redholm determinants of a {G}reen's operator} 
{Erika Gallo$^{1},$ John Zweck$^{1},$ and Yuri Latushkin$^{2}$}
{1: University of Texas at Dallas, 2: University of Missouri} 
{Using the symmetric split-step method, we locate a soliton solution of the cubic-quintic Complex {G}inzburg-{L}andau Equation. Linearizing the {CGLE} about such a solution,  we classify the soliton’s stability by computing eigenvalues of the linearized differential operator. We do this using the 2-modified {F}redholm determinant of the associated {G}reen’s kernel, an approach which is complementary to {E}vans function methods for solitons. We adapt a method of {B}ornemann to numerically approximate the {F}redholm determinant by a matrix determinant, and we quantify the error in this approximation.}


\vspace{1.5ex}
\abs
{Bright and Dark Multi-Solitons in a Fourth-Order Nonlinear Schr\"odinger Equation}
{Ross Parker, Alejandro Aceves}
{Southern Methodist University}
{We consider the existence and spectral stability of multi-pulse solitary wave solutions to a nonlinear Schr\"odinger equation which incorporates both fourth and second-order dispersion terms. We do this for both the bright and the dark soliton regimes. We first show that a discrete family of multi-pulse solutions exists, which is characterized by the distances between consecutive copies of the the primary solitary wave. In the bright soliton regime, we then reduce the spectral stability problem to computing the determinant of a matrix which is, to leading order, block diagonal. Under additional assumptions, which can be verified numerically and are sufficient to prove orbital stability of the primary solitary wave, we show that all bright multi-solitons are spectrally unstable. By contrast, using a similar approach in the dark soliton regime, we find that dark multi-solitons can be spectrally neutrally stable. Finally, we show results of numerical spectral computations, which are in good agreement with our analytical results. This is supplemented with numerical timestepping experiments, which are interpreted using our spectral computations.}


\vspace{1.5ex}
\abs
{Continuum limit of 2-d Nonlinear Schr\"dinger Equation}
{Brian Choi, Alejandro Aceves}
{Southern Methodist University}
{Recently, there has been an increased interest in non-local PDEs. While most of the research deals with continuum models, less is known about discrete systems showing global coupling with algebraic decay on the coupling strength. This work considers such a case in a two-dimensional lattice and centers on the question of the validity of a suitable continuum approximation. We prove that the solutions to the discrete Nonlinear Schr\"odinger Equation (DNLSE) with non-local algebraically-decaying coupling converge strongly in $L^2(\mathbb{R}^2)$ to those of the continuum fractional Nonlinear Schr\"odinger Equation (FNLSE), as the discretization parameter tends to zero. The proof relies on sharp dispersive estimates that yield Strichartz estimates that are uniform in the discretization parameter. An explicit computation of the leading term of the oscillatory integral asymptotics is used to show that the best constants of a family of dispersive estimates blow up as the non-locality parameter $\alpha \in (1,2)$ approaches the boundaries.}


