\mini
{mini01}
{Recent advances in large-scale inverse problems: Numerics, theory, and applications}
{Organizer: Alexander Mamonov, Andreas Mang \& Daniel Onofrei}
{Despite formidable advances in recent years, significant challenges remain. In inverse problems, parameters are typically related to indirect measurements by a mathematical model (for example, a PDE) in a highly nonlinear way, resulting in non-convex, non-linear optimization problems. These problems are challenging to solve in an efficient way. We will discuss recent advances in numerical methods and the theory of inverse problems to address these challenges. This minisymposium aims to attract researchers at the forefront of inverse problems, inference, and data science to present their latest work on fast algorithms and theory in inverse problems, and exciting applications.}
{Location: CEMO 101}


\begin{talks}
\item\talk
{\st{Some results on inverse problems to elliptic {PDE}s with solution data and their implications in operator learning}}
{\st{Kui Ren}}
{\st{Department of Applied Physics and Applied Mathematics and Data Science Institute, Columbia University} {\color{p1color}Talk Cancelled: We start with Talk (b) at 9 AM}}
\item\talk
{Estimating the noise level in seismic data while overcoming cycle skipping}
{Susan Minkoff$^{1}$, Huiyi Chen$^{1}$ \& William Symes$^{2}$}
{1: Department of Mathematical Sciences, University of Texas at Dallas, 2:  Department of Computational and Applied Mathematics, Rice University}
\item\talk
{Conductivity imaging from thermal noise}
{Trent DeGiovanni \& Fernando Guevara Vasquez}
{University of Utah}
\item\talk
{{L}ippmann--{S}chwinger--{L}anczos algorithm for inverse scattering problems}
{V. Druskin$^{1}$, S. Moskow$^{2}$ \& M. Zaslavsky$^{3}$}
{1: Department of Mathematical Sciences, Worcester Polytechnic Institute, 2: Department of Mathematics, Drexel University, 3: Schlumberger-Doll Research Center}
\end{talks}
\room
