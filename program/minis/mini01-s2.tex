\mini
{mini01}
{Recent advances in large-scale inverse problems: Numerics, theory, and applications}
{Organizer: Alexander Mamonov, Andreas Mang \& Daniel Onofrei}
{Despite formidable advances in recent years, significant challenges remain. In inverse problems, parameters are typically related to indirect measurements by a mathematical model (for example, a PDE) in a highly nonlinear way, resulting in non-convex, non-linear optimization problems. These problems are challenging to solve in an efficient way. We will discuss recent advances in numerical methods and the theory of inverse problems to address these challenges. This minisymposium aims to attract researchers at the forefront of inverse problems, inference, and data science to present their latest work on fast algorithms and theory in inverse problems, and exciting applications.}
{Location: CEMO 101}

\begin{talks}
\item\talk
{Spatio-temporal quantification of pathological tau spreading in Alzheimer's disease}
{Zheyu Wen, Ali Ghafouri \& George Biros}
{Oden Institute, The University of Texas at Austin}
\item\talk
{TBD}
{Alexandros G. Dimakis}
{Oden Institute, University of Texas at Austin}
\item\talk
{{L}ippmann--{S}chwinger--{L}anczos algorithm for inverse scattering problems}
{V. Druskin$^{1}$, S. Moskow$^{2}$ \& M. Zaslavsky$^{3}$}
{1: Department of Mathematical Sciences, Worcester Polytechnic Institute, 2: Department of Mathematics, Drexel University, 3: Schlumberger-Doll Research Center}
\item\talk
{Conductivity imaging from thermal noise}
{Trent DeGiovanni \& Fernando Guevara Vasquez}
{University of Utah}
\end{talks}

\room

