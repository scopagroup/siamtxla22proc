\mini
{mini01}
{Recent advances in large-scale inverse problems: Numerics, theory, and applications}
{Organizer: Alexander Mamonov, Andreas Mang \& Daniel Onofrei}
{Despite formidable advances in recent years, significant challenges remain. In inverse problems, parameters are typically related to indirect measurements by a mathematical model (for example, a PDE) in a highly nonlinear way, resulting in non-convex, non-linear optimization problems. These problems are challenging to solve in an efficient way. We will discuss recent advances in numerical methods and the theory of inverse problems to address these challenges. This minisymposium aims to attract researchers at the forefront of inverse problems, inference, and data science to present their latest work on fast algorithms and theory in inverse problems, and exciting applications.}
{Location}

\begin{talks}
\item\talk
{Spatio-temporal quantification of pathological tau spreading in Alzheimer's disease}
{Zheyu Wen, Ali Ghafouri \& George Biros}
{Oden Institute, The University of Texas at Austin}
% {Tau lesions (tau) are one of the main biomarkers of Alzheimer's disease (AD). Quantitatively describing how tau spreads in human brains can help with AD diagnosis and prognosis. Tau can be imaged spatially using positron emission tomography (tau-PET). Our goal is to use tau-PET images along with traditional magnetic resonance imaging to learn a spatio-temporal model of tau propagation. In this talk we will discuss the mathematical and computational challenges of the underlying methodology as well as a set of new algorithms that enable quantification and classification of tau spreading. We test our method on a cohort of subjects selected from publically available datasets.}
\item\talk
{TBD}
{Alexandros G. Dimakis}
{Oden Institute, University of Texas at Austin}
% {TBD}


\item\talk
{{L}ippmann--{S}chwinger--{L}anczos algorithm for inverse scattering problems}
{V. Druskin$^{1}$, S. Moskow$^{2}$ \& M. Zaslavsky$^{3}$}
{1: Department of Mathematical Sciences, Worcester Polytechnic Institute, 2: Department of Mathematics, Drexel University, 3: Schlumberger-Doll Research Center}
% {Data-driven reduced order models (ROMs) are combined with the Lippmann[1]Schwinger integral equation to produce a direct nonlinear inversion method. The ROM is viewed as a Galerkin projection and is sparse due to Lanczos orthogonalization. Embedding into the continuous problem, a data-driven internal solution is produced. This internal solution is then used in the Lippmann-Schwinger equation, thus making further iterative updates unnecessary. We show numerical experiments for spectral domain data for which our inversion is far superior to the Born inversion and works as well as when the true internal solution is known.}

\item\talk
{Conductivity imaging from thermal noise}
{Trent DeGiovanni \& Fernando Guevara Vasquez}
{University of Utah}
% {We present a method for imaging the conductivity of a body from measurements of thermal noise currents between the body and the ground. Concretely we show that if the variances of the thermal noise currents are known, the inverse problem can be formulated as a deterministic internal functional inverse problem identical to the one occurring in ultrasound modulated electrical impedance tomography.}
\end{talks}

\room

