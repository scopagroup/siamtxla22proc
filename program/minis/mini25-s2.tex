\mini
{mini25}
{High Order Methods for Computational Hydrodynamics}
{Organizers: Madison Sheridan \& Bennett Clayton}
{The construction of accurate numerical methods is essential for simulating transport phenomena in the field of computational hydrodynamics. The aim of this mini-symposium is to discuss the current state of higher-order accurate methods for simulating realistic hydrodynamics. In particular, we are interested in numerical methods that are higher-order accurate in space and time while maintaining structure preserving properties. The partial differential equations we have in mind are those with dominant hyperbolic features such as: (i) the compressible Euler Equations; (ii) the Shallow Water Equations; (iii) the compressible Navier-Stokes Equations; (iv) radiation transport equations such as the Boltzmann Equation. These equations are often strongly nonlinear and pose numerous challenges when one tries to discretize and solve them numerically.}
{Location: CBB 214}

\begin{talks}
\item\talk
{A high-order explicit Runge-Kutta method for approximating the Shallow Water Equations with sources}
{Eric J. Tovar}
{Los Alamos National Laboratory}
\item\talk
{Modeling Shallow Water Flows through Obstacles with Windows}
{Suncica Canic$^{1}$, Alina Chertock$^{2}$, Shumo Cui$^{3}$, Alexander Kurganov$^{3}$, Xin Liu$^{4}$, Abdolmajid Mohammadian$^{4}$ and Tong Wu$^{5}$}
{1: University of California, Berkeley, and University of Houston, 2: North Carolina State University, 3: Southern University of Science and Technology, 4: University of Ottawa, and 5: The University of Texas at San Antonio}
\item\talk
{A positivity-preserving and conservative high-order flux reconstruction method for the polyatomic Boltzmann–BGK equation}
{Tarik Dzanic$^{1}$}
{1: Texas A\&M University}
\item\talk
{Greedy invariant-domain preserving approximation for hyperbolic system}
{J.-L. Guermond$^1$, M. Maier$^1$, B. Popov$^1$, L. Saavedra$^2$, I. Tomas$^3$}
{1: Department of Mathematics, Texas A\&M University, 3368
	TAMU, College Station, TX 77843, USA.
	2: Departamento de Matem\'atica Aplicada a la Ingenier\'ia
	Aeroespacial, E.T.S.I. Aeron\'autica y del Espacio, Universidad
	Polit\'ecnica de Madrid, 28040 Madrid, Spain.
	3: Department of Mathematics and Statistics,
	Texas Tech University, 2500 Broadway Lubbock,
	TX 79409, USA.}
\end{talks}
\room
