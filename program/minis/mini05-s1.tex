\mini
{mini05}
{Deep learning methods for biomedical image analysis \& modeling}
{Organizers: E. Castillo}
{Topics related to the use of deep learning, either as an ancillary method or as the primary mode of analysis, for biomedical applications will be presented. Discussion areas will include theoretical and computational approaches, as well as clinical translation considerations.}
{Location: CBB 108}

\begin{talks}
\item\talk
{Respecting Variation in Physician Clinical Practice }
{Steve Jiang}
{University of Texas Southwestern Medical Center}
\item\talk
{Application of Deep Learning for real-time non-invasive continuous monitoring for enhanced peripheral oxygen saturation in intensive care unit (ICU) and Operating Room (OR)}
{Sungsoo Kim$^{1,2}$, Sohee Kwon$^{1}$, Alan Bovik$^{2}$, Mia Markey$^{2}$, and Maxime Cannesson$^{1}$}
{1: The University of California Los Angeles, 2: The University of Texas at Austin}
\item\talk
{Imaging based estimation of pathology in a large adult glioma population}
{D. Fuentes, E. Gates, A. Celaya, D. Suki, J. Weinberg, S. Prabhu, D. Schellingerhout}
{The University of Texas MD Anderson Cancer Center}
\item\talk
{Neural Network ’Finite Element’ Based Models for Cardiac Simulations}
{Michael Sacks, Shruti Motiwale, and Christian Goodbrake}
{The University of Texas at Austin}
\end{talks}
\room
