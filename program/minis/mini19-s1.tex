\mini
{mini19}
{Modeling the heart-brain axis and age-related pathology}
{Organizers: Travis Thompson}
{The methodical study of human anatomy dates back to at least the 16th century when the Belgian physician Andreas Vesalius published his seminal work, ``De humani corporis fabrica libri septem''.  We now understand a good deal about the large-scale mechanics and function of the many organs and systems within the human body and clinical progress has greatly extended our life spans.  Extended life spans have led to new concerns, including the need to more fully understand age-related pathologies such as heart and brain diseases.

The heart and brain are central to the study of human physiology and pathology.  Increasing evidence implicates the heart-brain axis in several age-related diseases and disorders, including heart failure, epilepsy, stroke and dementia, among others.  The in-vivo study of the heart and brain is often invasive and impractical.  Mathematical modeling, using numerical methods and medical imaging, provides an alternative means to noninvasively study the heart and brain in humans.

This minisymposium brings together an interdisciplinary community of mathematicians and medical researchers who are designing and using mathematical models, numerical methods, machine learning and imaging techniques to study important topics towards developing an understanding of the heart-brain axis and its relationship to age-related pathology, including: cardiomechanics; circulation; oxygen transport; neural network activity; neuroglia and neurodegenerative diseases.}
{Location: CEMO 101}

\begin{talks}
\item\talk
{Senescence, Sangre, Senility and Simulation: Mathematics at the intersection of the heart and the brain}
{Travis B.~Thompson$^{1}$}
{1: Department of Mathematics and Statistics, Texas Tech University}
\item\talk
{Multi-scale computational models of cardiac and brain tissues}
{Michael S. Sacks$^{1}$ and David S. Li$^{1}$}
{1:Willerson Center for Cardiovascular Modeling and Simulation, Oden Institute and the Department of Biomedical Engineering, University of Texas at Austin }
\item\talk
{Oscillopathies of Brain and Heart: Lessons From the Computational Medicine Clinic}
{David Paydarfar$^{1,2}$}
{1: Dell Medical School, Mulva Clinic for the Neurosciences 2: Oden Institute for Computational Engineering and Sciences,
The University of Texas at Austin}
\end{talks}
\room
