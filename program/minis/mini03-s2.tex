\mini
{mini03}
{Mathematical modeling for biological dynamics}
{Organizers: Zhuolin Qu, Lale Asik \& Xiang-Sheng Wang}
{Mathematical models are powerful tools for understanding and informing complex biological phenomena. In recent years, there has been broad interest in applying mathematics to study a variety of biological fields, such as epidemiology, ecology, and neurology. Mathematical models at different spatial and temporal scales have been developed to focus on population-level dynamics, within-host processes, as well as multiscale dynamics that span several biological scales and capture the feedback between them. The utility of the proposed models requires a solid model formulation from realistic biological phenomena, rigorous analysis using mathematical theories, and accurately solved by numerical methods. This mini-symposium will highlight the new developments in these areas and bring together researchers who work on various models for biological systems from the perspectives of modeling, analysis, and computation. It will serve as a platform to present recent progress, exchange research ideas, extend academic networks, and seek future cooperation.}
{Location: CBB 106}

\begin{talks}
\item\talk
{Global Dynamics of Discrete Mathematical Models of Tuberculosis}
{Saber Elaydi$^{1}$ and Rene Lozi$^{2}$}
{1: Trinity University, 2:  Universite de Provence Côte d'Azur}
\item\talk
{Population Persistence in Stream Networks: Growth Rate and Biomass}
{ T. D. Nguyen$^{1}$*, Y. Wu$^{2}$, A. Veprauskas$^{3}$, T. Tang$^{4}$, Y. Zhou$^{5}$, C. Beckford$^{6}$, B. Chau$^{7}$, X. Chen$^{8}$, B. D. Rouhani$^{9}$, A. Imadh$^{10}$, Y. Wu$^{11}$, Y. Yang$^{12}$, and Z. Shuai$^{13}$ }
{1: Texas A$\&$M University, 2: Middle Tennessee State University, 3: University of Louisiana at Lafayette, 4: San Diego State University, 5: Lafayette College, 6: University of Tennessee, 7: University of Alberta, 8: University of North Carolina at Charlotte, 9: University of Texas at El Paso, 10: University of Central Florida, 11: Georgia State University, 12: The Ohio State University, 13: University of Central Florida}
\item\talk
{Assessing Southern Pine Beetle Infestation Risks Using Agent-Based Modeling}
{John G. Alford$^{1}$, William I. Lutterschmidt$^{1}$, and Abigail Miller$^{2}$}
{1: Sam Houston State University, 2: American Institutes for Research}
\item\talk
{Global dynamics of a cholera model with two nonlocal and delayed transmission mechanisms}
{Xiang-Sheng Wang$^{1}$}
{1: University of Louisiana at Lafayette}
\end{talks}
\room
