\label{mini29}

\miniabs
{Applications and Computation in Algebraic Geometry}
{Organizers: Jordy Lopez Garcia, Josu\'e Tonelli-Cueto \& Thomas Yahl}
{Many applied problems, such as the description of the position of a robotic arm or the equilibria points of a biochemical reaction network, can be formulated in terms of polynomials. Therefore, to deepen our understanding of these problems, we need to develop computational methods that exploit the appearing algebraic and geometric structures. In this mini-symposium, we gather researchers working at the intersection of algebraic geometry, computation, and applications.}


\begin{addmargin}[2em]{0em}
\vspace{2ex}
%Session 1
\abs
{On the geometry of geometric rank}
{Runshi Geng$^{1}$}
{1: Texas A\&M University}
{Geometric Rank of tensors was introduced by Kopparty et al. as a useful tool to study algebraic complexity theory, extremal combinatorics and quantum information theory. In this talk I will introduce Geometric Rank and results from their paper, in particular showing the relation between geometric rank and other ranks of tensors. Then I will present recent results of our study on geometric rank, including the connections between geometric rank and spaces of matrices of bounded rank, and classifications of tensors with geometric rank one, two and three.}


\vspace{1.5ex}
\abs
{Approximate Rank for Real Symmetric Tensors}
{Alperen Ergur$^{1}$}
{1: University of Texas at San Antonio}
{We investigate the effect of an epsilon--room of error tolerance to real symmetric tensor rank. We provide two main results and three algorithms for approximate rank estimation. Our aim is to exploit the specific nature of real geometry, rather than using complex algebraic tools, so we rely on ideas from convexity, additive combinatorics, and probability. This is joint work with Petros Valettas and Jesus Rebollo Bueno. \href{https://arxiv.org/pdf/2207.12529.pdf}{https://arxiv.org/pdf/2207.12529.pdf}}


\vspace{1.5ex}
\abs
{Computing Galois groups of Fano problems}
{Thomas Yahl$^{1}$}
{1: Texas A\&M University}
{A Fano problem consists of enumerating linear spaces of a fixed dimension on a variety, generalizing the classical problem of 27 lines on a cubic surface. Those Fano problems with finitely many linear spaces have an associated Galois group that acts on these linear spaces and controls the complexity of computing them in coordinates via radicals. Galois groups of Fano problems were first studied by Jordan, who considered the Galois group of the problem of 27 lines on a cubic surface. Recently, Hashimoto and Kadets nearly classified all Galois groups of Fano problems by determining them in a special case and by showing that all other Fano problems have Galois group containing the alternating group. We use computational tools to prove that several Fano problems of moderate size have Galois group equal to the symmetric group, each of which were previously unknown. }


\vspace{1.5ex}
\abs
{The Point Scheme and Line Scheme of a Certain Quadratic Quantum $\mathbb{P}^3$}%$\mathbbm{P}^{3}$}
{Jos\'{e} Lozano$^{1}$}
{1: University of Texas at Arlington}
{In the past 35 years, some research in noncommutative algebra has been driven by attempts to classify regular algebras of global dimension four, also known as quantum $\mathbb{P}^{3}$s. In this presentation, we consider a certain quadratic quantum $\mathbb{P}^{3}$ having a finite point scheme and a line scheme that is a union of three distinct lines, with multiplicities 8, 6, and 6, respectively.}


\vspace{1.5ex}
%Session 2
\abs
{The dimension of the semirings of polynomials and convergent power series.}
{Kalina Mincheva$^1$, Netanel Friedenberg$^2$}
{1: Tulane University, 2: Tulane University}
{In this talk I will define the Krull dimension of a (topological) semiring as the number of strict inclusions in a chain of prime congruences. I will compute the dimensions of the polynomial semirings with coefficients in the tropical numbers or in any of its sub-semifields. I will also give bounds for the dimension of the semiring of convergent power series with coefficients in the tropical numbers or any of its sub-semifields. I will explain how these computations align with our geometric intuition (from tropical geometry).}


\vspace{1.5ex}
\abs
{Generating random points uniformly distributed on a parametric curve}
{Josué Tonelli-Cueto$^{1}$, Apostolos Chalkis$^{2}$, and Christina Katsamaki$^{3}$}
{1: The University of Texas at San Antonio 2: Quantagonia, 3: Inria Paris \& IMJ-PRG}
{Given a polynomial parametric curve $\gamma:[-1,1]\rightarrow \mathbb{R}^n$, we want to sample random points $\mathfrak{x}\in\gamma([-1,1])$ uniformly distributed with respect the arc-length. However, even assuming that we can sample random points uniformly in $[-1,1]$ and perform exact arithmetic operations, sampling such points is not an easy task. In this talk, we discuss an algorithm for sampling points approximately uniformly by paying special attention to controlling the error (with respect to the total variation distance).}


\vspace{1.5ex}
\abs
{Algebraic, Geometric, and Combinatorial Aspects of Unique Model Identification}
{Brandilyn Stigler$^{1}$}
{1: Southern Methodist University}
{Biological data science is a field replete with many substantial data sets from laboratory experiments and myriad diverse methods for analysis and modeling.  Given the abundance of both data and models, there is a growing need to group data sets to reveal salient features of the data and ultimately of the underlying network.   For discrete data, a special class of discrete models called \emph{polynomial dynamical systems} (PDSs) can be used to capture all models which fit the given data from a network with $n$ nodes.  Typically a data set can have a large number of associated models, requiring model selection as a post-processing step.  In parallel experimental design can be utilized as a preprocessing step to minimize the number of resulting models

In this talk we consider two problems related to inferring PDSs from an experimental design point of view.  First, we aim to characterize data sets that uniquely determine a PDS.  More specifically, we study input sets viewed as affine varieties $V$ over a finite field $F$ as well as the corresponding ideal of points $\mathbb I(V)$ so that the associated quotient ring $F[x_1,\ldots,x_n]/\mathbb I(V)$ has a unique basis (up to scalar multiple) as a vector space over $F$.  Here we connect geometric properties of $V$ with algebraic aspects of Gr\"obner bases for $\mathbb I(V)$.

Next we relax the condition of requiring a unique PDS and focus on identifying data sets with a unique \emph{wiring diagram}, that is a directed graph representing the connections in the network.  In fact we show how to select minimal sets of edges using only the input data, extending previous results. We present algebraic conditions on $\mathbb I(V)$ that guarantee that there is a \emph{unique} minimal set.  We also provide a geometric condition on the data that guarantees the existence of \emph{multiple} minimal sets.

An outcome of this joint work with E. Dimitrova, C. Fredrickson, N. Rondoni, and A. Veliz-Cuba is a much-reduced number of models

to validate experimentally since we connect sparsest model selection with optimal data selection.}


\vspace{1.5ex}
\abs
{Identifiability of Cycle Compartmental Models With Two Leaks and Identifiability of Submodels}
{Dessauer, Paul R$^1$, Tanisha Grimsley$^2$, Jose Lopez$^3$}
{1. Department of Mathematics, University of Texas El Paso, El Paso, Texas 2. Department of Mathematics, Juniata College, Huntingdon, Pennsylvania 3. Department of Mathematics, California State University - Fresno, Fresno, California}
{Directed cycle models are comprised of compartments, inputs, outputs,
leaks, and edges to model how a substance flows within a network. These models are
said to be identifiable if and only if the rank of the Jacobian matrix for the coefficient
map of the edges and leaks is equal to the number of compartments plus the number of
leaks. We proved that a submodels coefficient map is obtainable through transformations done on the coefficient map of its parent model. Furthermore, for models comprised of two leaks, the identifiability of a parent model does not always imply the identifiability of the submodel; and we conjecture which cases do preserve identifiability. Additionally, we wrote a program that determines identifiability for any model with one output; using this program, a database was constructed with the identifiability status of all two leak models with three, four, five, and six compartments. With this database, we conjecture that all cycle compartmental models with two leaks and one output are identifiable if and only if the output location is between the leak locations.}

\end{addmargin}




