\bigskip
{\bf A hands-on tutorial for data-driven Full Waveform Inversion}\label{tutorial}

\bigskip
{\bf Description:} Understanding subsurface velocity structures is critical to a myriad of subsurface applications, such as carbon sequestration, reservoir identification, subsurface energy exploration, earthquake early warning, etc. They can be reconstructed from seismic data with full waveform inversion (FWI), which is governed by partial differential equations (PDEs).\\
Data-driven FWI leverages neural networks to learn the inverse mapping from seismic data to velocity maps. In this tutorial, we will walk through the anatomy of two data-driven FWI networks: InversionNet and VelocityGAN. The codes written in Pytorch are available on GitHub and examples of the Jupyter notebook will be given for attendees to get hands-on practice. The attendees would run experiments and make modifications to see how they are trained with an open-access seismic FWI dataset, OpenFWI. The broader goal of this tutorial is to provide a deeper understanding of how machine learning methods are implemented in solving scientific inverse problems.\bigskip

{\bf Dr. Youzuo Lin}\\
Team Leader and Staff Scientist, Sensors and Signatures Team\\
Geophysics Focus Lead, Center for Space and Earth Sciences\\
Earth and Environmental Sciences\\
Los Alamos National Laboratory\bigskip

{\bf Dr. Shihang Feng}\\
PostDoc\\
Los Alamos National Laboratory
